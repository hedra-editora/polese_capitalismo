
%\textbf{Palavras-Chave:} política identitária, diversidade, empresas, machismo, racismo, homofobia, lucro.

\chapter*{Prefácio}
\addcontentsline{toc}{chapter}{Prefácio, \emph{por João Bernardo}}
\hedramarkboth{Prefácio}{}

\begin{flushright}
\textsc{joão bernardo}
\end{flushright}

\noindent{}Este livro de Pablo Polese promete uma acesa polémica. Ou será que, de
tão escandaloso e difícil de refutar, há"-de ser silenciado?

O desenvolvimento económico e o que em geral se denomina progresso
resultam daquilo a que em termos marxistas podemos chamar aceleração dos
ciclos de mais"-valia relativa, ou seja, a abertura de novos patamares de
exploração da força de trabalho. Pablo Polese explica abundantemente
este conceito no livro, mas desde já adianto que, em palavras correntes,
o aumento de qualificação dos trabalhadores, permitindo"-lhes
dedicarem"-se a actividades mais rentáveis, e a intensificação do tempo
de trabalho, diminuindo pausas e apressando ritmos, tornam possíveis
inovações tecnológicas e ao mesmo tempo estimulam"-nas, desencadeando
novos ciclos de aumento da produtividade. É neste processo que devemos
concentrar a atenção se quisermos compreender a dinâmica social.

Ora, Pablo Polese mostra neste livro que as administrações de empresa
mais inovadoras estão a incorporar as reivindicações e até as formas do
identitarismo, usando"-as como um dos factores de crescimento da
produtividade. Esta integração é fundamentalmente diferente do processo
de assimilação das lutas dos trabalhadores. Neste último caso as lutas
são, numa primeira fase, derrotadas internamente através de uma
paulatina burocratização. A partir de então são como que viradas do
avesso, e só nesta forma desnaturada é que são recuperadas e assimiladas
pelo capitalismo. No processo de integração dos identitarismos, porém,
não se verifica nenhuma inversão interna nem nenhuma deturpação da forma
originária. Os identitarismos são incorporados tal e qual nos mecanismos
da mais"-valia relativa.

Esta incorporação facilita, por um lado, o pleno aproveitamento de
aptidões, tanto entre os trabalhadores como entre os gestores, que antes
se encontravam subestimadas por preconceitos sexuais ou raciais. Por
outro lado, reduzem"-se os motivos de insatisfação e de conflito, o que
contribui para que os trabalhadores executem com mais boa vontade as
suas tarefas e para que os gestores se dediquem com mais zelo à sua
actividade. O capitalismo só é repressivo quando não pode ser outra
coisa, e a paz social é uma condição para o bom funcionamento da
exploração. Por outro lado ainda, na medida em que os identitarismos
estimulam a união entre pessoas invocando afinidades de sexo, de
preferência sexual ou de cor da pele, contribuem poderosamente para
diluir a noção de classe trabalhadora e para criar uma harmonia entre
pessoas pertencentes a classes económica e socialmente opostas. Nestes
termos, a paz social fica reforçada. Finalmente, na era da
transnacionalização do capital, que ultrapassou as velhas fronteiras, o
identitarismo é o sucedâneo do nacionalismo, com todos os efeitos
nocivos do nacionalismo e mais um --- a tendência das presumidas
identidades a multiplicarem"-se sem limites. A convergência de classes,
que antes era concitada pelo nacionalismo para exigir aos trabalhadores
um maior esforço económico, é hoje estimulada pelos identitarismos para
aumentar o entusiasmo de trabalhadores do mesmo sexo ou com as mesmas
preferências sexuais ou com o mesmo tom de pele dos seus patrões.

Tudo somado, o identitarismo surge como um dos factores de aumento da
produtividade no quadro do capitalismo, ou seja, de agravamento da
exploração da força de trabalho mediante a mais"-valia relativa, e ao
desvendar esta função do identitarismo Pablo Polese abre uma perspectiva
de análise com implicações políticas directas. Nas suas palavras, «a
ocupação de espaços dentro do sistema não constrói a sociedade que
queremos, em termos de horizontalidade, justiça e igualdade substantiva.
Pelo contrário, reforça as bases políticas, econômicas, culturais e
ideológicas do sistema hierárquico de exploração e opressão a que
chamamos Capitalismo». E conclui: «Atualmente a lucratividade das
empresas está totalmente articulada ao \emph{politicamente correto}».
Mas que o leitor não tenha receio ou, quem sabe, perca as ilusões. Não
se trata neste livro de lucubrações ideológicas nem de deduções a partir
de velhos textos dos clássicos defuntos. Trata"-se de factos --- números
e relatórios de empresas. Fica assim revelada, por detrás da máscara, a
realidade do identitarismo. Depois de ler a esmagadora quantidade de
factos sistematizados neste livro, parece"-me indiscutível que as
administrações de empresa não operaram uma selecção de identitarismos,
usando uns e rejeitando outros, mas aproveitaram o identitarismo no seu
conjunto. As leitoras e os leitores não encontram aqui um catálogo dos
identitarismos que permita distinguir os bons dos maus, quaisquer que
eles sejam. Encontram a demonstração factual de que os identitarismos,
globalmente considerados, são um dos motores do desenvolvimento do
capitalismo.

A teia de ideologias e práticas políticas que hoje usurpou o nome e o
lugar da antiga esquerda conjuga dois grandes movimentos de opinião,
numa estreita simbiose: a ecologia e o identitarismo. Tal como desde há
muitos anos tenho procurado mostrar em livros e artigos, os
\emph{lobbies} ecologistas pressionam à regressão das condições de vida
e, portanto, ao reforço da mais"-valia absoluta. Contrariamente à
mais"-valia relativa, a mais"-valia absoluta baseia"-se na extensão das
jornadas de trabalho, em vez do melhor aproveitamento de jornadas mais
curtas; na estagnação ou até na regressão tecnológica; e na paralisação
ou declínio da produtividade --- ou seja, na deterioração do nível de
vida. Graças aos ecologistas, e pela primeira vez em duzentos anos,
passou a emanar da mais"-valia absoluta um \emph{glamour} de modernidade.
Mas embora o capitalismo nunca prescinda de formas absolutas de
mais"-valia, só a rapidez do seu desenvolvimento e a abertura de novos
horizontes económicos e sociais, asseguradas pela mais"-valia relativa,
lhe dão a capacidade de ultrapassar crises e contestações. É a
aceleração da mais"-valia relativa que garante ao capitalismo a
capacidade de sobrevivência, e Pablo Polese mostra que enquanto as
discriminações sexuais e raciais estão a ser relegadas para a esfera da
mais"-valia absoluta, a adopção pelos empresários dos temas e das formas
do identitarismo tem sido um factor de dinamização da mais"-valia
relativa. «As políticas de cotas e demais bandeiras inclusivas da
política identitária minimizam a eficácia dos métodos de exploração da
mais"-valia absoluta, forçando os capitalistas a se modernizarem, a
desenvolverem as forças produtivas e, assim, a recolocarem a exploração
prioritariamente sobre as bases da mais"-valia relativa», conclui Polese.

É este o panorama que hoje se apresenta, e nos é apresentado, como
constituindo a esquerda. Que esta esquerda sirva o desenvolvimento do
capitalismo não é um fenómeno novo. No final do século \versal{XIX} e nos
primeiros anos do século \versal{XX} a esquerda hegemónica na Segunda
Internacional desempenhou um papel importante no desenvolvimento da
burocracia e na adequação das formas administrativas às necessidades do
crescimento capitalista. Depois, durante a maior parte do século \versal{XX},
aquela esquerda que assumiu o controle da revolução russa e depois se
exibiu e impôs como modelo de todas as revoluções confundiu a luta
contra o capitalismo com a generalização da propriedade de Estado e
apresentou o comunismo como um capitalismo de Estado. Inaugurou"-se agora
um ciclo equivalente, que este livro de Pablo Polese tem o enorme mérito
de elucidar.



\chapter{Nota do autor}

Foi entre 2014 e 2016 que realizei a pesquisa que deu origem a este
livro. Minha ênfase recaiu em revistas, periódicos e relatórios de
instituições públicas e privadas de Administração e Economia, de
diversos países, incluindo o Brasil. A pretensão original era a de
efetuar um robusto levantamento de dados acerca do modo como as empresas
transnacionais lidam com a ``pauta identitária'', desvendando, assim,
uma das tendências presentes no ``cenário'' capitalista atual. O texto,
então, seria publicado como artigo ou série de artigos, mas por motivos
políticos optei por deixá"-lo ``na gaveta''. Eu tinha a impressão de que
minhas conclusões acerca da relação entre capitalismo e políticas
identitárias seriam recebidas com muita desaprovação no meio militante e
não me sentia suficientemente confiante para enfrentar os debates
públicos que porventura surgiriam. Decidi, portanto, não publicar o
texto enquanto não considerasse que estavam amadurecidas as teses que
nele defendo e a pesquisa que fundamenta minhas conclusões. Nesse
sentido, a redação do texto ficou de lado por alguns anos, enquanto eu
pesquisava mais sobre o tema, num permanente levantamento de notícias,
dados estatísticos e análises acerca da relação entre economia e
mulheres, economia e negros, economia e \textsc{lgbt}s. Trata"-se de um tema
quente não só no Brasil, mas em todo mundo.

A política identitária está na ordem do dia e isso foi percebido não
apenas pelos partidos, sindicatos, movimentos sociais e coletivos
militantes, mas também pelas empresas e instituições empresariais, que
não apenas perceberam como se anteciparam e passaram a oferecer
respostas lucrativas a tais tendências. Comparando as plataformas
ideológicas de partidos de esquerda e as análises empresariais, como por
exemplo, as da revista \emph{The Economist}, podemos concluir que
enquanto os militantes e teóricos de esquerda estão presos no século \textsc{xix}
os gestores das empresas transnacionais estão de olho no século \textsc{xxii}.

Ao abordar a relação entre economia e identitarismo este livro toca em
debates delicados. A relação entre o capitalismo e as formas de luta
contra as opressões de gênero, raça e sexualidade vem sendo objeto de
inúmeras disputas teóricas, ideológicas e políticas dentro e fora da
academia, das organizações militantes e dos espaços de trabalho e lazer.
O tema da identidade e das opressões de gênero e raça está cada vez mais
presente em nosso cotidiano, fazendo com que todos tenham uma opinião e
se posicionem, de uma forma ou de outra, acerca da questão. Você recebe
um panfleto na rua e ele fala em empoderamento feminino, assiste ao
jornal e vê comemorações pela promoção de uma mulher negra a âncora e
protestos pela pouca indicação de negros ao Oscar, acessa a Netflix e lá
estão, cada vez mais, seriados e filmes com protagonistas femininos e
negros. Não deixa de ser ilustrativo, quanto a isso, o fato de que no
momento em que redigia a primeira versão desta \emph{Nota do autor}, em
24 de novembro de 2018, me chegou por e"-mail uma reportagem de
\emph{Intercept Brasil} intitulada ``O que falta para negros aderirem ao
black money?'', com o \emph{lead} ``A população negra consome R\$1,7
trilhão por ano e precisa ver o retorno desse dinheiro''. Nesta
reportagem lemos:

\begin{quote}
Quando a gente liga a \textsc{tv}, nem parece que 54\% do Brasil é negro. Mesmo
que a pesquisa do Instituto Locomotiva mostre que negros movimentam
R\$1,7 trilhão ao ano, somente 2,1\% dos filmes de 2016 foram dirigidos
ou roteirizados por homens negros segundo a Ancine --- nenhuma produção
foi dirigida ou roteirizada por uma mulher negra. Esse mesmo movimento
se repete em outros setores, fazendo com que o dinheiro não retorne para
esses 54\% da população. Existe um vácuo. ``Se 54\% da população aqui é
preta, por que não temos mais pretos em cargos de chefia ou na mídia?
Isso se dá pela construção do negro na sociedade brasileira que precisa
se encaixar em um sistema branco. Os negros ainda estão no lugar de que
precisam de um branco para se dar bem na vida. Mas, se esses 54\% se
organizarem, a gente se levanta muito mais rápido. O dinheiro é mal
gasto e não vem pra gente. Falta a consciência de comprar com os
nossos'', me disse o ator angolano Licínio Januário, que integra a
equipe da ``Tela Preta'' --- uma \textsc{tv} com equipe negra que foi lançada no
último dia 20. A \textsc{tv} é uma construção coletiva de profissionais de
audiovisual negros que buscam novas narrativas de protagonismo preto
tendo como base a filosofia Black Money. ``A ideia é que quem não se
sinta representado quando liga a televisão, vá assistir ao nosso canal.
Estamos colocando em prática um movimento que é muito maior que a gente.
Queremos preparar o terreno para a próxima geração.'' A ideia da
filosofia Black Money é criar uma forma de consumo consciente para que
negros consumam de afroempreendedores (desde roupas até atendimento
médico), assistam a produções negras e empreguem profissionais negros. A
ideia é fazer o dinheiro circular entre os negros. ``Não é algo
segregatório. É uma lógica de consumo interno praticada por judeus,
orientais e até em algumas comunidades periféricas onde só não se dá o
mesmo nome. É uma forma de prestigiar os seus. Aqui nós vivemos o mito
da democracia racial, os norte"-americanos viveram segregação
oficializada pelo estado. O que fez com que eles buscassem alternativas
e se organizassem. Aqui o banco finge que me aceita mas me trava na
porta. A relação econômica também é uma forma de poder e cria uma rede
de proteção. A população negra consome muito, mas consome errado'',
argumenta o sociólogo e produtor cultural carioca Rodrigo França. Os
teatros do centro do Rio de Janeiro tiveram cinco espetáculos negros em
cartaz ao mesmo tempo, no meio deste ano, com plateia lotada e 85\% de
público negro. Dados do Sebrae apontam que mais da metade dos
empreendedores do Brasil são negros --- desses a maior parte são mulheres
--- e já trata os afroempreendedores como uma categoria específica.
Algumas iniciativas se propõem a facilitar os negócios e o consumo. O
aplicativo Kilombu reúne anúncios de serviços e negócios de
profissionais negros. A plataforma Movimento Black Money produz conteúdo
e promove cursos para afroempreendedores. Além da \textsc{reafro}, uma rede de
empreendedores negros e do Instituto Feira Preta, que mapeia os
afroempreendedores. ``Se a gente pensar quantos afroempreendedores
existem e que são poucos os que estão em situação tranquila, veremos que
é preciso um trabalho de conscientização. É algo que só vai acabar
quando tirarmos dos nossos a ideia de que somos todos humanos'',
concluiu Licínio Januário.
\end{quote}

O apelo a que negros ``prestigiem os seus'' é replicado por outros:
mulheres que devem se apoiar em mulheres, \textsc{lgbt}s que devem se fortalecer
entre \textsc{lgbt}s, moradores das periferias que devem apoiar os ``favelados''.
``Elas por elas'', ``nós por nós'' e ``aqui é favela'', são apenas
alguns dos \emph{slogans} identitários atualmente em voga. Essa
estratégia de fortalecimento e de trocas (desde emotivas até econômicas)
``entre os seus'' se apresenta inicialmente como algo progressista e
admirável, na medida em que sujeitos historicamente oprimidos estariam
se unindo em prol do combate conjunto à marginalização e ao racismo,
machismo e lgbtfobia a que são submetidos cotidianamente. Ganhando voz,
portanto, em um mundo que ao longo de séculos os excluiu e silenciou. O
corte identitário das formas de resistência e organização destes
sujeitos possui, no entanto, uma série de elementos contraditórios,
muitas vezes reforçando aquilo que buscam combater. Algumas dessas
contradições serão abordadas mais à frente.

Após a leitura da reportagem de \emph{Intercept Brasil} reproduzida
acima já podemos ter uma ideia da força e abrangência da rede
institucional que articula identitarismo e economia, com os
afroempreendedores, as empreendedoras mulheres, os empreendedores \textsc{lgbt}s
e os empreendedores periféricos já constituindo categorias econômicas
específicas. Em janeiro de 2020 o Banco Goldman Sachs anunciou que só
fará \textsc{ipo} (abertura e oferta inicial de capital na Bolsa de Valores) de
empresas com mulheres na liderança. Trata"-se de uma notícia intrigante,
não? Ao longo do livro vou expor algumas das formas de articulação e
tornar claras algumas das armadilhas e contradições inerentes à forma
como tem sido construída a relação entre identitarismo, empresas e
desenvolvimento econômico, bem como, é claro, o papel ali desempenhado
pelas teorias e organizações que lutam contras as opressões de gênero,
raça e sexualidade.

Depois da primeira redação deste pequeno livro pretendi, sem sucesso,
englobar os dados e análises mais recentes, que confirmam e muitas vezes
tornam mais complexas as questões abordadas, porém uma revisão do
manuscrito me convenceu de que uma profunda atualização não era
necessária para garantir a consistência da análise que empreendo e para
que o livro cumpra aquilo que almeja: fomentar o debate. Em 2018
pareceu"-me, então, que o manuscrito já havia descansado o suficiente e
já poderia vir a público, até porque já haviam surgido, aqui e ali,
estudos que concluem coisas semelhantes às que concluo. Além disso,
surgiram nos últimos anos pesquisas e publicações que apresentam dados e
análises acerca da absorção da diversidade pelo mundo empresarial, como,
apenas a título de exemplo, o livro de Pedro Jaime, \emph{Executivos
Negros: Racismo e Diversidade no mundo empresarial} (2017); o livro
\emph{Mulheres e Poder,} de H. P. Melo e D. Thomé (2018) e o livro
\emph{O lado negro do empreendedorismo: afroempreendedorismo e black
money} (2019), de Maria Angélica Dos Santos. Vale ainda pontuar que após
a redação inicial do livro houve na \emph{The Economist} um debate sobre
se ``As ações afirmativas deveriam ser descartadas?'' (2018) e outros
dois acerca da relação entre economia e \textsc{lgbt}s: ``As empresas devem
trabalhar para promover os direitos das pessoas \textsc{lgbt} de forma ampla, em
vez de se concentrarem apenas em seus próprios funcionários?'' (2016) e
``As empresas devem ser livres para se recusar a disseminar ideias com
as quais discordem (como assar um bolo com uma mensagem pro"-gay)?''
(2018).\footnote{Cf.
  \textless{}\emph{https://debates.economist.com/debate/affirmative-action}\textgreater{};
  \textless{}\emph{https://debates.economist.com/debate/should-businesses-work-advance-lgbt-rights-broader-society-rather-just-their-own-employees}\textgreater{}
  e
  \textless{}\emph{https://debates.economist.com/debate/businesses-should-not-be-compelled-law-endorse-ideas-which-they-disagree-such-baking-cake-pro}\textgreater{}.}

Após 2016 enviei o manuscrito deste livro para alguns pesquisadores e
militantes próximos, que atuam nas lutas negra e feminista, a fim de
colher críticas e sugestões. Isso resultou em algumas revisões de
passagens e inserção de notícias, dados e análises mais recentes. A eles
agradeço o incentivo e especialmente as críticas e sugestões. Espero,
agora, recebê"-las dos novos leitores.

Por fim, um agradecimento especial ao João Bernardo, à Suellen Abreu, ao
Lucas Monteiro (Legume), ao Manuel Nascimento (Manolo), ao Leo La Selva,
à Ingrid Fernandes e aos colegas do grupo de pesquisa do \textsc{ifms}, primeiros
interlocutores que contribuíram com críticas e sugestões quanto à forma
e matéria do livro. Ao João agradeço ainda por ter aceitado,
gentilmente, o convite para escrever o prefácio. Agradeço também ao
Tales Ab'Sáber pela cuidadosa leitura e incentivo à publicação do
material, e ao Paulo Arantes, por me convidar para apresentar o trabalho
no Seminário das Quartas (\textsc{usp}), onde recebi valiosos comentários e duras
críticas. Como de praxe, os acertos e desacertos do livro são de minha
inteira responsabilidade.

\part[racismo, machismo, capitalismo identitário]{\textsc{racismo, machismo,\\ capitalismo identitário}}

\chapter{O capital transnacional e a seleção de gestores}

Nas últimas décadas o capitalismo passou por mudanças no plano da
divisão internacional do trabalho que impõem um olhar crítico à antiga
visão acerca do par conceitual ``centro\,--\,periferia''. Os rumos tomados
pela divisão internacional do trabalho, conforme avançava o
desenvolvimento capitalista, levaram ao esvanecimento da cisão ``norte\,--\,sul'' no plano global, onde o norte representava as economias
avançadas enquanto no sul habitavam as economias subdesenvolvidas etc. A
própria lógica de divisão do trabalho entre ``nações'', se um dia foi
correta, hoje não se sustenta. Em larga medida, pensar criticamente a
questão do ``centro'' e ``periferia'' do sistema capitalista é o mesmo
que tratar das formas de ser do imperialismo hoje, o que exige que
observemos as táticas das empresas transnacionais a fim de se garantirem
enquanto empresas lucrativas e aparelhos de poder.

É com respeito ao caráter das empresas enquanto aparelhos de poder que
buscarei demonstrar os modos como as empresas assimilam as pressões
sociais decorrentes das lutas identitárias, em especial as lutas
feminista e negra. Por meio dos mecanismos de mais"-valia relativa as
empresas se tornam aptas a integrar as demandas dessas lutas. Ao assim
proceder, reforçam suas próprias raízes políticas, ideológicas e
culturais nos locais onde atuam, estreitando os laços econômicos entre
patrões e trabalhadores. Com a dinamização das elites empresariais,
decorrente das pressões das lutas identitárias, ganha novo fôlego o
desenvolvimento capitalista. O texto a seguir trata, portanto, do modo
como o capitalismo absorve as pautas identitárias e converte as lutas
antiopressão (especialmente machismo e o racismo) em algo lucrativo.

A título de exemplificação dos problemas decorrentes da manutenção do
uso do par analítico centro\,--\,periferia sugerimos que se observe a questão
dos investimentos externos diretos (\versal{IED}). O próprio Lenin, um dos
primeiros teóricos do imperialismo, já alertara, no início do século \versal{XX},
para o fato de que a dinâmica de expansão do capitalismo em sua ``fase
imperialista'' deve ser medida não pela exportação de mercadorias, mas
pela exportação de capital, cuja modalidade atual mais importante é o
\versal{IED}.

\begin{quote}
Definem"-se assim os investimentos originários de um país e dirigidos
para outro, que asseguram ao investidor o controle ou, pelo menos, um
interesse duradouro e uma influência decisiva na empresa onde o capital
é aplicado. Considera"-se habitualmente que o investimento externo é
directo quando permite adquirir uma participação superior a 10\% do
capital de uma empresa. (\versal{BERNARDO}, 2011: 1)
\end{quote}

A exportação de mercadorias gera efeitos econômicos reduzidos se
comparados aos efeitos duradouros e a repercussão em múltiplas direções
dos \versal{IED}. Por meio desses investimentos a empresa se reforça
economicamente no âmbito do país de origem mas, ao expandir sua atuação
para outro espaço geográfico, se sustenta também no país que recebeu o
\versal{IED}, portanto pode suceder que com os \versal{IED} a empresa não se reforce no
país de origem e sim no país que recebe o investimento. Isso ocorre
porque os \versal{IED} permitem que se ponha em prática uma \emph{estratégia de
deslocalização}. Por meio dos \versal{IED} as geografias empresariais se tornam
distintas das geografias nacionais e deixam de coincidir com elas.
Veremos que esse modo de operar do capital transnacional traz fortes
implicações sobre a forma como as empresas e governos passarão a
responder às pressões das lutas sociais organizadas em torno de pautas
de combate à discriminação de raça, gênero e sexualidade e, em especial,
aquelas que demandam a integração econômica das ``minorias''.

Essa amplitude dos tentáculos econômicos fortifica política, econômica e
ideologicamente a empresa transnacional. É por meio dos \versal{IED}, e não da
exportação de mercadorias, que as empresas multi e transnacionais tecem
a rede de internacionalização de seus capitais.

Em \emph{The Globally Integrated Enterprise} um dos principais nomes da
\versal{IBM}, Samuel Palmisano, afirmou que foi em meados do século \versal{XIX} que
surgiu o que se pode designar como ``companhia internacional''.
Tratava"-se, fundamentalmente, de empresas que buscavam abrir ou
controlar rotas comerciais internacionais a fim de usá"-las para a
importação de matérias"-primas e exportação de mercadorias. Palmisano
aponta que a ``segunda fase da vida das companhias iniciou"-se em 1914,
com a primeira guerra mundial e o subsequente colapso das economias nos
Estados Unidos e na Europa'', quando a Guerra causa uma interrupção das
rotas comerciais e as políticas protecionistas dos anos 1920 e 1930 se
tornam obstáculos para o comércio internacional, o que leva à formação
das companhias multinacionais. Segundo Palmisano essas empresas eram
``híbridas'': ``Por um lado, adaptaram"-se às barreiras comerciais
desenvolvendo a produção no local. {[}\ldots{}{]} Por outro lado, as
companhias multinacionais prosseguiram num âmbito global algumas
tarefas, nomeadamente a pesquisa e desenvolvimento e o \emph{design} de
produtos'' (\versal{PALMISANO}, 2006: 2).

Na etapa seguinte, a partir da década de 1970, teria se dado a formação
de companhias de fato integradas em âmbito global. Estas empresas
``moldam a estratégia, a gestão e as atividades tendo em vista um novo
objetivo: a integração da produção e a obtenção de valor à escala do
mundo inteiro'', de modo que ``as fronteiras dos países definem cada vez
menos os limites do pensamento e da ação das companhias'' (ibid: 3).
Complementando a análise de Palmisano, João Bernardo acrescentou que ``a
crescente subcontratação das actividades permite às companhias
ultrapassar quaisquer quadros nacionais e converterem"-se em integradoras
de actividades especializadas'' (\versal{BERNARDO}, 2011: 3).

O papel das empresas enquanto \emph{integradoras de atividades
especializadas} se soma a uma lista de papéis estratégicos exercidos
pelos gestores no sentido de articulação e integração global entre as
empresas. Outro elemento característico das transnacionais --- e
importantíssimo para o tema deste livro --- reside no modo de
recrutamento dos gestores:

\begin{quote}
Está morta e enterrada a época do ``fardo do homem branco'', quando
administradores com a nacionalidade da matriz eram metidos em barcos,
depois em aviões, para irem dirigir as filiais no outro lado do mundo,
reprodutores e propagandistas da cultura originária da companhia. Agora,
a tendência para dar um carácter plurinacional e intercultural,
verdadeiramente cosmopolita, às administrações das empresas é um efeito
e uma condição da transnacionalização. (\versal{PALMISANO}, 2006: 4)
\end{quote}

No que diz respeito à esquerda, esse aspecto da transnacionalização do
capital toca diretamente na questão do ganho de expressividade das
políticas identitárias, na medida em que elas adentram o processo de
seleção e disponibilização de sujeitos para um quadro gestorial mais
``colorido''. Este livro buscará demonstrar alguns elementos desta
problemática de implicações profundas para a prática política da
esquerda e em especial para os coletivos e movimentos de luta feminista,
contra o racismo e contra a homofobia (englobando o preconceito contra
\versal{LGBTIQ}: lésbicas, gays, bissexuais, transgênero, travestis/transexuais,
intersexuais e queer/não binários).

\chapter{A infraestrutura social da mais"-valia}

Quando a empresa indiana \emph{Tata Iron} surgiu, em 1907, o alto
funcionário britânico do ramo dos caminhos de ferro, Sir. Frederick
Upcott, observou: ``Você quer dizer que Tata está propondo fazer trilhos
de aço para especificações britânicas? Por quê? Eu me comprometo a comer
cada libra de trilho de aço que eles conseguirem fazer''. Exatamente 100
anos depois a \emph{Tata Steel} adquiria a anglo"-alemã \emph{Corus}, uma
das maiores empresas transnacionais do ramo siderúrgico. Algumas
promessas não chegam a ser pagas.

João Bernardo observou certa vez que para que uma empresa indiana
pudesse se desenvolver em um ramo até então monopolizado pelos
colonizadores do país foi necessária uma enorme luta antirracista dos
indianos, de modo que fossem aceitos como seres humanos iguais aos
britânicos. O capitalismo recuperou essa luta e assimilou"-a, tal como a
dos chineses e dos africanos, e foi graças a isso que adquiriu novo
fôlego com os \versal{BRICS} e a atual reorientação dos centros geográficos
hegemônicos.

O desenvolvimento econômico capacita as empresas mais dinâmicas a
anteciparem os conflitos sociais por meio de concessões materiais aos
trabalhadores, as quais são oferecidas (ou arrancadas) em termos de
maior quantidade de remuneração ou de bens e serviços que entram, no
quadro em sua totalidade, como formas de remuneração direta e indireta.
Essas concessões não se dão \emph{em termos de valor} e sim de
\emph{produtos e serviços}, por isso a classe trabalhadora passa a
receber mais, em termos de salário e demais formas de \emph{remuneração
e ``inputs''}, mas recebe menos em termos de valor. Sendo assim, ao
contrário do que a aparência indica, com estes aumentos salariais a
classe trabalhadora é agora mais explorada que antes, em termos
relativos. Estamos aqui na seara do aumento da produtividade, onde um
número superior de bens de uso e de serviços pode corresponder a uma
quantidade inferior de tempo de trabalho incorporado (valor) nestes
bens. Trata"-se da ação dos mecanismos da mais"-valia relativa, os
mecanismos de aumento da exploração via desenvolvimento das forças
produtivas. Estes mecanismos estão à disposição dos capitalistas em suas
disputas nos conflitos de classe e entre capitais.

Os setores avançados das classes capitalistas sabem como articular os
mecanismos de mais"-valia relativa e as pautas democráticas das lutas
sociais. Aqueles que pretendem erguer barreiras racistas contra os
imigrantes, como Donald Trump ou Ted Cruz, nos Estados Unidos, ou como
os países do \emph{Grupo de Visegrád} (Hungria, Polônia, República Checa
e Eslováquia), na União Europeia, representam não os grandes ou médios
capitalistas, mas uma parte do proletariado dessas nações, a parte menos
qualificada, diga"-se de passagem. Isso se dá por conta de que muitas
vezes a entrada de mão"-de"-obra imigrante rebaixa o nível dos salários e
aumenta a concorrência entre trabalhadores, operando uma seleção que,
obviamente, prejudica os trabalhadores menos qualificados. Não por acaso
foram estes trabalhadores que na Grã"-Bretanha votaram a favor do Brexit
e, na última eleição presidencial da França, em Marine Le Pen. Por outro
lado, seguindo a linha estratégica progressista do principal veículo
formativo da classe gestorial, a revista \emph{The Economist}, a
chanceler alemã Angela Merkel pretende regularizar a entrada de
refugiados na Europa, assim como nos Estados Unidos as Câmaras de
Comércio defendem o fluxo livre dos imigrantes. Atualmente, em todas as
grandes economias encontramos o conflito entre grupos mais voltados ao
protecionismo/bilateralismo e grupos mais afeitos à
globalização/mundialização do capital.

Os mecanismos de mais"-valia relativa, impulsionadores do incremento da
produtividade e, assim, do desenvolvimento econômico, permitem a
recuperação das demandas dos trabalhadores, incorporando"-as de modo
lucrativo. São estes elaborados métodos de assimilação das lutas dos
trabalhadores que garantem a \emph{infraestrutura social} da
mais"-valia relativa, a estruturação social adequada para o livre
desenvolvimento das forças produtivas e seus meios refinados de
incremento da exploração dos trabalhadores. Para compreender
adequadamente o modo como operam os mecanismos de mais"-valia relativa
deve"-se ter sempre em mente que a mais"-valia não é e não se mede apenas
pelos salários e demais formas de remuneração, mas em termos de
asseguramento da continuidade e aprofundamento de um versátil sistema de
exploração do tempo de trabalho. A mais"-valia não diz respeito apenas a
uma equação matemática entre trabalho pago e trabalho não pago; diz
respeito a uma relação social e às condições para que essa relação
social continue operando, depois de passar por mutações e se ajustar às
condições históricas, políticas, ideológicas, econômicas e culturais
impostas no decorrer da luta de classes.

As formas políticas dentro das quais se desdobram as lutas de classes
assentam nos métodos de extração de mais"-valia. Assim, a democracia não
pode ser adequadamente compreendida se não entendermos como o
capitalismo articula a assimilação das lutas e a recuperação das
conquistas dos trabalhadores. Os dispositivos estatais e empresariais de
exploração e de dominação precisam constituir uma hegemonia estável, o
que demanda que os aparelhos de poder tenham \emph{capilaridade} no
tecido social. A hegemonia do capital por sobre os trabalhadores exige
que os aparelhos de poder possuam canais de proliferação e comunicação
desde o topo até a base, desde o amplo e geral até o minucioso e
particular. Os aparelhos de poder precisam operar e operam tanto no
âmbito macro da articulação entre Estados e entre empresas quanto na
microfísica, nos vasos capilares da vida social, e não se compreende
esse elemento da dominação de classe senão por meio da dinâmica própria
aos mecanismos da mais"-valia relativa e seu correlato político, a
democracia.

Em termos marxistas clássicos, portanto, a democracia é a superestrutura
política que corresponde ao que na infraestrutura econômica constitui a
mais"-valia relativa. Não se quer, com isso, dizer que na democracia não
há exploração do trabalho via mecanismos de mais"-valia absoluta, de
aumento do tempo e intensidade da jornada de trabalho em condições
técnicas e organizacionais estáticas, ou que não haja exploração via
mais"-valia relativa em regimes autoritários e ditatoriais. Trata"-se do
acento, o polo hegemônico, a forma política melhor adaptada para dar
livre vazão aos determinantes sistêmicos da forma econômica, e sob o
capitalismo contemporâneo esta reside na democracia e sua forma
econômica mais desenvolvida e dinâmica, a mais"-valia"-relativa.

Vale notar, contudo, que os mecanismos de mais"-valia relativa não
eliminam a extração de mais"-valia absoluta, mas tão somente a deixam em
segundo plano, usando, quando necessário, os mecanismos de incremento da
exploração via força, coerção e violência ou, o que diz respeito mais de
perto ao tema deste livro, utilizando os elementos de discriminação de
gênero, de raça, de sexualidade, de nação, de etnia e cultura enquanto
elementos de rebaixamento geral do salário, muitas vezes violando a lei
do valor (a lei da troca de equivalentes) e potencializando a exploração
de mais"-valia absoluta. Defendo, aqui, que o sistema capitalista não
precisa e não lhe convém manter"-se atrelado aos mecanismos de mais"-valia
absoluta que reproduzem as opressões e discriminações.

\chapter{Machismo, racismo, capitalismo}

Ser mulher numa posição de liderança numa área tecnológica tem sido um
percurso com dificuldades que hoje em dia já está mais generalizado?
Esta foi a pergunta feita à representante do \emph{Facebook} em Portugal
e Espanha, Irene Cano, que respondeu: ``Eu creio que ser mulher hoje em
dia é menos complicado que há 30 anos. Para mim nunca vi que ser mulher
fosse uma dificuldade para me impor nesta área e aqui no Facebook já
estou há nove anos''.\footnote{Cf. \textless{}\emph{https://bit.ly/2y6jhuF}\textgreater{}.}

Para quem compõe o \emph{staff administrativo} das empresas
transnacionais de hoje em dia a resposta de Irene, negando os
costumeiros (e anacrônicos) pesos restritivos impostos pelo machismo à
ascensão das carreiras das mulheres de negócios, não causa espanto. Nas
grandes empresas, o caso da executiva do Facebook é a regra, e não a
exceção, pois as altas esferas das empresas capitalistas mais dinâmicas
já superaram há muito tempo os preconceitos contra mulheres, negros e
homossexuais. Estão nestas empresas os setores capitalistas mais
inteligentes e cosmopolitas, antenados com o modernismo e a
modernização, atentos, digamos assim, aos ares dos novos tempos.
Trata"-se de gestores despojados que ainda que ocasionalmente se mostrem,
em suas vidas privadas, machistas, homofóbicos ou racistas, em suas
práticas laborais não deixam, pois não podem deixar, essas
características influenciarem em suas formas de gestão do grande
capital.

Os gestores encarregados das empresas transnacionais costumam ser
pessoas qualificadas que rapidamente aprendem como recuperar e assimilar
os resultados das lutas das ``minorias'' ativas de negros, amarelos,
mulheres, gays, lésbicas, transexuais etc., enquanto o pequeno capital e
a maioria da classe trabalhadora permanecem muito mais tempo apegados ao
nacionalismo e à reprodução dos preconceitos de raça, gênero e
sexualidade. Este é um dos motivos que levam a que ganhem vitalidade
renovada, \emph{no seio da classe trabalhadora}, as opressões de gênero,
raça e sexualidade, o que leva os setores organizados dos trabalhadores
a terem de empreender lutas de reação a essas opressões,
identificando"-as, muitas vezes, como inerentes ao próprio capitalismo,
uma vez que são historicamente persistentes e ocorrem tanto nas áreas de
desenvolvimento econômico avançado quanto nas periferias.

De fato as discriminações de raça, gênero e sexualidade têm força e
caráter global. Quanto à questão do assédio sexual, por exemplo, os
dados da \versal{ONG} Catalyst apontam que cerca de 50\% das mulheres da União
Europeia denunciaram algum tipo de assédio sexual no local de trabalho.
A Organização Internacional do Trabalho (\versal{OIT}) indica que, no mundo todo,
mais de 50\% das mulheres já foram vítimas de assédio sexual, mas a
maioria não denuncia por falta de provas.\footnote{Há que se ponderar,
  contudo, que à medida que se tornaram um instrumento de chantagem nas
  disputas entre trabalhadores e capitalistas, a própria denúncia de
  assédios --- especialmente quando não se define claramente o que se
  quer dizer com a palavra --- acaba compondo dados não confiáveis. Não
  se quer dizer com isso, é obvio, que o assédio (inclusive sexual) não
  seja uma prática recorrente em empresas, mas o caráter problemático da
  questão transparece de diversos modos. Veja"-se, por exemplo, o curioso
  caso que tem ocorrido em Wall Street após a repercussão das denúncias
  do movimento \#MeToo: ``Os efeitos do movimento MeToo estão a fazer
  com que os homens evitem viajar ao lado de mulheres. Ou, se estiverem
  em reuniões privadas, manterem a porta aberta. Em vez de assistir a
  uma correcção de comportamentos, as mulheres estão a ser excluídas e
  os homens afastam"-se de uma queixa de assédio sexual para serem
  acusados de discriminação com base no género''. Cf. ``Em Wall Street,
  os homens já não querem estar ao lado de mulheres''. Disponível em:
  \textless{}\emph{https://bit.ly/3bJAA3g}\textgreater{}.}

O racismo, a lgbtfobia e o machismo cotidiano podem assumir muitas
formas, inclusive se manifestando como microagressões. Algumas delas
podem ser sutis, como quando alguém pressupõe que um colega de trabalho
ocupa um cargo inferior àquele que de fato exerce. Já outras podem ser
mais explícitas, como quando alguém ofende, assedia ou humilha um colega
de trabalho. Essas formas de desrespeito não recaem igualmente por sobre
todo e qualquer trabalhador, e sim refletem a desigualdade de gênero e
raça, pois as microagressões são quase sempre direcionadas a pessoas com
menos poder, como mulheres, negros, lésbicas, gays, bissexuais,
transgêneros e pessoas queer. Estima"-se que para mais de 60\% das
mulheres as microagressões são uma realidade no local de trabalho. Mais
do que os trabalhadores homens, elas geralmente precisam comprovar sua
competência e são com mais frequência questionadas em sua área de
especialização. Além disso, as mulheres são duas vezes mais propensas
que os homens a serem confundidas com alguém em uma posição laboral
inferior, e a situação das mulheres negras é particularmente mais grave.

Também a orientação sexual e identidade de gênero resultam em maiores
discriminações contra as pessoas \versal{LGBTIQ}+. Cerca de 70\% das mulheres
lésbicas já lidaram com microagressões, e no caso delas há algumas
dificuldades específicas: elas são mais propensas que outras mulheres a
ouvir comentários humilhantes no ambiente de trabalho e, além disso, a
maioria sente que não pode falar sobre suas vidas pessoais no trabalho,
devido à probabilidade de sofrerem com o preconceito e discriminação.

O impacto do preconceito pode ser observado em diversos níveis, a
começar pelo local onde se ganha o sustento. As mulheres que
experimentam microagressões veem seus locais de trabalho como menos
justos e são três vezes mais propensas a querer deixar o emprego.

O assédio sexual, por seu turno, continua a permear o local de trabalho.
35\% das mulheres empresárias dos \versal{EUA} experimentaram assédio sexual em
algum momento de suas carreiras, desde ouvir piadas machistas até serem
tocadas de maneira sexual. Para algumas, a experiência é ainda mais
comum: 55\% das mulheres na liderança sênior, 48\% das mulheres lésbicas
e 45\% das mulheres nos campos técnicos relatam que foram assediadas
sexualmente. Segundo as pesquisas\footnote{Cf. Relatórios \emph{Women in
  the workplace} 2015, 2016 e 2017. Disponível em:
  \textless{}\emph{https://cutt.ly/wysu0nR}\textgreater{}.} da McKinsey \& Co de onde
retirei estes dados uma linha comum conecta esses grupos: mulheres que
não se conformam às expectativas femininas tradicionais, mantendo a
autoridade, não sendo heterossexuais e trabalhando em campos dominados
por homens, são mais frequentemente alvos de assédio sexual.

No Brasil, de acordo com o Instituto de Pesquisa Econômica Aplicada
(Ipea), 67\% dos casos de violência contra as mulheres são cometidos por
parentes próximos ou conhecidos das famílias das vítimas, 70\% das
vítimas de estupro são crianças e adolescentes e apenas 10\% dos
estupros são notificados. Entre as mulheres negras brasileiras, os
assassinatos aumentaram 54\% entre 2003 e 2013, segundo o Mapa da
Violência de 2015, elaborado pela Faculdade Latino"-Americana de Ciências
Sociais (Flacso), \versal{OPAS}, \versal{ONU} Mulheres Brasil e Ministério das Mulheres,
Igualdade Racial e Direitos Humanos. Esse número é elevado mesmo se
comparado aos 21\% de incremento nos assassinatos de mulheres no mesmo
período. Das mortes violentas, 50,3\% são cometidas por familiares e
33,2\% por parceiros ou ex"-parceiros.

Com relação à cor da pobreza, os dados do último Relatório
Socioeconômico da Mulher mostram que as mulheres negras são as que mais
morrem por causas obstétricas: 64\% das mortes se dão entre mulheres
negras e 34\% ocorre entre mulheres brancas. Dos 774 milhões de adultos
analfabetos no mundo, 64\% são mulheres, de acordo com o relatório da
Unesco publicado em 2013. As mulheres são as principais responsáveis
pelos afazeres domésticos, de acordo com nossa experiência diária e
também com uma pesquisa do Ipea.\footnote{Para maior detalhamento ver
  ``Mudanças no mercado de serviços domésticos: uma análise da evolução
  dos salários no período 2006-2011'', \versal{DOMINGUES}, E. P. \& \versal{SOUZA}, K. B,
  2012. Disponível em: \textless{}\emph{https://bit.ly/2Y7WdGu}\textgreater{}.} A
média de dedicação semanal das mulheres a esse tipo de trabalho é de 25
horas semanais, contra a média de 10 horas semanais entre homens. Em
nível global, um relatório de 10 anos de pesquisa da McKinsey \& Co
tematizando avanços e dificuldades no processo de ascensão de mulheres a
cargos de comando das empresas concluiu:

\begin{quote}
No início de nossa pesquisa, em 2007, destacamos o duplo ônus das
mulheres: sua responsabilidade relativamente maior pelas tarefas
domésticas enquanto mantêm um emprego. Na Europa naquela época, as
mulheres passavam o dobro do tempo em tarefas domésticas que os homens.
As mulheres que entrevistamos enfatizaram como isso --- juntamente com a
necessidade de se tornarem disponíveis a qualquer hora, em qualquer
lugar, para mostrar que estavam falando sério sobre o trabalho --- era uma
grande barreira para seu avanço. Seu fardo não se tornou muito mais
leve. O relatório \emph{Women in the Workplace 2017} descobriu que mais
da metade das mulheres entrevistadas faz todo ou a maior parte do
trabalho doméstico. E as mulheres com filhos e parceiros têm 5,5 vezes
mais probabilidade de fazer a totalidade ou a maior parte do trabalho
doméstico do que os homens na mesma situação familiar. Não
surpreendentemente, talvez, também descobrimos que as mulheres que fazem
a maior parte do trabalho doméstico têm aspirações mais baixas de subir
para os degraus mais altos da escada corporativa em comparação com as
mulheres que compartilham a responsabilidade.\footnote{Cf. \textless{}
  \emph{https://mck.co/3eYVNIv}\textgreater{}.}
\end{quote}

No Brasil, a média salarial feminina corresponde a 74,5\% da média
salarial masculina, de acordo com a Pesquisa Nacional por Amostra de
Domicílio (Pnad) de 2014. Segundo os dados da \versal{RAIS} (Relação Anual de
Informações Sociais) de dezembro de 2014\footnote{Cf.
  \textless{}\emph{https://bit.ly/358doJw}\textgreater{}.}
a diferença de remuneração média entre homens e mulheres é um pouco
menor, correspondendo a 82,39\%, entretanto há diferenças significativas
a depender do grau de instrução, com as mulheres de Superior Completo
recebendo apenas 61,67\% do salário recebido por homens de mesma
qualificação. Mulheres com mestrado e com doutorado recebem,
respectivamente, em média 68\% e 77,9\% do que é pago a homens com mesmo
grau de instrução. Nos estratos menos qualificados, a diferença é menor,
porém também é expressiva: mulheres analfabetas recebem cerca de 83\% do
salário de homens analfabetos, enquanto mulheres com ensino fundamental
incompleto, ensino fundamental completo, ensino médio incompleto e
ensino médio completo recebem, em média, de 67\% a 72\% do que é pago a
homens com o mesmo grau de instrução.

As mulheres negras, além de receberem menos, apresentam uma maior
concentração em ocupações de menor remuneração: um estudo de 2009 do
\versal{IPEA} apontou que 21\% das mulheres negras no Brasil são trabalhadoras
domésticas, contra 12,5\% das mulheres brancas, e apenas 22\% têm
carteira assinada.\footnote{Em agosto de 2014 entrou em vigor a Lei das
  Domésticas, que previa carteira assinada, jornada de trabalho definida
  e pagamento de horas extras às trabalhadoras deste ramo. Três anos
  depois, entretanto, 70\% das domésticas seguiam na informalidade.
  Segundo pesquisadores isso se devia ao encarecimento dessa força de
  trabalho em um cenário de crise econômica, o que teria levado muitas
  pessoas a preferir diaristas e a evitar a contratação de empregadas
  domésticas conforme a lei. Disponível em:
  \textless{}\emph{https://bit.ly/3aL02nI}\textgreater{}.}

Em relação aos dados nacionais da \versal{RAIS} referentes aos dados de emprego
(celetistas ativos) por Raça/Cor e Sexo em 2013 e 2014 (31 de dezembro)
podemos perceber que, num cenário de perda de empregos (com carteira
assinada) de homens brancos a uma taxa de -2,57\% houve aumento do
número de celetistas ativos negros e pardos a uma taxa de +1,11\% e
+2,98, respectivamente. Uma diferença expressiva. Do mesmo modo,
enquanto o número de mulheres brancas empregadas manteve"-se estacionado
em praticamente a mesma proporção, na comparação entre 2013 e 2014, as
mulheres negras e pardas com vínculo celetista ativo cresceram +5,71 e
+7,22\%, respectivamente. A taxa de variação relativa no número de
empregos de mulheres indígenas apresenta o impressionante número de
-16,35\% (ou 6,7 mil menos mulheres indígenas empregadas), frente a um
acréscimo de +4,37\% de homens indígenas. Em números totais, entre 2013
e 2014 houve um aumento global de empregos celetistas a uma taxa de
0,41\% para homens (acréscimo de 98,7 mil vínculos), enquanto isso o
número de mulheres empregadas aumentou a uma taxa de 3,06\%, o que
representa um acréscimo de mais de 580 mil empregos. Enquanto 333 mil
homens brancos perderam o vínculo empregatício, cerca de 54 mil negros
(na maioria mulheres) e 566 mil pardos (na maioria mulheres) passaram a
ter emprego com carteira assinada. O conjunto dedados demonstra que
houve, portanto, transferência de empregos de homens para mulheres e de
pessoas brancas para pessoas negras e pardas.\footnote{Os dados podem
  ser ainda mais expressivos se pensarmos que, em 2013, 2.85 milhões e,
  em 2014, 3.16 milhões de trabalhadores não identificaram sua cor de
  pele, se tratando, muito provavelmente, de pessoas não brancas. Este
  contingente ``não identificado'' apresentou altos índices de
  crescimento nas taxas de vínculo celetista: 181,7 mil homens e 133,6
  mil mulheres passaram, de 2013 para 2014, a ter emprego com carteira
  assinada, o que significa um aumento anual de 10,5\% e de 12\%.}

A explicação para estes dados pode ser encontrada na própria
concorrência entre trabalhadores por postos de trabalho celetista, uma
vez que haja diferenciação salarial para menos quando o empregado não é
branco e nem homem. Nesse sentido, os patrões estariam substituindo o
corpo de funcionários masculinos e brancos por pessoas ``dispostas'' a
receber menos enquanto desempenham a mesma função. Por isso cabe
observar os dados referentes a essa variável.

Quanto à diferença na remuneração de acordo com o grau de instrução e
Raça/Cor, os dados da \versal{RAIS} referentes à remuneração média do mês de
dezembro de 2014 são os seguintes: os negros com ensino superior
completo recebem em média apenas 67,58\% do salário de brancos com a
mesma qualificação. Já os pardos com ensino superior recebem 72,35\% do
salário pago aos brancos. Os pardos com ensino superior incompleto
recebem em média 79,8\% do que é pago aos brancos, e os negros com
superior incompleto recebem em média 82,8\% do valor pago a brancos. Nos
demais níveis de instrução (analfabeto, até o 5º ano do ensino
fundamental, 5º ano completo, do 6º ao 9º ano incompleto, ensino
fundamental completo, ensino médio incompleto e ensino médio completo),
que abarcam a vasta maioria dos empregos, o salário de negros e pardos
costuma representar entre 84\% e 91\% do que é pago aos brancos de mesma
qualificação, portanto a disparidade salarial é maior nos estratos de
maior grau de instrução e maior remuneração. Do ponto de vista
estritamente econômico a substituição de mão de obra celetista branca e
masculina por trabalhadores de cor e mulheres estaria, a princípio,
relacionada a estes ganhos de se pagar em média mais de 10\% a menos
para mulheres e pessoas de cor. Em termos salariais essa diferença
representa pagar 100 reais a menos para negros e pardos analfabetos
(salário de pessoa branca sendo de em média R\$1250), 150 reais a menos
para negros e pardos com ensino médio incompleto (salário de pessoa
branca sendo em média R\$1465), significa pagar de 200 a 300 reais a
menos para negros e pardos com ensino médio completo (salário de brancos
sendo de em média R\$1840), e significa pagar, em média, R\$2719 reais
para brancos, R\$2170 para pardos e R\$2252 para negros, portanto cerca
de 500 reais a menos para pessoas de cor. Já no estrato com ensino
superior completo a disparidade é ainda maior absoluta e relativamente:
paga"-se em média R\$5589 reais para brancos, R\$3777 para negros e
R\$4044 para pardos, portanto os patrões pagam de 1500 a 1800 reais a
menos para pessoas de cor.

Para os capitalistas, as vantagens econômicas de se contratar mulheres e
pessoas de cor são, portanto, por si mesmas evidentes. Contudo, neste
livro defendemos a hipótese de que a substituição de mão de obra
(vislumbrada acima) também está ligada a fatores políticos e ideológicos
que extrapolam os ganhos patronais em termos de diferença salarial e
remetem para ganhos mais amplos. Ou seja, a empresa contrata mulheres e
pessoas de cor não apenas porque lhes paga em média um salário inferior,
mas também porque junto com seus lucros cresce também a imagem da
empresa como empresa ``cidadã'', de ``responsabilidade social'', que
adere à ``agenda da diversidade'' e se preocupa com a ``inclusão
social'', num processo com efeitos em cascata que terminam por reforçar
a capilaridade do poder empresarial por sobre os trabalhadores, na exata
medida em que eles se mostram mais obedientes e menos conflitivos quando
seu superior hierárquico é do mesmo gênero e cor.

Os dados, entretanto, confirmam pela enésima vez que há disparidade
salarial motivada por fatores de gênero e raça. Em primeiro lugar, é
preciso pontuar que, se inerentes ao capitalismo ou não, o fato é que
existem e se reproduzem no capitalismo formas nefastas de machismo,
racismo e homofobia, formas que precisam ser combatidas com vigor pela
classe trabalhadora organizada, uma vez que é ela a mais prejudicada
pelas opressões e discriminações. O problema, contudo, é que o atraso e
a debilidade histórica das organizações dos trabalhadores na luta contra
as opressões de gênero, raça, etnia e sexualidade tem levado, nas
últimas décadas, a formas de reação organizada a essas opressões que
permitem que se entranhe na esquerda um conjunto de políticas
identitárias pautadas na \emph{teoria dos privilégios}.

A teoria dos Privilégios reconhece que as opressões são estruturais e
históricas, porém enfatizam exageradamente os comportamentos e
pensamentos individuais como a principal forma de abordar o racismo, o
machismo e outras opressões. Essa teoria tem um conjunto de princípios
básicos:

\begin{enumerate}
\def\labelenumi{\alph{enumi}.}

\item A Teoria dos Privilégios argumenta que os espaços do
movimento devem ser seguros para todos os grupos oprimidos. Uma forma de
tornar tais espaços seguros é negociando as relações entre uns e outros
de formas não opressivas. Isto significa, por exemplo, que homens
brancos heterossexuais deveriam falar menos ou pensar sobre seus
privilégios quando se discute uma ação ou questão política.

\item A Teoria dos Privilégios alega que a militância e a
sofisticação política são o domínio de uma elite privilegiada baseada em
privilégios de classe, gênero e raça.

\item A Teoria dos Privilégios atribui erros políticos e
estratégicos aos privilégios pessoais que as pessoas carregam para
dentro do movimento.

\item A Teoria dos Privilégios busca lidar com essas questões
primeiramente através da educação, com formações e debates. (\versal{WILL}, 2014:
1)
\end{enumerate}

Com a teoria dos privilégios a própria luta contra as opressões passa a
ser tematizada não em termos de organização de classe, mas em termos de
organização e resistência dos oprimidos, vetando"-se, inclusive, que
sujeitos políticos não diretamente oprimidos se somem, em termos
igualitários, em uma luta contra determinada opressão. A política posta
em prática pela teoria dos privilégios aprofunda e complexifica a
fragmentação dos trabalhadores, diluindo suas consciências de classe. Ao
mesmo tempo, os capitalistas consolidam a sua unificação à medida que,
embora concorram entre si, promovem relações sociais de exploração em
comum e são organizados por uma mesma tecnocracia gestorial capitalista,
que é quem elabora as diretrizes dos programas socioeconômicos.

Se no âmbito teórico o identitarismo se alimenta das teorias
pós"-estruturalistas, no plano político o multiculturalismo e o
identitarismo podem ser entendidos como formas de nacionalismo adaptadas
à época do capital transnacional. Tal como ocorria com os nacionalismos,
os identitarismos apresentam como sendo homogêneas algumas
pseudo"-identidades que, na realidade, são rasgadas por diferenças de
classe. Os identitarismos são o nacionalismo da época da
transnacionalização na mesma medida em que atualmente as fronteiras
nacionais não dividem cada identidade. Isso permite que os
identitarismos multipliquem os defeitos dos nacionalismos, num sentido
ainda mais profundo, pois enquanto o nacionalismo se autolimita por
conta da questão da língua e das fronteiras territoriais não existe nada
que limite ou trave as subdivisões feitas sob a égide da noção de
identidade. Sobre isso, João Bernardo comenta\footnote{Cf.
  \textless{}\emph{https://bit.ly/2VImVnA}\textgreater{}.} que
``a conhecida tese de que «o corpo é político» é o limite último do
identitarismo, a identidade reduzida ao indivíduo''. Não por acaso vimos
noticiados, nos últimos anos, alguns casos esdrúxulos, como por exemplo
a história da mulher que afirma ser um gato num corpo humano\footnote{Cf.
  ``Transespécie: Mulher afirma ser um gato num corpo humano''.
  Disponível em: \textless{}\emph{https://bit.ly/2W5Pbzl}\textgreater{}.} e a do homem que
entrou com um processo para mudar de idade e ficar 20 anos mais jovem.
Emile Ratelband, em entrevista à \versal{BBC}, disse o seguinte: ``Quando estou
no Tinder e digo que tenho 69 anos, ninguém me responde. Quando eu digo
que tenho 49, com o rosto que tenho, estarei em uma posição de luxo'', e
arrematou: ``Vivemos em um tempo em que você pode mudar seu nome e seu
gênero. Por que não posso decidir sobre minha própria
idade?''.\footnote{Cf. ``Dutchman, 69, brings lawsuit to lower his age
  20 years''. Disponível em: \textless{}\emph{https://bbc.in/3cPDKCM}\textgreater{}.}

Com expressividade no plano do pensamento acadêmico, e aceitas enquanto
elementos a uma vez táticos e estratégicos postos em prática pelas lutas
contra as opressões, as lutas identitárias se mostram como algo distinto
do que aparentam, ou seja, não como elementos de subversão da lógica
machista, racista e homofóbica presente no sistema, o que lhes daria um
caráter subversivo e antissistêmico, mas enquanto agentes que dinamizam
o capitalismo, agentes de incremento dos mecanismos de desenvolvimento
capitalista que, por isso, tornam o sistema mais forte, ao invés de
debilitá"-lo. Alguns dos motivos que levam os movimentos identitários a
reforçar o capitalismo serão expostos a seguir.

\chapter{Identitarismo e desenvolvimento capitalista}

Os movimentos identitários são um agente do dinamismo capitalista porque
defendem políticas afirmativas que têm como resultado a integração de
negros, mulheres, \emph{\versal{LGBT}s} e ``minorias'' étnicas nas camadas
dominantes. Sendo assim, contribui com a renovação das elites e fornece
os quadros gestores a serem convocados pelas transnacionais. A teoria e
prática identitária são, portanto, agentes do dinamismo do capitalismo
na medida em que as lutas identitárias resultam na dinamização das
elites, substituindo o que há de velho e anacrônico. A renovação das
classes dominantes reforça a capilaridade do poder ao estender e
reforçar a legitimação do domínio capitalista para as trabalhadoras,
para os trabalhadores negros, para os trabalhadores e trabalhadoras
\versal{LGBTI} e de ``minorias'' étnicas. Essa legitimação se dá de modo tácito,
conforme os trabalhadores passam a ver o sistema como menos hostil à sua
própria ascensão social. Passam a sonhar com a mobilidade social
ascendente e a arquitetar, politicamente, as formas de garantir tal
ascensão, seja valorizando nichos de mercado ocupados e geridos por
representantes dessas minorias (pense"-se, por exemplo, nas táticas do
\emph{Movimento Black Money}), seja pressionando Estado e empresas a
absorverem representantes negros, mulheres, \versal{LGBT}s.

Quando uma luta identitária por integração cidadã das ``minorias''
termina sendo vitoriosa estamos diante de algo contraditório, afinal uma
vez que o racismo, o machismo e demais formas de discriminação e de
opressão são usados pelos capitalistas de modo a que a exploração de
classe assente nas bases nefastas da mais"-valia absoluta --- que todo
trabalhador quer evitar --- qualquer avanço no sentido do solapar de tais
bases seria, por si só, algo a se comemorar. A questão é até que ponto a
integração cidadã e trabalhista do negro, da mulher e de outros grupos
\emph{excluídos} de fato contribui com o fim das discriminações
estruturais que histórica e cotidianamente os afligem.\footnote{A ideia
  de ``exclusão social'' é corretamente criticada pelo sociólogo José de
  Souza Martins, que a considera ``inconceituável, imprópria, vaga e
  indefinida''. O conceito de exclusão substitui a ideia de ``processo
  de exclusão'', tornando a questão contraditória da posição social dos
  sujeitos algo mecânico e fixo, como se houvesse um dentro e um fora da
  sociedade de classes. Ao invés de excluídos, o que temos no
  capitalismo são sujeitos que são socialmente \emph{incluídos de modo
  rebaixado} em processos sociais, políticos e econômicos desiguais.
  Empobrecido, o conceito de exclusão expressa algo como ``o destino dos
  pobres'', remetendo a ``situações objetivas de privação'' (\versal{MARTINS},
  2002, p.\,43), o que distancia a análise do essencial: a luta por
  transformações sociais que quebrem os fundamentos dos processos
  sociais de exclusão, e não meramente por integração, o que implica
  estar a favor das relações sociais existentes, que, no entanto, se
  mostram ``inacessíveis a uma parte da sociedade'' (2002, p.\,47). Por
  fim, segundo Martins, discutindo a exclusão ``deixamos de discutir as
  formas pobres, insuficientes e, às vezes, até indecentes de inclusão''
  (1997, p.\,21).}

A mais"-valia absoluta, como sabemos, é caracterizada pelas formas de
aumento da exploração pela via da \emph{força}, sub"-remuneração, emprego
informal, terceirização, prolongamento da jornada e demais formas de
precarização do trabalho e corrupção da lei do valor. Apesar dos avanços
das últimas décadas em termos de integração dos grupos subalternos, a
situação de desigualdade entre homens e mulheres no mercado de trabalho
perdura, o que permite que capitalistas explorem as camadas femininas,
negras e \versal{LGBTI} da força de trabalho, bem como minorias étnicas,
emigrantes etc. a níveis crescentes, rebaixando o nível médio salarial
e, assim, incrementando os níveis de exploração da classe trabalhadora
como um todo.

A supressão da diferença salarial assentada em discriminações de gênero,
raça, nacionalidade etc. é uma bandeira secular das lutas dos
trabalhadores, lutas por \emph{igualdade} que nada tem a ver com as
lutas identitárias atuais por \emph{reservas} de espaços que operam por
meio do nivelamento de identidade e classe. Esse nivelamento apaga os
traços de classe e pressupõe que negros, imigrantes, mulheres e \versal{LGBT}s
são, via de regra, mais explorados, o que não é verdade. As mulheres e
os imigrantes não são sempre mais explorados, mas sim as mulheres e os
imigrantes pior qualificados. Os negros na África do Sul não são
igualmente explorados ou superexplorados, havendo conflitos entre
trabalhadores igualmente africanos pertencentes a grupos étnicos rivais,
porém igualmente negros, e resultando em níveis distintos de exploração
e mesmo, é claro, de elites negras que exploram trabalhadores negros.

Do mesmo modo e tocando a fundo nos limites das categorias identitárias,
há ainda as empresas, cada vez mais comuns, constituídas inteiramente de
mulheres ou inteiramente de negros ou de imigrantes ou de \versal{LGBT}s, o que
reduz a mobilidade destes trabalhadores e trabalhadoras e, assim,
rebaixa seus níveis salariais, proporcionando maior lucro para os
empresários proprietários de tais empresas. Os trabalhadores mais
qualificados são menos explorados, não importa sua origem ou identidade.
Não por acaso um dos dilemas do Brexit é justamente como barrar a
entrada de imigrantes de baixa qualificação e ao mesmo tempo promover a
vinda de imigrantes qualificados.

A potencialização da exploração por meio do machismo, racismo e variadas
formas de discriminação constitui, evidentemente, um problema
apresentado pelo capitalismo, mas um problema cuja solução, longe de se
apresentar como algo impossível, tem se mostrado algo muito lucrativo.
De acordo com um relatório de 2015 do \emph{McKinsey Global Institute},
a resolução da desigualdade de gênero em todas as suas dimensões
adicionaria \versal{US}\$ 28 trilhões ao \versal{PIB} global em 2025.\footnote{Apesar de
  representarem 50\% da população global em idade ativa, os dados de
  2018 mostram que, globalmente, as mulheres geram 37\% do \versal{PIB}. A
  contribuição média global para o \versal{PIB} mascara, contudo, grandes
  variações regionais: a parcela da produção regional do \versal{PIB} gerada por
  mulheres é de apenas 17\% na Índia, 18\% no Oriente Médio e Norte da
  África, 24\% no Sul da Ásia (excluindo a Índia) e 38\% na Europa
  Ocidental. Na América do Norte e Oceania, China e Europa Oriental e
  Ásia Central, a participação é de 40 a 41\%.} No Brasil, essa mudança
poderia gerar um \versal{PIB} 30\% maior, em 2025, com até \versal{US}\$ 850 bilhões a
mais em circulação.

Não por acaso, é possível dizer que a maioria dos preconceitos de raça,
de gênero e de preferência sexual situam"-se no interior da classe
trabalhadora, sendo coisa do passado para os capitalistas proprietários
e especialmente para os capitalistas gestores. A resolução na prática
destes preconceitos e discriminações no interior das relações entre
capitalistas --- bem como sua persistência nas relações entre
trabalhadores --- se deve à forma como a transnacionalização do capital
uniu os capitalistas e fragmentou os trabalhadores, entre outras coisas,
ao impor"-lhes rígidos processos de competição por vagas de trabalho e
lutas pela manutenção de padrões salariais e direitos trabalhistas.

Segundo os dados do \emph{\versal{IBGE}} (\emph{Instituto Brasileiro de Geografia
e Estatística}) a diferença de salário entre brancos e negros/pardos
diminuiu em 2015, quando os trabalhadores negros ganharam, em média,
59,2\% do rendimento dos brancos. Isso significa que a média de
rendimento de trabalhadores negros e pardos é de cerca de R\$ 1.510,00,
enquanto brancos recebem, em média, R\$ 2.550,00. Esse número mostra um
avanço em relação a 2003, quando os negros não ganhavam nem metade
(48,4\%) do salário dos brancos, mas a disparidade segue bem demarcada.
Com respeito à disparidade de gênero o resultado foi de que em 2015 as
mulheres ganharam, em média, 75,4\% do rendimento dos homens. Além de
comprovar os impactos materiais do machismo e do racismo, tais dados
expressam o processo perverso de rebaixamento dos níveis salariais, num
ciclo de reforço do machismo e racismo por parte de homens e brancos
prejudicados pela \emph{existência} de mão"-de"-obra sendo vendida a valor
inferior (atualmente é mais comum que esse movimento, que opõe
trabalhadores entre si, seja percebido enquanto tal quando há oferta de
mão de obra barata imigrante, mas de modo algum se trata de um fenômeno
restrito ao trabalho imigrante).

Embora os dados atuais apontem forte disparidade de rendimentos, de 2003
a 2013 a desigualdade de salários de brancos e negros diminuiu: o
salário dos negros subiu em média 51,4\%, enquanto o dos brancos
aumentou uma média de 27,8\%. É preferível pensar que não, mas é
possível que essa redução da desigualdade resulte no reforço do racismo
de brancos ``prejudicados'' pela ascensão econômica dos negros, em
especial se tal ascensão se deu por meio de qualquer tipo de
facilitamento legal, como por exemplo a imposição de cotas. Certamente,
a disparidade de aumentos salariais impacta positivamente no ganho de
expressividade dos discursos identitários, uma vez que comporta uma
unidade entre ideologia e ganhos materiais, inclusive imediatos, o que é
essencial para que uma luta em torno de determinadas pautas ganhe corpo
e aceitação frente aos interessados em construí"-la.

A desigualdade de gênero e de cor da pele se reflete também, embora em
menor grau, na dificuldade de obter emprego. De acordo com dados de
2013, do \emph{Dieese}, 53,9\% dos trabalhadores que procuravam emprego
há menos de um ano eram mulheres e 53,3\%, negros, sendo que a taxa
aumenta entre os desempregados há mais de um ano: nesta situação, 63,2\%
são mulheres e 60,6\% negros.\footnote{Cf.
  \textless{}\emph{https://bit.ly/2KEXYDq}\textgreater{}
  e
  \textless{}\emph{https://bit.ly/3aBs3y8}\textgreater{}.}
No entanto, o progresso na redução da desigualdade parece ser notável
quando observamos que entre 2003 e 2013 o desemprego entre mulheres
negras caiu de 18,2\% para 7,7\% (\versal{SPITZ}, 2013). No caso da desigualdade
com relação ao grupo \versal{LGBTI} temos a mesma lógica de discriminação, e
potencializada (cf. \versal{OTONI}, I., 2014; \versal{MORAES E SILVA}, S. F., 2012; \versal{REDE BRASIL ATUAL}, 2016). Como se explica que os mesmos grupos sociais que
recebem menos sejam também os que mais estão sujeitos ao desemprego? A
resposta, nos parece, deve necessariamente levar em conta os níveis de
impedimento e de acesso, por parte destes grupos, aos processos de
qualificação da força de trabalho, o que remete ao tema das cotas nas
seleções de Universidades etc.

No que diz respeito à composição de cor das elites no Brasil, um
levantamento feito pela \emph{Folha de São Paulo} constatou que são
brancos 80\% dos deputados federais, 74\% dos governadores, 90\% dos
reitores e vice"-reitores das universidades, 84\% dos atores das 5
novelas em exibição na \versal{TV} brasileira em 2015, 75\% dos presidentes dos
Conselhos regionais e federal de Medicina.\footnote{Cf.
  \textless{}\emph{https://bit.ly/3bIMk6h}\textgreater{}.}
Quanto à disparidade de gênero nas remunerações da força de trabalho,
segundo Helena Hirata ``ainda há um diferencial de salário da ordem de
30\% no caso do Brasil, entre 20\% e 25\% na França, e que chega a 40\%
ou até 50\% no Japão, onde as mulheres continuam ganhando quase a metade
com relação aos homens''.\footnote{Cf.
  \textless{}\emph{https://bit.ly/358FsfO}\textgreater{}.}

As lutas identitárias assentam, portanto, em bases indiscutíveis de
desigualdade de gênero, cor de pele, etnia e também sexualidade (embora
seja difícil encontrar dados sobre o tema). A questão deste livro não
diz respeito à legitimidade da luta feminista, da luta contra o racismo
e contra as diferentes formas de discriminação, mas ao modo como tais
questões são pautadas pelos movimentos identitários. O aspecto central,
a meu ver, diz respeito à forma como as pessoas se organizam em torno
daquelas pautas e, especialmente, o modo como o capitalismo responde a
essas lutas e demandas, incorporando"-as de modo a se fortalecer.

Desde sua criação o \emph{Instituto Ethos de Empresas e Responsabilidade
Social} buscou se articular com o ``Movimento de Responsabilidade Social
Empresarial'', o ``Pacto Global'' e as ``Metas do Milênio'', programas e
orientações internacionais que vinham ganhando espaço desde pelo menos a
\emph{\versal{IV} Conferência Mundial sobre População e Desenvolvimento},
realizada em 1994, no Cairo. Já em 2004, por exemplo, o Instituto
defendia que a

\begin{quote}
construção de um modelo de desenvolvimento sustentável vem cumprindo o
papel de elo entre as agendas dos governos, das empresas, dos movimentos
sociais e das organizações da sociedade civil em todo o mundo. A última
década do século \versal{XX} viu crescer o movimento de responsabilidade social
empresarial como balizador das relações de mercado. Grandes corporações
internacionais têm adotado a responsabilidade social como um dos
critérios para avaliar e selecionar seus fornecedores e parceiros.
Bancos e agências financiadoras vêm incluindo cláusulas sociais e
ambientais em suas políticas e nos contratos de concessão de crédito. As
empresas brasileiras também participam dessa tendência. O crescimento do
próprio Instituto Ethos de Empresas e Responsabilidade Social é um
indicador desse fato. Criado em 1998 por um grupo de apenas 11 empresas,
o Ethos conta hoje com 876 associadas, cujo faturamento totalizado
equivale a 30\% do \versal{PIB} brasileiro. Em todo o país, cada vez mais
empresas têm considerado as práticas de responsabilidade social como
critério para selecionar fornecedores, conceder crédito ou mesmo dirigir
seus investimentos. (2004: 15)
\end{quote}

Atento às tendências sociológicas, culturais, políticas e ideológicas
das lutas de classes e à possibilidade de se antecipar a elas de modo
lucrativo, Jorge Abrahão, diretor"-presidente do \emph{Instituto Ethos},
afirma em entrevista que as empresas precisam avançar mais rapidamente
na ``valorização de diversidades, sejam elas de gênero, raça ou
orientação sexual''. Comentando o estudo feito pelo Instituto onde se
observou que somente 5\% dos cargos executivos no Brasil são ocupados
por negros, sendo que sua representação na sociedade brasileira é de
51\%, o diretor pontuou: ``As empresas têm que reconhecer esse problema,
mas não como uma tarefa, e sim como um entendimento de que a diversidade
é uma riqueza para qualquer companhia''. Abrahão defende que as empresas
invistam em ``ações de qualificação direcionadas para as minorias e usem
a busca pela diversidade como critério em processos seletivos''.

A pesquisa do \emph{Instituto Ethos} foi feita em parceria com o
\emph{Ibope Inteligência}, coletando em 2010 dados das 500 maiores
empresas do Brasil. De 1162 diretores apenas 62 eram negros e 119 eram
mulheres, sendo apenas seis mulheres negras. Dentre assistentes e
secretários do quadro funcional 31,1\% eram negros, número que caía para
25,6\% nos cargos de supervisão e apenas 13,2\% estavam em cargos de
gerência. No ritmo de crescimento de 2007 a 2010 a equiparação de gênero
e cor nos cargos de chefia seria alcançada apenas daqui a 150 anos
\footnote{Cf. \textless{}\emph{https://bit.ly/2KCYP7B}\textgreater{}.
  Em nível global também se constata um ritmo mais lento do que o ideal:
  com base em dados de 462 empresas que empregam mais de 19,6 milhões de
  pessoas, um relatório da \emph{McKinsey Global Institute} concluiu que
  embora esteja havendo um movimento global de adoção empresarial da
  agenda da diversidade as mulheres permanecem sub"-representadas,
  especialmente as mulheres de cor, de modo que ``as empresas precisam
  mudar a maneira como contratam e promovem funcionários de entrada e de
  nível gerencial para obter um progresso real''. Cf.
  \textless{}\emph{https://mck.co/2W5qKC6}\textgreater{}.},
mas como a desproporção tem sido reduzida a um ritmo acelerado ano a ano
(em 2013 5,3\% dos postos de comando do topo da hierarquia corporativa
das grandes companhias eram ocupados por negros, sendo que dez anos
antes este dado era de 1,8\%) este dado provavelmente não reflete o
movimento histórico real, ou seja, a continuar o ganho de expressividade
destas pautas, como tem ocorrido em diversos países, a igualdade de
gênero e de raça nos postos de comando das empresas será alcançada em
poucas décadas.

Em 2016 o \emph{Instituto Ethos} divulgou uma pesquisa feita em parceria
com o Banco Interamericano de Desenvolvimento (\versal{BID}) e a Secretaria de
Promoção da Igualdade Racial da Cidade de São Paulo (\versal{SMPIR}), o
``\emph{Perfil Social, Racial e de Gênero dos 200 Principais
Fornecedores da Prefeitura de São Paulo}''. Essa pesquisa faz parte das
ações do \emph{Fórum São Paulo Diverso --- Fórum de Desenvolvimento
Econômico Inclusivo}, promovido pela \versal{SMPIR} e pelo \versal{BID}. No relatório
final é posto que a pesquisa ``pretende contribuir para a construção de
políticas indutoras das ações afirmativas'', uma ``estratégia que dará
sustentabilidade ao desenvolvimento nacional'', de modo que ``as
corporações necessitam compreender que elas e a sociedade são entes
interdependentes''.

O \emph{Fórum São Paulo Diverso} pretende estimular a ``adoção de
práticas empresariais inclusivas e ações afirmativas na gestão interna e
em relações com \emph{stakeholders}'' (público estratégico); em 2015
``contou com autoridades dos governos federal e municipal e \versal{CEO}s de
grandes corporações'' (por exemplo, \emph{Bayer, Microsoft, \versal{IBM}, Grupo
Kroton, Carrefour, Itaú"-Unibanco e Citibank}) ``para discutir e associar
a temática étnico"-racial a questões como ações afirmativas, educação,
tecnologia e empreendedorismo'', de modo que os presentes puderam
debater suas ``expectativas para a inclusão social por meio de políticas
corporativas de afirmação étnico"-racial, da própria legislação
brasileira, da educação e do empreendedorismo''. Comentando o papel das
empresas na ``agenda da diversidade'' (a qual pode ser definida como uma
maior proporção de mulheres e uma composição étnica e cultural mais
mista e ``colorida'' na liderança de grandes empresas), o
diretor"-presidente do \emph{Ethos}, Jorge Abrahão, afirmou:

\begin{quote}
Muitas são as ações que podem ser estabelecidas nas empresas: a
preparação de pessoas negras para cargos gerenciais e executivos, o
estabelecimento de ações afirmativas para negros em processos seletivos
de trainees, o treinamento das equipes de recursos humanos, mais
notadamente, o recrutamento e seleção para a adoção de critérios de
diversidade em seus processos seletivos, entre outras. {[}\ldots{}{]} É
preciso investir continuamente na promoção da igualdade de oportunidades
e incorporar a valorização da diversidade na cultura organizacional,
transformando a empresa em um ambiente que acolha as diferenças e
valorize os seus talentos, \emph{tendo em vista o melhor desempenho
dos negócios} e as transformações que queremos na sociedade. Não será em
um dia que mudaremos desigualdades que foram construídas em centenas de
anos, mas os programas de inclusão podem nos ajudar a acelerar um
processo de reversão de uma tendência, que, além de desvantajosa para o
resultado do negócio, é inaceitável em um mundo como o corporativo, que,
muitas vezes, \emph{antecipa o futuro}.\footnote{Cf.
  \textless{}\emph{https://bit.ly/2W4MVIK}\textgreater{}.}
\end{quote}

Na visão estratégica de longo prazo dos gestores a capacidade de
antecipar o futuro é o que diferencia uma empresa que irá à bancarrota
econômica ou que verá resultados positivos nas estratégias de expansão.
Em notícia de 2012, Paulo Itacarambi, vice"-presidente executivo do
\emph{Ethos}, afirmou que

\begin{quote}
De modo geral, as empresas reagem de duas maneiras perante a questão das
metas para promoção de diversidade: uma pequena parte sai na frente e
adota \emph{iniciativas pioneiras que, às vezes, estão mais adiantadas
que as próprias leis aprovadas}. Outras resistem enquanto podem e só
mudam a maneira de agir quando não têm mais jeito. A experiência
demonstra que \emph{as empresas que saem na frente garantem seu lugar
nos corações e mentes dos consumidores} e seu lugar no futuro. Elas
acumulam resultados tangíveis e intangíveis.\footnote{Cf.
  \textless{}\emph{https://bit.ly/2KEYmBS}\textgreater{}.}
\end{quote}

O reconhecimento do valor dos resultados intangíveis é um dos elementos
que diferencia os capitalistas gestores, com sua visão voltada para a
conversão em longo prazo destes resultados intangíveis em sustentados
resultados tangíveis, dos capitalistas burgueses, que são apenas os
proprietários das empresas e predominantemente têm sua atenção voltada
para os lucros a curto prazo.

Já em 2004 o \emph{Instituto Ethos} colocava, sem rodeios, a relação
íntima entre integração das mulheres e maior lucratividade das empresas:

\begin{quote}
Promover as mulheres no mundo do trabalho em geral também é interessante
para as empresas, pois isso colabora para aumentar a qualificação dos
profissionais disponíveis no mercado. Além disso, mulheres mais
qualificadas e com maior renda resultam num mercado consumidor maior e
mais dinâmico, com repercussões positivas em toda a economia. Outro
aspecto significativo é o crescente segmento do mercado consumidor que
orienta suas opções de compra por critérios de responsabilidade social.
Uma empresa que contribui para a igualdade de oportunidades entre homens
e mulheres é reconhecida pela sociedade, especialmente pelas próprias
mulheres, que hoje representam uma grande força na opinião pública e no
mercado consumidor. (2004: 17)
\end{quote}

Quanto ao potencial feminino em termos de mercado consumidor, o
\emph{Instituto} informava que:

\begin{quote}
Metade da população brasileira, as mulheres também são uma força no
mercado consumidor. Além de serem as maiores responsáveis pelas decisões
de compra de alimentos, cosméticos, joias, roupas e eletrodomésticos,
sua opinião também tem peso na aquisição de produtos como
microcomputadores, previdência privada e seguro de vida e é decisiva na
hora de escolher bens de consumo duráveis, como o carro da família.
Metade dos cartões de crédito existentes no país está em mãos de
mulheres. (2004: 19)
\end{quote}

Antes do próximo tópico cabe, no entanto, um breve parêntesis: a
superação do racismo, do machismo, do sexismo etc. entre os gestores é
visível nas grandes empresas transnacionais e também nas instituições
políticas e econômicas internacionais, como por exemplo o \versal{FMI}, a \versal{ONU} e o
Banco Mundial. No plano nacional os preconceitos são mais persistentes e
historicamente duradouros, sobretudo em países onde perdura entre as
elites políticas e econômicas uma mentalidade escravocrata, como é o
caso do Brasil.\footnote{Apenas para ilustrar: ``Mulheres negras
  enfrentam discriminação, racismo ainda persiste no trabalho''.
  \versal{CONTRAF}, 2015. Disponível em:
  \textless{}\emph{https://bit.ly/2Y9OrMp}\textgreater{}
  e ``Transexualidade e discriminação no mercado de trabalho''. \versal{MORAES E
    SILVA}, S. F., 2012. Disponível em:
  \textless{}\emph{https://bit.ly/2KCHonv}\textgreater{}.}
Essa realidade, inclusive, certamente atrasará nos brasileiros a
percepção e enfrentamento às armadilhas das lutas identitárias. Conforme
as empresas transnacionais varram as empresas nacionais comandadas por
estas elites retrógradas a tendência é uma maior absorção social das
pautas identitárias, o que tende a ser refletido em substituições dos
caciques políticos que não se adequarem ao espírito \emph{colorido} do
tempo. Nesse sentido, o ganho de expressividade do identitarismo anda
junto com os processos de modernização econômica e política.

\chapter{Empoderamento feminino e lucro}

De acordo com uma pesquisa de 2012 da \emph{McKinsey}\footnote{Cf.
  \textless{}\emph{https://mck.co/3bIJqOF}\textgreater{}.} as
empresas com os quadros executivos mais diversos em termos de raça e
gênero tiveram resultados melhores entre 2008 e 2010. A análise do
desempenho de 180 empresas na França, Alemanha, Reino Unido e Estados
Unidos mostrou que companhias classificadas no topo do ranking de
diversidade tiveram uma taxa de retorno do investimento dos acionistas
53\% maior na média e um \versal{EBIT} (\emph{Earnings Before Interest and
Taxes}, o lucro antes de encargos financeiros e impostos) 14\% mais
alto. Pesquisa de 2013 da \emph{Bain Company}\footnote{Cf.
  \textless{}\emph{https://bit.ly/2HeCbEp}\textgreater{} %%%%%
  e
  \textless{}\emph{https://bit.ly/2SdBWvr}\textgreater{}.
  Do mesmo modo, matéria pontua que ``Com a valorização de diferentes
  estilos de liderança, mais mulheres poderão conquistar posições de
  alta gestão em empresas brasileiras'' Cf.
  \textless{}\emph{https://bit.ly/3aELCWd}\textgreater{}.}
enfatizou que o apego às ``diferenças de estilos'' entre a liderança de
homens e de mulheres não constituem uma ``boa estratégia em momentos de
tomadas de decisão'', uma vez que assim se perde a capacidade de
respostas inovadoras em face dos problemas. Enquanto isso, uma pesquisa
de 2014 da \versal{\emph{AT\&T}}, intitulada \emph{Diversidade e inclusão global:
promovendo a inovação por meio da diversidade na força de
trabalho}\footnote{Cf.
  \textless{}\emph{https://bit.ly/2Sev80D}\textgreater{}.},
também afirmava que a diversidade é um fator de lucratividade,
concluindo que a inclusão é uma forma de assegurar que serviços,
produtos e atendimentos estejam de acordo com ``anseios dos mais
diversos públicos consumidores''.

Já um relatório de 2014 do \emph{Gallup Institute}, relativo a 800
unidades de negócios de duas empresas varejistas, mostrou que as
empresas cujo número de mulheres e homens é proporcional tiveram uma
receita média 14\% maior do que as menos diversas. Outro estudo, feito
com mais de 150 empresas alemãs ao longo de cinco anos, mostrou que uma
média de 30\% de participação feminina nos conselhos de administração já
indica melhores resultados financeiros em relação a conselhos sem
mulheres.\footnote{Cf.
  \textless{}\emph{https://bit.ly/2KCI0tj}\textgreater{}.}

A superação, dentre os capitalistas (em suas práticas empresariais), das
discriminações contra mulheres, negros e \versal{LGBT}s não advém de maior
esclarecimento. Ela assenta em uma base material muito sólida.

O \emph{Peterson Institute for International Economics} publicou, em
fevereiro de 2016, um documentado estudo intitulado ``\emph{Is Gender
Diversity Profitable? Evidence from a Global Survey}'', onde informa que
a análise de uma pesquisa global de 21,980 empresas de 91 países sugeriu
que a presença de mulheres em posições de liderança corporativa pode
melhorar o desempenho da empresa. Segundo o estudo, esta correlação
poderia ser um reflexo tanto da recompensa às empresas que não
discriminam quanto do fato de que as mulheres aumentam a diversidade de
habilidades (\emph{skills}) de uma empresa, dando a ela um diferencial
funcional. A presença das mulheres na liderança das empresas é
correlacionada de modo positivo com outras características das empresas,
tais como o tamanho. Não se chegou a resultados positivos ou negativos
quanto ao impacto decorrente das quotas de gênero, existentes em alguns
países, sobre o desempenho da empresa, mas confirmou"-se que podem ser
significativas as recompensas por políticas que facilitam a ascensão de
mulheres na hierarquia das empresas: uma passagem de 0 a 30\% de
mulheres nos cargos de comando está associada a um acréscimo de 15\% nos
lucros. Segundo essa pesquisa, os maiores ganhos são para a proporção de
mulheres executivas, seguidos pela proporção de membros do conselho do
sexo feminino. Conclui"-se, no entanto, que a presença de \versal{CEO}s do sexo
feminino, por si só, não tem qualquer efeito perceptível no desempenho
da empresa, o que ressalta a importância da criação de uma escalada de
gerentes do sexo feminino e não simplesmente de se ter mulheres
solitárias no topo da empresa.

Erhardt, Werbel, e Shrader (2003) e Carter et~al. (2007) constataram que
um maior equilíbrio de gênero entre os líderes empresariais está
associado a valores de estoque mais elevados e maior rentabilidade.
Outra pesquisa (McKinsey 2012b) sobre as empresas norte"-americanas
concluiu que os conselhos mistos de gênero superam os conselhos
masculinos. Do mesmo modo, as empresas da \emph{Fortune 500} com a maior
proporção de mulheres nos seus conselhos têm desempenho
significativamente melhor do que as empresas com a menor proporção
(Catalyst, 2011). A empresa de contabilidade \emph{Rothstein Kass}
(2012) concluiu que os fundos de \emph{hedge} chefiados por mulheres
superam os chefiados por homens. Mais diversidade nos cargos de comando
também contribui positivamente para o desempenho das empresas na América
Latina (McKinsey, 2013) e Espanha (\versal{CAMPBELL} e \versal{MÍNGUEZ"-VERA}, 2008 e
2009).

Outros estudos perceberam ganhos decorrentes de um equilíbrio maior de
gênero em setores e circunstâncias específicas. Dezso e Gaddis Ross
(2011), por exemplo, constataram que a adição de mulheres líderes
melhora o desempenho em empresas orientadas para a inovação. Lindstädt,
Wolff e Fehre (2011) perceberam resultados positivos no aumento da
liderança feminina em empresas orientadas para o consumidor. Eles também
perceberam que as empresas que têm uma força de trabalho
predominantemente feminina são muito beneficiadas quando têm líderes
femininas. No mesmo sentido, Flabbi, Macis, e Schivardi em ``\emph{Do
Female Executives Make a Difference?: The Impact of Female Leadership on
Firm Performance and Gender Gaps in Wages and Promotions}'' (2012)
concluem que ``os executivos do sexo feminino fazem a diferença'', uma
vez que ``a interação entre a liderança feminina e trabalhadoras
mulheres na empresa tem um impacto positivo significativo sobre o
desempenho da empresa''. Sugere"-se que um mecanismo importante por trás
dessa interação é a política salarial onde a existência de uma liderança
feminina implica aumentos salariais para as mulheres no topo da
distribuição de renda e diminuição de salários para as mulheres na parte
inferior. Haveria, portanto, ``custos de produtividade associados com a
sub"-representação das mulheres no topo da empresa''.

Jurkus, Park, e Woodard (2011) constataram que o aumento da equidade de
gênero pode ser benéfico em empresas com governança externa fraca. Nick
Wilson e Ali Altanlar (2009) analisaram dados de 900,000 empresas e
perceberam uma correlação entre composição de gênero dos cargos de
chefia e risco de insolvência. Maran Marimuthu (2009) analisou dados das
100 maiores empresas da Malásia e concluiu que há uma relação entre
melhor desempenho financeiro e maior diversidade étnica nos cargos de
direção. Abdullah, Ismail, e Nachum (2012) analisaram 841 empresas de
capital aberto na Malásia e notaram um impacto positivo em termos de
contabilidade empresarial, relacionado à presença de mulheres nos cargos
de chefia. Os pesquisadores concluíram que ``mulheres diretores criam
valor econômico, o que é subestimado pelo mercado''.

O \emph{Instituto de Pesquisa Credit Suisse} (2012) concluiu que em
``mercados desafiadores'' as empresas com mulheres em seus conselhos têm
melhor desempenho do que as empresas com conselho totalmente masculino.
Na sequência da crise econômica de 2008, por exemplo, o crescimento do
lucro líquido para as empresas com mulheres em seus conselhos foi em
média de 14\%, em comparação com 10\% para as empresas com direção
totalmente masculina.\footnote{Além dos diversos estudos publicados pela
  Catalyst desde 2002 e, no Brasil, pelo Instituto Ethos, outras
  pesquisas que trazem enfoques variados acerca dos benefícios
  econômicos da inclusão de mulheres e negros em cargos de comando das
  empresas são: \versal{AHERN} \& \versal{DITTMAR}, 2012; \versal{AVERY}, 2012; \versal{ABBOTT}, L, 2012;
  \versal{TERJESEN} \& \versal{SINGH}, 2008; \versal{TORCHIA}, 2011; \versal{SMITH} \& \versal{VERNER}, 2006; \versal{EU},
  2010 e 2011; \versal{FOLKMAN}, 2012; \versal{FRANCOEUR}, 2008; \versal{BRAMMER}, 2009;
  \versal{DALE"-OLSEN}, 2014; \versal{DEZSO} C. \& \versal{ROSS}, 2008 e 2012; \versal{BEAR}, 2010; \versal{HERRING},
  2009; \versal{HOMAN} \& \versal{GREER}, 2013; \versal{BLAZOVICH}, 2013; \versal{MATSA}, 2013; M. \versal{JOECKS},
  2012; \versal{LARKIN}, 2012; \versal{NIELSEN} \& \versal{NIELSEN}, 2013; \versal{PAUL}, L. \& \versal{DONAGGIO},
  2013; \versal{PURI}, 2016; \versal{REN} \& \versal{WANG}, 2011.}

Até mesmo a legalização do casamento gay, nos Estados Unidos, resultou
em ganhos econômicos. Dario Sansone, economista da Georgetown
University, estudou o efeito da legalização do casamento gay na
discriminação e no mercado de trabalho nos estados americanos. O
pesquisador relacionou o casamento gay e dados do censo para estimar o
impacto sobre as taxas de emprego, percebendo um aumento de 2,3\% na
probabilidade de ambos os parceiros em um relacionamento gay ou lésbico
estarem trabalhando, juntamente com um aumento nas horas trabalhadas e
uma redução no trabalho autônomo. O dado é significativo, pois segundo
Sansone o que se esperava era uma redução da participação de \versal{LGBT}s na
força de trabalho, por conta dos cerca de 1.138 benefícios, direitos e
privilégios legais que o casamento proporciona, que vão desde leis
tributárias e de herança até a cobertura de saúde em planos de um
cônjuge. A legalização poderia ter incentivado alguns recém"-casados ​​a
deixarem um emprego que só mantinham a fim de ter acesso aos benefícios,
bem como encorajado mais casais a começar uma família, aumentando o
número de pais e mães que passam a se concentrar em trabalhos domésticos
e criação dos filhos etc. A pesquisa concluiu, ainda, que o aumento no
emprego após a legalização se aplicou a parceiros do mesmo sexo sendo
casados ​ou não. Outra pesquisa, do Departamento Nacional de Pesquisa
Econômica, coordenada por Chang"-Tai Hsieh, analisou o impacto econômico
da crescente igualdade no local de trabalho ao longo do último meio
século em relação a raça e gênero. Nos \versal{EUA}, em 1960, 94\% dos médicos e
advogados eram homens brancos, sendo que hoje essa proporção é de 64\%.
A pesquisa estimou que o impacto da mudança sobre a produtividade de
afro"-americanos e mulheres previamente excluídas e que possuem talentos
inatos para profissões específicas poderia representar 25\% do aumento
da produção por pessoa nos \versal{EUA} entre 1960 e 2010. Mais que uma questão
de equidade, o combate à discriminação seria também uma questão de
eficiência e, portanto, produtividade e lucro.\footnote{Cf. \emph{The
  costs of discrimination: How equal rights can boost economic growth}.
  Disponível em: \textless{}\emph{https://econ.st/2yS2dsg}\textgreater{}.}

Basicamente, portanto, as centenas de pesquisas a que tive acesso têm em
comum o fato de que de uma forma ou de outra apontam que maior
diversidade da força de trabalho e dos cargos de comando das empresas é
sinônimo de maior lucratividade, ou, como se afirma no documento
\emph{Princípios de Empoderamento das Mulheres,} elaborado pela
\emph{\versal{ONU} Mulheres} e o \emph{Pacto Global das Nações Unidas}, quanto à
integração das mulheres ``em todos os níveis'': \emph{``Igualdade
significa, de fato, negócios''} (p.\,5).

Ancorados em centenas de pesquisas que comprovam a maior lucratividade
inerente a um quadro gestor mais colorido, alguns institutos têm se
dedicado a investigar as causas da baixa presença de mulheres em cargos
de decisão, bem como a forma de enfrentar o problema. Num destes
estudos, levado a cabo por inúmeras universidades, publicado em dezembro
de 2017 e intitulado \emph{What Prevents Female Executives from Reaching
the Top?}\footnote{Cf. \textless{}\emph{https://bit.ly/3aEjKBL}\textgreater{}.}
conclui"-se o seguinte:

\begin{quote}
Dados excepcionalmente ricos da Suécia tornam possível estudar a lacuna
de gênero na progressão de carreira dos executivos e investigar suas
causas. Seguimos as carreiras de todos os futuros executivos nascidos
entre 1962 e 1971 no período de 1992 a 2011 e perguntamos como suas
qualificações, progressão na carreira e assuntos familiares explicam seu
sucesso profissional em 2011, ou seja, quando eles têm 40-49 anos de
idade. Descobrimos que a criação de filhos desempenha um papel crucial
na formação de lacunas de gênero nas principais nomeações executivas. A
maioria dessas lacunas de gênero surge durante os cinco anos seguintes
ao nascimento do primeiro filho, uma época em que as lacunas de gênero
no horário de trabalho dos executivos e a ausência do trabalho são
maiores. As mulheres estão em carreiras semelhantes antes do parto, mas
ganham substancialmente menos que os homens cinco anos após o parto.
Esta penalidade das crianças permanece grande durante o curso restante
das carreiras dos executivos. Estes resultados sugerem que \emph{as
mulheres aspirantes podem não alcançar a suíte executiva sem negociar a
vida familiar.} (p.\,43)
\end{quote}

O Relatório ``\emph{Economic empowerment of women}'' (Empoderamento
econômico da mulher), de 2012, coloca alguns dos benefícios econômicos
do empoderamento feminino:

\begin{enumerate}
\def\labelenumi{\alph{enumi}.}

\item Quando mais mulheres trabalham, as economias crescem. Se as taxas de
emprego das mulheres remuneradas forem elevadas ao mesmo nível que a dos
homens, o produto interno bruto dos \versal{EUA} aumentaria 9\%, o da Zona do
Euro iria subir em 13\% e o do Japão seria impulsionado em 16\%. Além
disso, em 15 grandes economias em desenvolvimento a renda \emph{per
capita} aumentaria em 14\% até 2020 e em 20\% em 2030.

\item Uma análise das empresas \emph{Fortune 500} descobriu que aqueles com
a maior representação das mulheres em cargos de gerência entregou um
retorno total aos acionistas que era 34\% maior do que para empresas com
menor representação.

\item A evidência de uma série de países mostra que o aumento da proporção
da renda familiar controlada por mulheres, seja através dos seus
próprios rendimentos ou transferências de dinheiro, muda os gastos de
forma a beneficiar as crianças.
\end{enumerate}

Quanto às mulheres no mundo do trabalho, o documento afirma que, se as
mulheres tivessem o mesmo acesso que os homens aos bens de produção, a
produção agrícola em 34 países em desenvolvimento aumentaria em uma
média estimada de até 4\%, o que poderia reduzir o número de pessoas
subnutridas nos países em até 17\%, traduzindo"-se em até 150 milhões a
menos de pessoas com fome no mundo. Uma pesquisa de 2015 do
\emph{McKinsey \& Company}\footnote{The Power of Parity: How Advancing
  Women's Equality Can Add \$12 Trillion to Global Growth. New York.}
concluiu que a paridade de gênero no plano econômico pode adicionar 25\%
à produção global, em um acréscimo de cerca de \versal{US}\$12 trilhões.

Não por acaso muitas das maiores empresas transnacionais estão dentre as
financiadoras de projetos, organizações, instituições e movimentos de
luta contra a discriminação e pela igualdade de raça, gênero e etnia. A
\emph{\versal{ONU} Mulheres}, por exemplo, possui um \emph{Fundo para a Igualdade
de Gênero} e um \emph{Fundo Fiduciário para Eliminar a Violência contra
a Mulher}, o qual tem dentre seus maiores financiadores as empresas
\emph{Coca"-Cola Company, Ford Foundation, Petrobras, Itaipu Binacional,
Johnson \& Johnson, Kraft Foods Middle East and Africa Ltd., La
Foundation L'Occitaine, Loomba Foundation, Microsoft Corporation,
Rockefeller Foundation, Kuait America Foundation, Women's Self Worth
Foundation e Zonta International}.\footnote{Informe Anual 2012-2013. \versal{ONU}
  Mulheres.}

Também não é por acaso que já é possível encontrar estudos e documentos
oficiais onde se fala não só em ``epistemologia feminista'' (cf. \versal{MATOS},
2008), mas também em \emph{Economia feminista}. Já em 2003 Verônica
Montesinos defendia que na América Latina pós"-democratização
``pressupostos e práticas preconceituosos quanto ao gênero têm sido
apenas parcialmente abordados, em parte porque o processo de elaboração
de políticas é controlado por economistas, um grupo profissional com uma
postura particularmente hostil às análises de gênero'', concluindo que
``mudanças no interior da (disciplina) Economia poderiam colaborar na
tarefa de tornar a democracia mais sensível às demandas das mulheres''
(\versal{MONTESINOS}, 2003).

Na coletânea ``\emph{Orçamentos sensíveis a gênero: conceitos}'',
publicada pela \emph{\versal{ONU} Mulheres} em 2012, podemos ler quatro artigos
em torno do tema: ``Análise econômica para a igualdade: as contribuições
da economia feminista'', ``O papel dos orçamentos sensíveis a gênero na
construção da igualdade e do fortalecimento das mulheres'', ``Condições
de vida: perspectivas, análise econômica e políticas públicas'' e
``Gastos, tributos e equidade de gênero: uma introdução ao estudo da
política fiscal a partir da perspectiva de gênero''.

Em publicação paralela, o livro ``\emph{Orçamentos sensíveis a gênero:
experiências}'' traz diversos casos de aplicação dos conceitos da
``economia feminista'', desde ``A política tributária como ferramenta
para equidade de gênero: o caso do imposto de renda sobre pessoas
físicas na Argentina'' até ``Mulheres em ação pelas mulheres: o caso das
finanças públicas sensíveis a gênero de Timor"-Leste''. Com o ganho de
expressividade e a amplitude do leque de práticas e teorias destinadas a
fundamentar a igualdade de gênero no âmbito empresarial e dos orçamentos
e políticas estatais, não é de se admirar o fato de que ano a ano os
dados referentes à presença de mulheres nos cargos de comando da
economia e política têm aumentado.

Em fim de 2018, a Primeira"-Ministra britânica, Theresa May, organizou,
em sua residência oficial, a primeira ``Conferência mundial de mulheres
parlamentares'', que contou com a participação de 120 mulheres, de 86
países. Segundo a representante portuguesa tratou"-se de um evento
voltado para o ``encorajamento para que se faça esta rede de mulheres
parlamentares'', de modo que a deputada buscou ``destacar aquilo que
Portugal tem vindo a fazer para diminuir as assimetrias entre homens e
mulheres'', como por exemplo ``a recente aprovação do aumento de 33,3
para 40\% da quota de representação dos dois sexos nos órgãos de poder
político e nos cargos dirigentes da administração pública''.\footnote{Cf.
  \textless{}\emph{https://bit.ly/2yM8khV}\textgreater{}.}

Nos dados de 2016 para a cidade de São Paulo a disparidade de gênero no
que tange à ocupação de cargos de direção é bastante reduzida, se
comparada aos dados nacionais da Pesquisa de 2010, também feita pelo
\emph{Instituto Ethos}:

\begin{quote}
Em todas as edições da pesquisa nacional do Ethos, as mulheres
repre­sentavam menos de 15\% do quadro de diretores das empresas que
participaram da pesquisa. Con­siderando apenas as empresas da cidade de
São Paulo, dos setores aqui pesquisados, esse número mais do que dobra,
representando as mulheres 37\% do quadro executivo das empresas. No
município de São Paulo, elas também ficam com mais de 40\% do total dos
cargos de ge­rência e supervisão, situação que não encontra paralelo no
cenário nacional, em que elas ocupam menos de 30\% desses cargos, como
de­monstra o histórico da citada pesquisa do Insti­tuto Ethos. (Perfil
Social, Racial e de Gênero dos 200 Principais Fornecedores da Prefeitura
de São Paulo, p.\,42)
\end{quote}

Devido a certo boicote a parte do questionário aplicado pelo Instituto,
os dados usados pelo Ethos são os da pesquisa \versal{RAIS}/\versal{MTE} (Relação Anual de
Informações Sociais, feita pelo Ministério do Trabalho e Emprego) de
2013. O modo como o Instituto se apropria destes dados falseia parte dos
resultados; lemos no relatório da última pesquisa acerca de São Paulo,
por exemplo, que do total de 10.327 mulheres em cargos de diretoria
(contra 17.411 homens) apenas 450 se autodeclaram negras (4\%). Número
bastante expressivo, que serve aos propósitos políticos do Instituto,
que, entretanto, não informa que 52\% das mulheres diretoras não
informaram sua cor ou raça. O dado completo quanto ao cargo mais alto,
portanto, é o seguinte: 43\% das diretoras são brancas, 4\% negras, 1\%
amarelas e 52\% não informaram (a meu ver os dados em aberto devem ser
pensados como possivelmente se tratando de não brancas, o que implica
pensar que o número de negras e pardas está subdimensionado). Nos demais
cargos hierárquicos ocupados por mulheres (gerente e supervisora) há
predominância de mulheres brancas, com cerca de 15 a 20\% de negras,
contra 60 a 80\% de brancas. Quanto à disparidade de gênero, além de
37\% das diretorias, as mulheres estão em 43\% e 47\% dos cargos de
gerente e supervisor nas empresas paulistas, um dado razoável, em
especial se notarmos que esses percentuais de ocupação de cargos de
comando estão mais ou menos de acordo com o percentual dos empregos em
sua totalidade (incluído funcionários e aprendizes) ocupados por
mulheres em São Paulo: 46\%.

Uma pesquisa de 2012, coordenada pelo \emph{Instituto Brasileiro de
Governança Corporativa} (\versal{IBGC}), mostrou que de um total de 508 empresas
listadas no banco de dados da \emph{\versal{BM\&F} Bovespa}, 197 contavam com
pelo menos uma mulher no conselho de administração (38,78\%) e 165
possuíam pelo menos uma conselheira efetiva (32,48\%). Na Europa, 72\%
das empresas possuem ao menos uma mulher no conselho e, no Canadá,
71,42\%. ´

Em nível global, uma pesquisa de 2018 feita pela \emph{McKinsey \& Co} e
\emph{LeanIn.org,} com 279 empresas norte"-americanas que, juntas,
empregam mais de 13 milhões de pessoas, apontou que as empresas devem
ter ``no mínimo duas mulheres'' e ``quanto mais, melhor'' em cargos de
topo de gestão, a fim de evitar o ``isolamento'' da diversidade e suas
consequências contraproducentes e discriminatórias. O relatório
ressalta, ainda, a importância de um olhar atento à entrada de maior
diversidade desde os cargos mais baixos, além de medidas sólidas visando
garantir a qualificação e promoção dessas funcionárias.\footnote{Cf.
  \textless{}\emph{https://on.wsj.com/2VFz8cA}\textgreater{}.
  O relatório da pesquisa pode ser consultado aqui:
  \textless{}\emph{https://go.aws/2W2xJMb}\textgreater{}.}

A tabela abaixo retrata a composição média de gênero e cor ao longo de
seis níveis hierárquicos das carreiras nas empresas pesquisadas:

\medskip

\noindent{}\resizebox{\textwidth}{!}{
\begin{tabular}{|l|c|c|c|c|}
\hline
\multicolumn{1}{|c|}{\textbf{Cargo}} & \multicolumn{1}{l|}{\textbf{\begin{tabular}[c]{@{}l@{}}Mulheres\\ de cor\end{tabular}}} & \multicolumn{1}{l|}{\textbf{\begin{tabular}[c]{@{}l@{}}Mulheres\\ brancas\end{tabular}}} & \multicolumn{1}{l|}{\textbf{\begin{tabular}[c]{@{}l@{}}Homens\\ de cor\end{tabular}}} & \multicolumn{1}{l|}{\textbf{\begin{tabular}[c]{@{}l@{}}Homens\\ brancos\end{tabular}}} \\ \hline
\textbf{\begin{tabular}[c]{@{}l@{}}Chefe executivo/\\ Presidente\end{tabular}} & 4\% & 19\% & 9\% & 68\% \\ \hline
\textbf{\versal{VP} Sênior} & 4\% & 19\% & 9\% & 67\% \\ \hline
\textbf{Vice Presidente} & 6\% & 24\% & 12\% & 59\% \\ \hline
\textbf{Diretor} & 8\% & 26\% & 13\% & 52\% \\ \hline
\textbf{Gerente} & 12\% & 27\% & 16\% & 46\% \\ \hline
\textbf{Nível de entrada} & 17\% & 31\% & 16\% & 36\% \\ \hline
\end{tabular}
}

\begin{center}
{\scriptsize{Fonte: Relatório \emph{Women in the Workplace} 2018 (Adaptado).}}
\end{center}

\bigskip

Segundo consistentes pesquisas da McKinsey \& Co, em 2017, em média, as
mulheres representavam 17\% dos membros do conselho corporativo e 12\%
dos membros do comitê executivo nas 50 principais empresas listadas no
G"-20. Segundo a companhia, esse dado poderia e deveria ser maior, posto
que uma pesquisa global de 2010 (também com 279 empresas) constatou que
aquelas com maior proporção de mulheres em seus comitês executivos
obtiveram um retorno sobre o patrimônio líquido 47\% maior do que
aqueles que não tinham membros executivos do sexo feminino. As pesquisas
da McKinsey relacionando diversidade e lucratividade foram alvo de
críticas, porém as respostas dadas em seus relatórios não deixam margens
para dúvidas:

\begin{quote}
É claro que uma correlação não prova a causalidade, e alguns acadêmicos
contestaram o que consideram o apelo intuitivo de um elo entre
diversidade e desempenho. No entanto, um crescente corpo de pesquisa da
McKinsey continua a fortalecer esse elo. Nossa incursão de 2018 por meio
da diversidade, abarcando mais de 1.000 empresas em 12 países, encontrou
uma correlação entre diversidade no nível executivo e não apenas
lucratividade, mas também criação de valor. As empresas do primeiro
quartil para a diversidade de gênero tinham 27\% mais chances de superar
sua média nacional em termos de lucro econômico --- uma medida da
capacidade de uma empresa de criar valor excedendo seu custo de capital
- do que as empresas de quartis inferiores. Houve também uma penalidade
por falta de diversidade de forma mais ampla. As empresas no quartil
inferior, tanto na diversidade de gênero quanto na diversidade étnica,
eram menos propensas a registrar lucros mais altos do que a média da
indústria nacional. {[}\ldots{}{]} Com base na pesquisa em psicologia
comportamental e no que a McKinsey chama de ``saúde organizacional'' de
uma empresa, mostramos que as mulheres tendem a encorajar um processo de
tomada de decisão mais participativo, como melhorar o componente
``ambiente de trabalho'' da saúde organizacional. Os homens, enquanto
isso, tendem a tomar ações corretivas com mais frequência quando os
objetivos não são alcançados para reforçar o componente ``coordenação e
controle'' da saúde organizacional. Nem todas as mulheres e homens podem
se enquadrar nessas categorias, é claro. No entanto, a McKinsey mostrou
uma forte correlação entre a saúde organizacional de uma empresa e o
desempenho financeiro. {[}\ldots{}{]} Se as mulheres participassem igualmente
da economia global, poderiam gerar um \versal{PIB} adicional de \versal{US}\$ 28 trilhões
até 2025. Esse montante equivale aproximadamente ao tamanho das
economias chinesa e norte"-americana juntas.\footnote{Cf.
  \textless{}\emph{https://mck.co/2zANJ0y}\textgreater{}.}
\end{quote}

\chapter{As empresas e a agenda da diversidade}

No que diz respeito à questão racial, os dados de 2013 da maior cidade
brasileira não são tão positivos quanto os de gênero (retratados acima),
mesmo se ponderarmos que as pessoas que não se identificam em termos de
cor e raça provavelmente não são brancas.

\medskip

\noindent\versal{DISTRIBUIÇÃO DE CARGOS COM VÍNCULOS ATIVOS NA CIDADE
DE SÃO PAULO POR COR OU RAÇA (EM NÚMEROS ABSOLUTOS)}

\smallskip

\noindent{}\resizebox{\textwidth}{!}{
\begin{tabular}{|l|c|c|c|c|c|c|}
\hline
\textbf{Cor\,/\,Raça} & \textbf{Diretor} & \textbf{Gerente} & \textbf{Supervisor} & \textbf{Funcionário} & \textbf{Aprendiz} & \textbf{Total} \\ \hline
\textbf{Branca} & 17462 & 156973 & 85883 & 2332661 & 16613 & 2609592 \\ \hline
\textbf{Negra} & 1566 & 29969 & 28930 & 1240812 & 13911 & 1315188 \\ \hline
\textbf{Indígena} & 34 & 268 & 221 & 7427 & 35 & 7985 \\ \hline
\textbf{Amarela} & 502 & 4109 & 1816 & 34105 & 146 & 40678 \\ \hline
\textbf{\begin{tabular}[c]{@{}l@{}}Não \\ Identificado\end{tabular}} & 8174 & 10232 & 17248 & 853875 & 1508 & 891037 \\ \hline
\textbf{Total} & 27738 & 201551 & 134098 & 4468880 & 32213 & 4864480 \\ \hline
\end{tabular}
}

\begin{center}
{\scriptsize{Fonte: Rais/\versal{TEM}, 2013, município de São Paulo, setores selecionados para amostra.}}
\end{center}

\medskip

Comparando os dados de brancos e negros temos o seguinte: no município
de São Paulo, dos 2,6 milhões de empregados brancos 0,66\% são
Diretores, 6,01\% são Gerentes e 3,29\% são Supervisores. Já dos 1,3
milhões de empregados negros 0,11\% são Diretores, 2,27\% são Gerentes e
2,19\% são Supervisores. Assim, 6,67\% dos empregados brancos são ou
Diretores ou Gerentes, enquanto 2,38\% dos negros empregados ocupam as
duas posições superiores da hierarquia das empresas. Os brancos ocupam
62,95\% dos cargos de Diretor, contra 5,64\% de diretores negros. Do
mesmo modo, são brancos 77,88\% dos Gerentes, frente a 14,86\% de
gerentes negros.

Na parte conclusiva da Pesquisa de 2016 citada mais acima (Perfil
Social, Racial e de Gênero dos 200 Principais Fornecedores da Prefeitura
de São Paulo) o \emph{Instituto Ethos} faz um alerta às empresas para
que adotem políticas afirmativas voltadas para a agenda da diversidade
no âmbito empresarial, com destaque para a questão racial: ``fica
evidente a necessidade de o mundo empresarial se posicionar e
\emph{compreender que os negros têm uma magnitude socioeconômica
estratégica}'' (p.\,61). O próprio Instituto explicita em todas as suas
publicações sua motivação econômica ao trazer para o debate pesquisas e
análises acerca da inclusão de minorias no âmbito dos cargos de chefia
das empresas, o que por si só devia levar os coletivos negros,
feministas e outros movimentos de esquerda (que se pretendam
anticapitalistas) a analisarem com maior atenção os interesses
confluentes entre suas práticas políticas e as defendidas no mundo
empresarial.

Dentre as ``Atividades estratégicas'' no âmbito de Políticas Públicas, o
próprio \emph{Instituto Ethos} tinha como objetivo, por exemplo, para os
anos de 2014 e 2015, ``promover condicionantes de raça e gênero em
processos licitatórios da Prefeitura de São Paulo'' e a ``aprovação de
cotas para negros em serviços públicos federais''. O movimento é global,
com exemplos em inúmeros países. Em Portugal, por exemplo, há desde 2013
o \emph{iGen"-Forúm Organizações para a Igualdade,} organização que junta
68 empresas e instituições que representam 2\% do \versal{PIB} português e que
estão ``comprometidas a defender igualdade de género em Portugal'',
dentre as quais está a Galp, Nestlé, Jumbo, Microsoft, edp, Coca"-Cola,
Ikea, Associação Dinova Portugal de Intervenção em Toxicodependências e
Desenvolvimento Social, Conceito de Consultoria de gestão, Escola de
Direito da Universidade do Minho, a empresa de segurança Esegur, o
Instituto Português da Qualidade, a Instituto Superior de Economia e
Gestão da Universidade de Lisboa e a Sociedade de advogados Vieira de
Almeida \& Associados. Segundo notícia veiculada na imprensa portuguesa
estas empresas ``todos os anos se comprometem a reavivar boas práticas''
na área de igualdade de gênero, de modo que ``há uma série de medidas
que as empresas sugerem a si próprias e colocam no anexo ao corpo de
adesão'', visando a que sejam implementadas ao longo do ano. ``São metas
que de alguma maneira contribuem para o bem"-estar social \emph{e para
uma produtividade maior}'' e, anualmente, há uma ``monitorização para o
cumprimento das metas''. No ano de 2017, por exemplo, 86\% das medidas
foram cumpridas.\footnote{Cf. \textless{}\emph{https://bit.ly/2SsRzj1}\textgreater{}.}

Em fevereiro de 2019 a grife italiana \emph{Gucci} anunciou que faria
``um grande esforço para aumentar a contratação de diversidade como
parte de um plano de longo prazo para aumentar a conscientização
cultural na empresa de moda de luxo''. A medida vem na sequência de um
alvoroço causado por um suéter preto de \versal{US}\$ 890 que reproduzia uma boca
vermelha caricata, se assemelhando, assim, ao \emph{blackface} (prática
teatral de atores que se pintam com carvão para representar personagens
negros de forma exagerada e caricata). A \emph{Gucci} se comprometeu
também a contratar um diretor global para diversidade e inclusão, cargo
recém"-criado que será baseado em Nova York, além de cinco novos
designers de todo o mundo para seu escritório em Roma e, ainda, lançará
programas de bolsas de estudos multiculturais em 10 cidades do mundo,
com o objetivo de construir um ``local de trabalho mais diversificado e
inclusivo em uma base contínua''. Por fim, e bastante sintomático: o
anúncio foi feito depois que o \versal{CEO} da Gucci, Marco Bizzarri, se
encontrou, no bairro de Harlem, em Nova York, com Dapper Dan, um
renomado designer negro, e outros membros da comunidade, para ``ouvir
suas perspectivas'' e saber ``quais ações a empresa deveria
tomar''.\footnote{Cf.
  \textless{}\emph{https://yhoo.it/2VEDWPs}\textgreater{}.}

Quanto à ``receita para o sucesso'' no que tange às práticas
empresariais voltadas para a agenda da diversidade, Hunt, Yee, Prince e
Dixon"-Fyle, em ``Delivering through diversity'' (McKinsey, 2018) falam
em ``quatro imperativos'':

\begin{quote}
As empresas relatam que melhorar materialmente a representação de
diversos talentos em suas fileiras, bem como utilizar efetivamente a
inclusão e a diversidade como facilitadoras do impacto nos negócios, são
objetivos particularmente desafiadores. Apesar disso, várias empresas em
todo o mundo conseguiram fazer melhorias consideráveis ​​na inclusão e
diversidade em suas organizações, e elas estão obtendo benefícios
tangíveis por seus esforços. Descobrimos que todas essas empresas
desenvolveram estratégias de inclusão e diversidade (\versal{I\&D}) que
refletiam seus objetivos e prioridades de negócios, com os quais estavam
fortemente comprometidos. Quatro imperativos emergiram como sendo
cruciais: \emph{Articular e conjugar o comprometimento do \versal{CEO} para
galvanizar a organização}. As empresas reconhecem cada vez mais que o
compromisso com a inclusão e a diversidade começa no topo, com muitas
empresas se comprometendo publicamente com uma agenda de \versal{I\&D}. As
empresas líderes vão mais longe, colocando em prática esse compromisso
em todas as suas organizações, especialmente para o gerenciamento médio.
Eles promovem a propriedade de seus negócios principais, incentivam a
modelagem de funções, responsabilizam seus executivos e gerentes e
asseguram que os esforços tenham recursos suficientes e sejam apoiados
centralmente. \emph{Defina prioridades de inclusão e diversidade que
se baseiem nos impulsionadores da estratégia de crescimento do negócio}.
Empresas de alto desempenho investem em pesquisa interna para entender
quais estratégias específicas melhor apoiam suas prioridades de
crescimento de negócios. Tais estratégias incluem atrair e reter os
talentos certos e fortalecer as capacidades de tomada de decisões. As
empresas líderes também identificam o mix de características inerentes
(como a etnia) e os traços adquiridos (como formação educacional e
experiência) que são mais relevantes para sua organização, usando
análises avançadas de negócios e de pessoas. \emph{Crie um portfólio
direcionado de iniciativas de inclusão e diversidade para transformar a
organização}. As empresas líderes usam o pensamento direcionado para
priorizar as iniciativas de \versal{I\&D} nas quais investem e garantem o
alinhamento com a estratégia geral de crescimento. Elas reconhecem a
necessidade de construir uma cultura organizacional inclusiva e usam uma
combinação de fiação ``dura'' e ``suave'' para criar uma narrativa
coerente e um programa que ressoa com funcionários e partes
interessadas, ajudando a impulsionar mudanças sustentáveis.
\emph{Adapte a estratégia para maximizar o impacto local.} Empresas
líderes e em rápida evolução reconhecem a necessidade de adaptar sua
abordagem --- a diferentes partes do negócio, a várias geografias e a
contextos socioculturais.
\end{quote}

Já quanto à presença da diversidade na mídia brasileira, o ``\emph{Guia
para Jornalistas sobre Gênero, Raça e Etnia}'', publicado em 2014,
afirma que do ponto de vista étnico e racial há pouca presença de
jornalistas negras e indígenas nas redações e postos de comando das
empresas de comunicação. Essa rara presença das trabalhadoras negras e
indígenas indica, segundo o Guia, uma estrutura ocupacional pouco plural
que ``fortalece a invisibilização dos problemas que afetam as mulheres
negras e indígenas e favorece a visão deslocada e estereotipada da
realidade desses grupos''. (2014, p.\,20). Quanto à presença de mulheres
nos espaços de poder o Guia informa que a plena participação das
mulheres nas decisões que afetam suas vidas e dos grupos aos quais
pertencem ``é um dos pilares para se alcançar a igualdade de gênero e o
empoderamento na perspectiva dos Objetivos de Desenvolvimento do Milênio
-- \versal{ODM}''.\footnote{Cf. \textless{}\emph{https://bit.ly/2y71e7A}\textgreater{}.}

Assim, o aumento da participação das mulheres na esfera política é visto
como uma questão ``estratégica para o desenvolvimento nacional alinhado
aos critérios estipulados pelos \versal{ODM}'' (2014, p.\,30). Embora esteja em
vigor uma lei de cotas que garante um percentual de 30\% das vagas para
candidaturas femininas nos partidos políticos, ``as mulheres brasileiras
continuam enfrentando barreiras impostas pelo sexismo, pelo racismo e
pelo etnocentrismo, o que as coloca em situação de desvantagem na
disputa eleitoral'' (2014, p.\,31). Na disputa por uma vaga na Câmara
Federal, por exemplo, do total de 1.346 candidaturas registradas pelo
Tribunal Superior Eleitoral (\versal{TSE}) em 2010, 22,9\% foram mulheres contra
77,71\% de candidaturas masculinas.

Apropriando"-se do vocabulário identitário referente à busca pela
``visibilização'' das opressões, o \emph{Guia para Jornalistas} coloca
que o enfrentamento às discriminações de gênero, raça e etnia, no âmbito
do jornalismo, ``começa quando profissionais da imprensa assumem o
propósito de visibilizá"-las (as mulheres negras e indígenas)
positivamente por meio de textos, sonoras e imagens na mídia'' (\versal{BASTHI},
2014, p.\,37). Entendendo que ``o discurso predominante nas narrativas
jornalísticas e imagéticas e que, portanto, prevalece para a sociedade
brasileira, é produzido numa perspectiva masculina e de dominação''
(2014, p.\,40) o \emph{Guia} pontua que, na prática, a perspectiva de
gênero com recorte de raça e etnia pode ser aplicada a partir da
``adoção de novos critérios para seleção e produção da notícia'', tais
como:

\begin{quote}
\forceindent{}1) assumir uma postura diversificada na escolha da pauta;

2) utilizar critérios de gênero, raça e etnia para escalar a fonte da
matéria;

3) definir, em caso de situação de risco da fonte, os critérios de
abordagem;

4) usar uma linguagem na perspectiva de gênero, raça e etnia;

5) optar, sempre que possível, por imagens positivas de mulheres negras
e indígenas para ilustrar o conteúdo de qualquer notícia digitalizada,
impressa, eletrônica ou sonora.

Para dar visibilidade às opiniões e imagens das mulheres livres de
estereótipos e numa perspectiva de gênero, raça e etnia, profissionais
da imprensa devem estar atentos a qualquer vestígio de preconceitos e
prejulgamentos que possam interferir na seleção e na construção da
notícia. (2014, p.\,38)
\end{quote}

No documento publicado em 2012 pela \emph{\versal{ONU} Mulheres} e \emph{Rio+20},
intitulado ``\emph{O futuro que as mulheres querem}'', resultado de uma
pesquisa financiada pela \emph{Petrobras}, \emph{Fundação Ford} e
\emph{Itaipu Binacional}, além de alguns governos, lemos algumas linhas
que, não por acaso, também lembram a gramática dos movimentos
identitários em torno do ``lugar de fala'', ``espaços seguros'' e demais
itens que conformam a política identitária e a \emph{teoria dos
privilégios}\footnote{Ver a série de artigos de Kevin Carson, Casey
  Given, Cathy Reisenwitz e Nathan Goodman, disponível aqui:
  \textless{}\emph{https://bit.ly/2y3n4ZN}\textgreater{} e o
  importante texto de Will (2014). Ver também \versal{BRUINIUS}, H. Should
  colleges provide `safe spaces'? de 2016. Disponível em:
  \textless{}\emph{https://bit.ly/2zDDA3j}\textgreater{}.}:

\begin{quote}
A falta de mulheres em posições de tomada de decisão e a ausência geral
de especialistas nas questões de gênero entre os decisores entravam de
modo significativo o engajamento das mulheres --- e a integração das
perspectivas de gênero --- no planejamento urbano e na concepção de
infraestruturas e serviços. Por sua vez, a falta de engajamento limita a
possibilidade de que as necessidades específicas das mulheres sejam
abordadas. (p.\,23)
\end{quote}

Noutras palavras, é defendido que apenas mulheres podem abordar as
necessidades específicas das mulheres, o que conflui com a defesa dos
movimentos feministas do ``protagonismo'' das mulheres na luta
feminista, dos negros na luta antirracista e das lésbicas, gays,
bissexuais, travestis, transexuais e transgêneros e intersexuais nas
lutas \versal{LGBTI}. Alguns dos movimentos de luta inspirados no identitarismo
pós"-moderno vetam aos companheiros e companheiras a construção
horizontal da luta em torno de uma mesma bandeira, posto que o gênero,
raça ou etnia etc. da pessoa a tornaria \emph{potencialmente machista
e/ou racista} etc. e, por isso, essa pessoa só poderia ser, quando
muito, ``apoiadora'' de uma luta ``que não lhe pertence''. A defesa do
``protagonismo dxs oprimidxs'' (com o X remetendo a uma língua
``neutra'') alcançou tal grau de legitimidade\footnote{Pense"-se, por
  exemplo, na pressão social para que mais atoras e atores negros fossem
  indicados e premiados na cerimônia do Oscar, ou no fato de que a
  \emph{Personalidade do ano} da \emph{Time}, em 2017, tenha sido as
  pessoas que denunciaram casos de assédio sexual: ``Publicação
  norte"-americana distinguiu as mulheres e os homens que deram `voz a
  segredos abertos': `Por moverem redes de murmúrios para as redes
  sociais, por nos motivarem a todos a parar de aceitar o inaceitável'.
  Cf. \textless{}\emph{https://bit.ly/3eUBUCz}\textgreater{}.}
que é francamente perigoso criticar suas práticas em público.

Ainda quanto a essa questão, vale citar dois recentes casos, que nos
parecem bastante emblemáticos no que diz respeito à teia de contradições
internas do identitarismo: ``\versal{IMS} Rio cancela evento sobre poesia após
críticas por falta de diversidade''.\footnote{Cf.
  \textless{}\emph{https://bit.ly/2y3i2MV}\textgreater{}.}
``\emph{Oficina Irritada (Poetas Falam)}, que aconteceria em maio, foi
alvo de reclamações por não contar com convidados negros''. No corpo da
matéria somos informados que as reclamações não vieram apenas do
público, mas também de autores da cena literária atual. ``Os espaços
privilegiados seguem cometendo erros em não abrir esse cânone
engessado'', publicou o escritor Jeferson Tenório em sua página no
Facebook.

Em agosto de 2019 o Centro de Estudos Africanos, da \versal{UFMG}, publicou uma
\emph{Nota Pública} onde informava o cancelamento do Simpósio
Internacional ``Novas epistemes para o estudo da África pré"-colonial:
agência africana e conexões'' devido à ``forma desmedida e agressiva das
`críticas' dirigidas à organização e à organizadora'' do evento devido
ao fato de que uma das mesas do Simpósio tinha uma composição racial de
5 pessoas brancas e, esclarece a Nota, dos 28 trabalhos inscritos, 10
eram de pessoas pretas e pardas, 12 de pessoas brancas e as outras 6
``não sabemos porque não as conhecemos. Todas as propostas foram
aceitas, sem escrutínio racial dos proponentes, porque atendiam aos
critérios estabelecidos na chamada pública para comunicações que se
enquadrassem no tema do evento''.\footnote{Cf. Publicação de 06 de
  agosto de 2019 em
  \textless{}\emph{https://bit.ly/2VI50gY}\textgreater{}.}
Em todo o planeta e em distintos setores a composição racial das
empresas tem sido alvo das políticas identitárias. Até mesmo o \emph{The
Guardian} publicou, em 2018, uma matéria tratando da falta de negros em
uma novela brasileira.\footnote{Cf.
  \textless{}\emph{https://bit.ly/358JODG}\textgreater{}.}

Fundamentando"-se na legitimidade inquestionável das questões
relacionadas às opressões e discriminações de raça, gênero e
sexualidade, as bandeiras e pautas das lutas identitárias têm sido
apropriadas por empresas e organizações governamentais para fins
políticos e econômicos. Em contrapartida, as organizações que pautam as
demandas identitárias em sentidos e formas organizativas convenientes
para a manutenção e reforço lucrativo do capitalismo recebem incentivos
que vão desde investimentos diretos até apoios ideológicos na mídia e
nos aparelhos ideológicos de Estados e empresas voltados para a
formatação da visão de mundo e do universo cultural e ideológico das
pessoas.

Não por acaso, um dos maiores financiadores globais dos estudos e
pesquisas feministas é a \emph{Fundação Ford}.\footnote{Ver \versal{SCHULD},
  Kimberly, 2004 e \versal{WOOSTER}, 2004.} Precursora desse tipo de atrelamento
estratégico entre imagem da empresa e institutos e teorias com
legitimidade no bojo da classe trabalhadora (feminismo, antirracismo,
cidadania etc.) é ela quem financia grande parte das edições e pesquisas
das maiores revistas científicas de diversas partes do planeta cuja
produção está centrada no desenvolvimento e divulgação de teorias e
debates sobre o feminismo. No Brasil, por exemplo, a Fundação Ford, que
já financiara o movimento negro\footnote{A \emph{Fundaçāo Ford}
  financiou a instalação do movimento negro no Brasil pós"-ditadura,
  garantindo que este se configurasse desde o início aos moldes do
  movimento negro estadunidense, com sua ênfase nas políticas
  afirmativas e por cotas.}, é próxima e financia inúmeras atividades da
\emph{Revista de Estudos Feministas,} uma das mais conceituadas revistas
da área. Esse apoio extrapola em muito o mero interesse em ter uma
imagem empresarial positiva ou a criação de um novo nicho de
consumidoras que, uma vez ``empoderadas'', manifestarão o próprio poder
do modo como se espera na lógica capitalista: comprando carros
etc.\footnote{O papel político da Revista Estudos Feministas no
  desenvolvimento e germinação das linhas estratégicas de \emph{um certo
  feminismo} é ainda um objeto para pesquisas futuras. Vale a pena, no
  entanto, ler suas publicações a fim de compreender as particularidades
  do tipo de feminismo defendido pela linha editorial da Revista. Ver,
  por exemplo: \versal{ADRIÃO}, K. G. et~al. 2011; \versal{ALVAREZ}, S., 2003; \versal{BARSTED},
  L. L., 2008; \versal{BLACKWELL}, M. \& \versal{NABER}, N, 2002; \versal{COSTA}, C. L., 2003 e
  2013; \versal{DINIZ}, D. \& \versal{FOLTRAN}, P., 2004; \versal{FERREIRA}, E. S. \& \versal{BORGES}, D.
  T., 2004; \versal{GROSSI}, M. P., 2004; \versal{LAGO}, 2009; \versal{MINELLA}, L. S. et~al.,
  2004; \versal{MINELLA}, L. S., 2004; \versal{MAYORGA}, C. et~al., 2013; \versal{MONTESINOS}, V.,
  2003; \versal{MOND}, N., 2003; \versal{MEYER}. D. E. et~al., 2014; \versal{MATOS}, M., 2008;
  \versal{LOPES}, M.M. \& \versal{PISCITELLI}, A., 2004; \versal{VEIGA}, A. M., 2009; \versal{SCHMIDT}, S.
  P., 2004; \versal{SILVA}, C., 2013; \versal{PORTO}, R. M., 2004.}

O direcionamento ideológico e o controle político e econômico da
produção teórica de temas que possuem potencialidade de fomentar
práticas críticas ao capitalismo e outros que estimulam práticas de
reforço do sistema e da lucratividade das empresas não se dá como um
processo linear ou centralizado em uma empresa ou governo de onde
emanariam as diretrizes estratégicas acerca de quais teorias e
organizações devem receber apoio logístico, ideológico e material. Não
se trata de uma teoria da conspiração, portanto, embora alguns estudos,
como por exemplo o artigo ``A \versal{CIA} lê a teoria francesa: sobre o trabalho
intelectual de desmantelamento da Esquerda cultural'' nos tragam
elementos verdadeiramente intrigantes.\footnote{Cf.
  \textless{}\emph{https://bit.ly/2W6XU4F}\textgreater{}.}

A nosso ver o direcionamento estratégico das linhas editoriais de
revistas, jornais, institutos, universidades etc. se dá mais por conta
de uma \emph{confluência de interesses} do que de qualquer teoria da
conspiração. O que temos em geral é uma complexa miríade de teorias
identitárias, multiculturalistas e pós"-estruturalistas que se
desenvolvem de modo desigual e divergem entre si tanto em termos
teóricos quanto em termos da prática social dos militantes por elas
empolgados.

Uma busca por palavras"-chave no diretório de \emph{grupos de pesquisa}
da Plataforma Lattes (palavras presentes no nome do grupo ou na
descrição da linha de pesquisa) resulta no seguinte, para cada entrada
buscada: \versal{IDENTIDADE}: 2053 registros. \versal{GÊNERO}: 1861 registros.
\versal{SEXUALIDADE}: 611 registros. \versal{NEGRA}: 156 registros. \versal{NEGRO}: 193 registros.
\versal{RACISMO}: 151 registros. \versal{FEMINISMO}: 183 registros. \versal{FEMINISTA}: 124
registros. \versal{CLASSE SOCIAL}: 57 registros. \versal{TRABALHADORES}: 226 registros.
\versal{OPRESSÃO}: 13 registros. \versal{REVOLUÇÃO}: 36 registros. \versal{COMUNISMO}: 13
registros. \versal{SOCIALISMO}: 18 registros. \versal{HOMOFOBIA}: 33 registros. \versal{GREVE}: 9
registros. \versal{SALÁRIO}: 27 registros. \versal{MULHER}: 816 registros. \versal{MULHERES}: 356
registros. \versal{HOMENS}: 30 registros. \versal{LGBT}: 37 registros. \versal{GAY}: 9 registros.
\versal{LÉSBICA}: 14 registros. \versal{TRAVESTI}: 14 registros. \versal{TRANSGENERO}: 5 registros.
\versal{QUEER}: 49 registros. \versal{CAPITALISMO}: 201 registros. \versal{DESIGUALDADE}: 323
registros. \versal{DOMINAÇÃO}: 43 registros. \versal{EXPLORAÇÃO}: 32 registros. \versal{POLÍTICA}:
4993 registros. \versal{ECONOMIA}: 719 registros. \versal{DISCURSO}: 624
registros.\footnote{Busca feita em novembro de 2018. Na entrada
  ``exploração'' foram excluídos os casos onde se tratava de exploração
  da natureza, exploração de cavernas etc. e mantidas as entradas onde o
  tema era a exploração econômica.~Na entrada ``gênero'' apliquei o
  filtro para os grupos da área de Ciências Humanas (1059), Ciências
  Sociais Aplicadas (375) e Letras, Linguística e Artes (427). Nesta
  última, entretanto, há casos em que a entrada diz respeito ao ``gênero
  textual''.}

No plano do pensamento social, o fato de que certas publicações e certas
linhas de pesquisa recebam fomento e contem com um complexo apoio
econômico e político não deve, no entanto, ser subestimado: a facilidade
de acesso das pessoas a certas teorias contém sempre atrás de si a
dificuldade de acesso a outras, o que dá às empresas financiadoras e
instituições financiadas uma vantagem nada desprezível em termos da
disputa por hegemonia no plano das teorias e práticas da luta contra o
feminismo, o racismo em tais e tais termos, sob tal e tal fundamento
teórico, em tal ou qual forma organizativa. Além, é claro, do fato de
que direcionadas para certo sentido (multiculturalista, identitário,
pós"-estruturalista, pós"-moderno etc.) tais problemáticas deixam de
desembocar ou de ter algo a ver com anticapitalismo. Digno de nota,
muitas vezes a posição de alguns movimentos de luta pela causa negra ou
feminista chega a ser mais retrógrada que a defendida por organizações 
como a \versal{ONU} e entidades empresariais, como veremos a seguir. 

\chapter[A \versal{ONU} Mulheres nas empresas, governos e universidades]{A \versal{ONU} Mulheres nas empresas,\break governos e universidades}

No Plano Estratégico da \emph{\versal{ONU} Mulheres} para 2014-2017, por exemplo,
lemos no parágrafo 33C que ``alcançar a igualdade de gênero requer uma
abordagem inclusiva, que reconheça o papel essencial dos homens como
parceiros na busca pelos direitos das mulheres''. Entendendo que
``Quando os homens têm priorizado a igualdade de gênero, grandes
resultados têm sido conseguidos'' (p.\,19), a campanha \emph{ElesPorElas}
(\emph{HeForShe}) ``fornece uma plataforma para que os homens se
identifiquem com as questões da igualdade de gênero e com seus
benefícios, que tem o poder de libertar não só as mulheres, mas também
os homens de papéis sociais prescritos e de estereótipos de gênero''.

Dentre as medidas práticas para o enfrentamento das desigualdades de
gênero, longe de reforçar alguns dos valores ``excludentes'' do
identitarismo, é posta a tarefa de ``Explorar os obstáculos à igualdade
de gênero em seu país e incentivar os homens a encontrarem soluções
inovadoras'' (p.\,9). No módulo referente às Universidades a questão dos
``espaços seguros'' é posta como um problema social que aflige ambos os
sexos, sendo tarefa do \emph{ElesPorElas} ``3. Desenvolver programas e
políticas para garantir que tanto estudantes do sexo feminino quanto
estudantes do sexo masculino estejam seguros no campus, e medidas de
emergência que garantam o seu acesso à justiça e que sua dignidade e
direitos sejam respeitados no caso de agressão ou de má conduta''
(p.\,12), ``6. Incentivar funcionários e professores do sexo masculino a
se imporem em nome da campanha, especialmente em departamentos e áreas
onde a igualdade de gênero possa ser mais difícil de ser observada''. E
``7. Organizar ou incentivar os alunos a organizarem, reuniões,
conferências e eventos especiais destinados a aumentar a conscientização
sobre a importância da presença dos homens na luta pela igualdade de
gênero''.

No Brasil, a Universidade de São Paulo aderiu ao movimento\footnote{Cf.
  \textless{}\emph{https://bit.ly/2zAOHtI}\textgreater{}.},
mas neste espaço o identitarismo está tão arraigado que a diretriz da
\emph{\versal{ONU} Mulheres} foi corrompida: a articulação entre homens e
mulheres na luta feminista foi posta de modo hierárquico, não como
``possibilidade'' e ``escolha'', mas como ``dever''.
%, como se nota em um dos cartazes da Campanha:

%\begin{center}
%\includegraphics[width=4cm]{./imgs/img1.png}
%\end{center}

Na plataforma prática de enfrentamento às agressões contra as mulheres,
por exemplo, a Campanha \emph{ElesPorElas} coloca a questão da seguinte
forma:

\begin{quote}
Acabar com a violência contra as mulheres requer uma abordagem
abrangente que envolva ativamente todos os segmentos da sociedade. Dessa
forma, todos os homens têm um papel a desempenhar. O movimento
ElesPorElas (HeForShe) fornece uma plataforma na qual, primeiramente e
acima de tudo, homens e meninos \emph{podem} prevenir a violência
contra mulheres e meninas estando conscientes sobre suas próprias
atitudes, valores e comportamentos em relação às mulheres, mudando
quando necessário, e garantindo que eles não se envolvam pessoalmente em
nenhuma discriminação ou violência. Nos casos em que a violência já foi
perpetrada, os homens \emph{podem} tomar a iniciativa de buscar apoio
para mudar o seu comportamento de forma a não cometer violência
novamente. {[}\ldots{}{]} Em segundo lugar, homens e meninos \emph{podem}
ser proativos e ``intervir'' quando testemunharem discriminação ou
violência por outros homens. A intervenção \emph{pode} assumir muitas
formas. Por exemplo, homens e meninos \emph{podem} expressar desagrado
quando os colegas estiverem fazendo comentários sexistas, degradantes ou
depreciativos. Eles \emph{podem} apoiar os amigos a fazerem escolhas
respeitosas em situações precárias ou de risco (por exemplo,
impedindo"-os de fazer abusos sexuais a uma mulher que esteja drogada).
Se forem testemunhas de uma situação de violência, eles também
\emph{podem} intervir diretamente após avaliar o contexto e determinar
que nenhum mal poderá vir a prejudicar a mulher ou a si mesmos, ou
contatar as autoridades competentes, como a polícia, para intervir.
(p.\,17 e 18, grifos meus).
\end{quote}

Enquanto os documentos da \emph{\versal{ONU} Mulheres} utilizam o verbo
``podem'', indicando a horizontalidade da luta e da tomada de posição
política de homens no enfrentamento do machismo, o identitarismo da
Universidade brasileira utiliza a palavra ``deve'', indicando a
existência de algum tipo de dívida e obrigação dos homens para com a
luta pela igualdade de gênero, abrindo a brecha para as rotineiras
práticas punitivistas (escracho, exposição etc.) contra homens que se
considere estarem agindo em contra sua dívida histórica. Aqui os homens
são considerados agressores em potencial, culpados pelo machismo e
patriarcalismo, numa biologização da questão que nem mesmo a \versal{ONU}
reproduz\footnote{Salvo alguns deslizes, por exemplo quando afirma que
  ``Se cada homem assumisse a responsabilidade por si mesmo, isso por si
  só seria suficiente para acabar com a violência contra as mulheres''
  (p.\,17), o que negligencia a questão da violência contra as mulheres
  perpetrada por mulheres.}, malgrado todo o seu interesse político,
econômico e ideológico na integração das mulheres. Assim, o machismo
deixa de ser visto como um problema que é construído nas relações
sociais entre as pessoas (e que, portanto, só pode ser solucionado
nestas mesmas relações) e passa a ser visto como causa exclusiva da
atuação machista \emph{dos homens} nestas relações. Ao mesmo tempo, o
acento no ``deve'' reforça a hierarquia interna das relações econômicas,
políticas e ideológicas sob a égide identitária: respeite seu chefe (ou
líder ativista), em especial se ele for negro, mulher ou \versal{LGBTIQ}.
``Respeite a minha história'', ``meu nome é \versal{XXXX}, sou mulher, negra e
periférica\ldots{}'', são frases que costumeiramente são forjadas visando
legitimar de antemão o valor do discurso e da atuação política a ser
defendida a seguir, o que opera um silenciamento e marginalização
daqueles sujeitos políticos que porventura não tenham tantas
``qualidades de oprimido'' para enumerar em sua apresentação pessoal.

A exclusão dos homens das políticas de combate ao machismo, ou sua
alocação como ``apoiadores'' contrapostos às ``protagonistas'' é ainda
mais trágica quando se observa, por exemplo, a questão do estupro de
mulheres enquanto ``tática'' e ``espólio'' de guerra e o estupro de
homens nas prisões masculinas. No enfrentamento a essa questão a
\emph{\versal{ONU} Mulheres} está à frente de muitos coletivos feministas do
campo identitário, reconhecendo não apenas a horizontalidade de homens e
mulheres nesta luta, mas o próprio ``protagonismo'' dos homens no
enfrentamento desta faceta violenta do machismo em ambiente bélico:

\begin{quote}
Uma das iniciativas globais da \versal{ONU} Mulheres, desde 2011, vem sendo o
treinamento de soldados militares de paz para prevenir e responder à
violência sexual em suas áreas de atuação. Aproximadamente 90\% dos
destinatários desta formação, que foi fornecida a milhares de forças de
paz, em 18 países que contribuem com tropas, até agora, têm sido homens,
devido aos baixos percentuais de mulheres nas forças armadas. A \versal{ONU}
Mulheres também vem oferecendo cursos online específicos sobre a
Resolução 1325 sobre mulheres, paz e segurança. Nos últimos três anos,
25 mil pessoas participaram destes cursos online e, aproximadamente,
80\% eram homens. (p.\,21)
\end{quote}

Sendo os preconceitos e discriminações frutos de relações sociais, cabe
às pessoas constituírem novas relações sociais, num processo em que
ambos os polos devem ser ativos na criação dessas novas modalidades de
relacionamento não mais machista, racista, homofóbico, xenófobo etc.
Nenhum esclarecimento, autocrítica ou ``desconstrução'' individual
avançará um milímetro sequer no enfrentamento da problemática geral, que
é estrutural, sistêmica e histórica. Aliás, um elemento frequentemente
usado nas críticas identitárias depõe contra suas próprias estratégias,
mostrando o quanto elas são condenadas de antemão ao fracasso: o fato de
que pessoas ``progressistas'' e ``de esquerda'' sejam também machistas,
racistas e homofóbicos, os chamados ``esquerdomachos''. Ora, se pessoas
mais conscientes que a média da população reproduzem na prática os
preconceitos e discriminações isso só prova que não é por meio de níveis
de ilustração e de autoreconhecimentos de ``privilégios'' que se
avançará na luta contra o sistema que repõe a lógica de tais
discriminações e privilégios, a começar pelo privilégio de explorar ao
invés de ser explorado.

\looseness=-1
Concluindo o tópico, não obstante o caso brasileiro ainda apresente
níveis significativos de desigualdade de gênero e cor no que tange ao
acesso a cargos de comando, níveis salariais e etc., a lista de
elementos que apontam para o \emph{caráter integrável} das mulheres e
negros no sistema capitalista, dando a este novo fôlego econômico, é
imensa, como pudemos ver em inúmeros dados apontados até aqui. O
filósofo húngaro István Mészáros defendia que a integração das mulheres
é impossível dentro do capitalismo, ou seja, constitui um ``limite
absoluto'' do sistema do capital, apontando para o caráter estrutural de
sua crise.\footnote{Cf. Capítulo 5 de \emph{Para além do capital}
  (2002). Refinando sua análise, posteriormente, Mészáros passa a
  articular este item à ideia de impossibilidade de uma ``igualdade
  substantiva'' dentro do sistema do capital, o que nos parece mais
  sensato, já que este sistema é um sistema de exploração que, enquanto
  sistema de exploração, assenta na desigualdade estrutural entre
  agentes que exploram e agentes que são explorados. O termo, ainda, tem
  a vantagem de englobar as diversas desigualdades substantivas
  relacionadas ao gênero, raça, sexualidade, nacionalidade etc.} Nossa
pesquisa parece mostrar, contudo, que um capitalismo colorido, que
mantenha suas estruturas de exploração e, no entanto, tenha nos postos
de comando dos trabalhadores e trabalhadoras pessoas de ambos os sexos,
cores, sexualidades etc. é perfeitamente concebível, o que nos remete
para a questão da ``sustentabilidade''.

\chapter[Empoderamento feminino e desenvolvimento sustentável]{Empoderamento feminino e \break desenvolvimento sustentável}

Uma das ideologias mais fortes do capitalismo atual tem sido a do
desenvolvimento sustentável enquanto alternativa para a perpetuação dos
níveis de crescimento econômico e lucratividade das empresas. Ao
contrário do que se poderia pensar, este tema não é posto em separado do
tema da integração das mulheres, pelo contrário.

O documento ``O futuro que as mulheres querem'' articula com grande
detalhamento o empoderamento feminino e o desenvolvimento sustentável,
colocando como objetivos a criação de ``um ambiente propício para a
igualdade de gênero no desenvolvimento sustentável'', onde 1) se
priorize ``a igualdade de gênero e o empoderamento das mulheres em
políticas e estratégias sobre comércio, cooperação para o
desenvolvimento, investimento externo direto, transferência de
tecnologia e desenvolvimento de capacidades''; 2) se integrem ``as
perspectivas de gênero ao planejamento e orçamento nacionais, e aos
mecanismos de implementação, monitoramento e avaliação, a fim de alinhar
os compromissos relativos à igualdade de gênero com os objetivos do
desenvolvimento sustentável''; 3) se garantam ``investimentos
financeiros específicos para a igualdade de gênero e o empoderamento das
mulheres em todos os programas e projetos, inclusive para programas
comunitários e apoio à infraestrutura local''; e 4) se engajem
``mulheres cientistas, inovadoras e decisoras de forma plena nos
processos de desenvolvimento e concepção de tecnologias verdes''. (p.\,38)

É informado, ainda, que em abril de 2012 uma coalizão de nove países
(África do Sul, Austrália, Dinamarca, Emirados Árabes Unidos\footnote{É
  um pouco irônico que um dos maiores poluidores seja signatário da
  iniciativa.}, Estados Unidos, México, Noruega, Reino Unido e Suécia)
lançou a iniciativa ``\emph{Clean Energy Education and Empowerment}''
(Educação e Empoderamento em Energia Limpa) com o objetivo de ``atrair
mais mulheres jovens para carreiras e posições de liderança nesse
setor'' (p.\,25). Noutro momento, depois de defender que mulheres ocupem
``empregos verdes'' o documento advoga a importância de que se adotem
``medidas explícitas para assegurar que as mulheres não sejam
marginalizadas em setores de empregos precários e mal remunerados'',
sendo também ``necessários esforços para garantir trabalho decente tanto
para mulheres quanto para homens, e apoiar a evolução de carreira das
mulheres para cargos de direção'' (p.\,28). Não é por acaso, ainda, que o
documento se vale de um conceito caro a muitos movimentos feministas, o
de patriarcado, a fim de defender uma mudança de paradigma rumo a uma
maior integração das mulheres em cargos e postos de comando na economia
e política:

\begin{quote}
A exclusão e sobrerrepresentação (sic) das mulheres em áreas
educacionais específicas estão fundamentadas na discriminação
socioeconômica que persiste dentro das famílias, das comunidades, dos
mercados e dos Estados. Tal discriminação está, com frequência,
enraizada nos princípios patriarcais de família e nos sistemas de
parentesco que veem apenas o menino como um membro permanente da família
de origem e sua linhagem econômica e social. As meninas, por sua vez,
são vistas como membros da futura família do cônjuge, destinadas pelo
casamento a contribuir com outra família e, portanto, vistas
frequentemente como dispensáveis, não merecedoras do investimento em
educação ou outros recursos --- sobretudo quando eles são escassos. O
maior investimento em educação é consequentemente feito nos meninos. Os
estereótipos familiares das mulheres como donas de casa e dos homens
como provedores --- e os estereótipos vinculados da mulher ``nutridora'',
emotiva, fisicamente atraente, e do homem racional, adepto da tecnologia
e intelectual --- continuam a manter as mulheres à margem da educação e
excluí"-las de certos empregos. Alguns fatores contribuem para essa
exclusão: segregação sexual, casamento precoce, controle sobre a
mobilidade e interação das mulheres, falta de infraestrutura e
instalações escolares adequadas a elas, falta de segurança no transporte
e no ambiente escolar e a falta de apoio público para a educação das
mulheres. {[}\ldots{}{]} Os conceitos de igualdade de gênero, empoderamento
das mulheres e direitos humanos devem ser integrados na educação
primária, secundária, terciária e superior. Devem ser tomadas medidas
para encorajar tanto mulheres quanto homens a seguir áreas de estudo não
tradicionais, como as ciências e a tecnologia para as mulheres, e a
enfermagem e outras áreas de cuidado para os homens. (p.\,36)
\end{quote}

\chapter{Sete princípios de empoderamento das mulheres}

Grandes empresas transnacionais estão articuladas em torno da integração
das mulheres em um modo economicamente vantajoso. Um bom exemplo para
ilustrar isso é o documento \emph{Princípios de Empoderamento das
Mulheres}, publicado pela \versal{ONU} Mulheres. Nele são expostos sete
princípios.

\section{A liderança promove a igualdade de gênero}

O primeiro deles, ``A liderança promove a igualdade de gênero'', nos
informa que no âmbito do movimento \emph{HeForShe}, \versal{CEO}s de diversas
empresas, como \emph{Mckinsey \& Co, PwC, Schneider Eletric, Twitter} e
\emph{Unilever} declararam compromissos globais para alcançar a
igualdade de gênero nas suas empresas. \emph{Avon, Itaipu} e
\emph{\versal{KPMG}}, que fazem parte do grupo impulsionador do movimento
\emph{ElesPorElas} (\emph{HeForShe}) no Brasil, e seus \versal{CEO}s também têm
feito declarações públicas em apoio ao movimento. O \versal{CEO} global da
\emph{Coca"-Cola Company}, Muhthar Kent, assumiu em 2010 o compromisso
público de empoderar cinco milhões de mulheres até 2020. Já Pierre
Nanterme, \versal{CEO} Global da \emph{Accenture}, gravou depoimento público
sobre seu compromisso por um mundo melhor para sua filha. Na
\emph{Itaipu}, a Política de Equidade de Gênero faz parte do mapa
estratégico da empresa e é implementada em todas as esferas, incluindo
Diretoria, Conselho e Fundações em que a Entidade for mantenedora, bem
como no relacionamento institucional com outros órgãos. Além dos
compromissos assumidos pela \emph{\versal{PWC}} e \emph{Unilever}, também o
presidente global da \emph{Schneider Electric} assumiu publicamente a
meta de, até 2017, aumentar o número de mulheres na entrada (cargos
iniciais) para 40\% e no topo para 30\%.

De acordo com o documento \emph{Princípios de Empoderamento das
Mulheres} uma prática cada vez mais disseminada entre as empresas são os
\emph{grupos de afinidades} (\emph{women's network, business resource
group}) que discutem a igualdade de gênero: várias empresas signatárias
dos Princípios da \emph{\versal{ONU} Mulheres} já têm essas iniciativas, como
\emph{Braskem, Dow, \versal{IBM} e Walmart}. Outras empresas, que participam do
programa Pró"-Equidade de Gênero e Raça do governo brasileiro têm o
comitê sugerido pelo programa, como \emph{Eletrobrás, Embrapa e Itaipu}.
Além disso, já existem no Brasil ``grupos formados por empresas que
tratam a questão de \emph{igualdade de gênero no nível estratégico},
contando com o engajamento de sua alta liderança''. De acordo com o
documento ``elas se reúnem para trocar boas práticas e fomentar o avanço
dessas discussões no ambiente empresarial'' (p.\,7). São exemplos: a
Aliança pelo Empoderamento das Mulheres, os Ciclos de Encontros
Regionais para o Fortalecimento da Equidade de Gênero e Raça no Mundo do
Trabalho, o \emph{Finance Women Network} (Rede de Mulheres em Finanças)
e o Movimento Mulher 360.\footnote{``Itaú Unibanco adere ao Movimento
  Mulher 360''. Disponível em:
  \textless{}\emph{https://bit.ly/3bUnPms}\textgreater{}.}
Por fim, algumas empresas têm dedicado orçamento específico às ações de
promoção da igualdade de gênero, como \emph{Accenture, Avon, Coca"-Cola
Brasil, Eletrobras, Eletronorte, \versal{EY}, Itaipu, \versal{KPMG}, Lojas Renner,
Nogueira, Elias, Laskowisk e Matias Advogados, PwC e Vale}.

\section{Igualdade de oportunidades, inclusão~e~não discriminação}

O princípio dois, ``Igualdade de oportunidades, inclusão e
não discriminação'', expõe algumas práticas voltadas à igualdade de
gênero nas seleções para oportunidades de trabalho.\footnote{Vale
  lembrar que pesquisas de 2007, feitas pela McKinsey \& Co sugeriam que
  as mulheres não conseguiam progredir em suas carreiras porque eram
  menos ambiciosas, não buscavam promoção ou optavam por abandonar o
  canal corporativo. Pesquisas subsequentes, de 2013 a 2016 trouxeram
  outro cenário. Em 2013, 79\% das mulheres de nível médio e sênior
  entrevistadas em nível mundial estavam interessadas em alcançar uma
  posição de alta gerência, o que representa a mesma proporção que os
  homens. Do mesmo modo, o relatório \emph{Women in the Workplace,} de
  2016, mostrou que, nos Estados Unidos, 74\% e 80\% das mulheres e dos
  homens, respectivamente, almejavam a promoção.}

Visando práticas de inclusão da diversidade e garantia de oportunidades
de entrada e ascensão dentro da empresa, a \emph{Accenture, Coca"-Cola
Brasil} e \emph{Walmart}, por exemplo, realizam painéis de entrevistas
formados obrigatoriamente por homens e mulheres. As mesmas empresas,
junto com outras, como por exemplo, a Embrapa e a \versal{IBM}, ``inserem em suas
avaliações de desempenho uma preocupação especial em evitar vieses e
estereótipos de gênero'' (p.\,9). \emph{Avon} e \emph{Dow} exigem
candidatas mulheres na mesma proporção de homens para as entrevistas, já
a empresa \emph{Coca"-Cola Brasil} possui uma meta estratégia específica
para área de finanças e operações a fim de equilibrar a participação de
homens e mulheres nessas funções, esforço também seguido pela \emph{Dow}
nas áreas de engenharia e manufatura.

As empresas \emph{Eletronorte} e \emph{\versal{PWC}} praticam a política de que
todas as vagas da empresa, em qualquer nível, precisam ter mulheres
candidatas. A \emph{Caixa Econômica Federal} incluiu em suas normas a
recomendação de que as bancas de avaliação de competências em seus
processos seletivos internos tenham composição equânime em gênero
(formadas por homens e mulheres), como uma das medidas para assegurar
oportunidades iguais a todos e todas. O \emph{Banco do Brasil} promoveu
o Fórum Equidade de Gênero em 2015 a fim de fomentar a discussão da
igualdade de gênero focando na ascensão profissional, equilíbrio entre
vida pessoal e profissional, educação corporativa e conscientização. O
``Guia de Conduta'' da \emph{Petrobras}, aprovado em novembro de 2014,
apresenta o que a empresa compreende por discriminação, assédio moral,
liberdade religiosa, respeito à diversidade e à igualdade de gênero. A
\emph{Eletrobrás Eletronorte} criou a cláusula ``Da Promoção da Equidade
de Gênero e Raça'' no seu manual de práticas de contratação. A ação
resultou em mais contratações de mulheres nas funções de motorista e
vigilante, ocupadas majoritariamente por homens. Para mostrar que
``valoriza a pluralidade de culturas, origens, raça, classes sociais,
gênero e orientação sexual'', o \emph{Carrefour} realizou uma campanha
interna sobre diversidade, abordando ``a importância do respeito às
diferenças''.

\section{Saúde, segurança e fim da violência}

Há também um conjunto de medidas empresariais sendo tomado quanto à
questão da ``Saúde, segurança e fim da violência'', o terceiro princípio
de empoderamento das mulheres desenvolvido no documento da \versal{ONU}. Para
``facilitar os cuidados com filhos e filhas e propiciar uma convivência
familiar mais próxima na primeira infância'', \emph{Avon, Boticário} e
\emph{Unilever} oferecem creches em algumas de suas unidades. A empresa
\versal{EY} oferece às gestantes orientações desde o momento em que a mulher
comunica a gravidez à empresa até o seu retorno após a
licença"-maternidade, sendo garantido um retorno ao trabalho com agenda
flexível, \emph{coaching} para seu momento profissional
e mentoria com uma profissional que já passou pela situação. No quesito
flexibilidade no trabalho, há seis anos foi implantada a Política de
Flexibilidade e Arranjos de Trabalho Flexível, que disponibiliza
trabalho remoto, carga semanal comprimida e jornada reduzida. Os
benefícios dos tratamentos não convencionais de saúde vêm sendo
percebidos pelas empresas, que passam a incluí"-los em seus planos de
saúde para os funcionários e funcionárias. \emph{Braskem, \versal{KPMG} e PwC},
por exemplo, cobrem psicoterapia. Por sua vez, \emph{Coca"-Cola Brasil},
\emph{Eletrobras} \emph{Eletronorte} e \emph{Vale} são empresas que dão
cobertura para a acupuntura. Também são cada vez mais populares os
patrocínios às academias e grupos de corrida.

Muitas empresas decidiram estender para 180 dias a licença"-maternidade,
como \emph{Braskem, Coca"-Cola Brasil} e \emph{Walmart.} A \emph{Caixa
Econômica Federal, Eletrobras, Embrapa, \versal{IBM}, Petrobras} e \emph{PwC},
além dos 180 dias de licença"-maternidade, também dispõem de
licença"-paternidade superior aos cinco dias previstos em lei. A
\emph{Petrobras} oferece 29 salas de amamentação instaladas
em refinarias, campos de exploração e fábricas\,de fertilizantes. A
\emph{Braskem} adaptou o uniforme de suas funcionárias às suas
necessidades específicas para aumentar o bem"-estar e a segurança. A
\emph{Embrapa} oferece auxílio financeiro mensal para filhos ou
dependentes com deficiência, para despesas com tratamentos e/ou
escolas especializadas. Além do auxílio, há a possibilidade de redução
da jornada de trabalho em casos de necessidade de assistência
comprovada.

Em relação às jornadas flexíveis, a \emph{Embrapa} também disponibiliza
essa opção para funcionários e funcionárias que estejam cursando
pós"-graduação. O \emph{Banco do Brasil} está realizando um projeto
piloto para criação de uma política de \emph{home"-office} (trabalho
remoto) e, dentre as práticas de flexibilidade, privilegia as mulheres
na escolha das agências em que vão trabalhar para que possam estar perto
de casa e conciliar demandas familiares (p.\,11).

\section{Educação e formação}

Quanto à educação e formação a visão estratégica de longo prazo das
empresas transnacionais salta aos olhos. A perspectiva de gênero,
voltada para programas e oportunidades de crescimento empresarial
``equânime'' entre funcionários e funcionárias tem começado a fazer
parte dos treinamentos em empresas como \emph{Bloomberg, Deloitte e \versal{JP}
Morgan. Accenture, Dow, \versal{EY}, \versal{KPMG}} e \emph{Unilever} têm programas de
mentoria específico para mulheres. A \emph{Avon} patrocinou a criação de
uma disciplina sobre a sub"-representação da mulher em espaços como o
alto escalão de empresas e a política no curso de graduação em
administração de empresas e administração pública da \emph{Fundação
Getúlio Vargas} (\versal{FGV}), em São Paulo. A Itaipu tem ações para promover e
incentivar as meninas e mulheres a seguir carreiras tecnológicas e
contribuir para o desenvolvimento da inovação. A \emph{Caixa Econômica
Federal} realiza periodicamente eventos de sensibilização dos
funcionários e funcionárias sobre os temas de gênero, diversidade e
violência doméstica.

A \emph{Whirlpool} identificou que a ambição das mulheres da empresa em
chegar a cargos de diretoria era 40\% inferior à dos homens. Feito o
diagnóstico, a empresa começou um trabalho para identificar o que suas
funcionárias encaram como barreira, para tentar criar políticas que as
removam. A \emph{Unilever} tem o \emph{Programa Aquarela}, cujo objetivo
é ser mais uma ferramenta de inclusão social: o programa seleciona
meninos e meninas de baixa renda no ensino médio público, em igual
proporção, com potencial para desenvolver uma carreira na empresa, e
acompanha os mesmos oferecendo suporte para ingresso e conclusão da
faculdade. (p.\,13).

\section{Desenvolvimento empresarial e práticas da cadeia~de~fornecedores~e~de~marketing}

Quanto ao quinto princípio de empoderamento das mulheres,
``Desenvolvimento empresarial e práticas da cadeia de fornecedores e de
marketing'' o documento elaborado pela \emph{\versal{ONU} Mulheres} nos informa
as seguintes práticas empresariais: a \emph{Caixa Econômica Federal}
incluiu a perspectiva da Diversidade --- por meio dos eixos temáticos de
gênero, raça/cor e etnia, orientação sexual e identidade de gênero, e
pessoas com deficiência --- em sua Política de Comunicação, qualificando
a maneira como retrata seus e suas clientes e empregados e empregadas em
suas peças de marketing e endomarketing. A \emph{Avon} lançou, por meio
do \emph{Instituto Avon}, a campanha Fale Sem Medo --- Não à Violência
Doméstica, em que usava o mote ``a violência não pode ser maquiada'', no
ano de 2015. A campanha mostrando que é possível se libertar do ``não é
pra mim'' da marca \emph{Quem disse, Berenice?}, do \emph{Grupo
Boticário}, e a Campanha Beleza Real da Dove, da \emph{Unilever},
investem na desconstrução do estereótipo da mulher ideal. A agência de
publicidade \emph{Heads} lançou a campanha \emph{Todxs por Elas}, que
promove conscientização para sua clientela e setor sobre como superar
estereótipos e adotar linguagem e representação mais inclusiva em sua
comunicação.

Em 2011, o \emph{Walmart} lançou uma campanha para ajudar a capacitar as
mulheres em toda a sua cadeia de suprimentos. Estabeleceu metas
ambiciosas para o prazo de 5 anos: dobrar o volume de mercadorias
adquiridas de mulheres fornecedoras internacionalmente; promover
treinamento e oportunidades de acesso ao mercado às mulheres na
agricultura e nas fábricas; realizar
capacitação formal a mulheres de baixa renda de diferentes áreas de
atuação para ajudá"-las a chegar ao ensino superior e terem acesso ao
mercado de trabalho; apoiar seus fornecedores e fornecedoras a aumentar
a diversidade de gênero em suas empresas. \emph{Accenture, Apple,
Cargill, Cisco, Coca"-Cola Brasil, Cummins, \versal{EY}, \versal{HP}, \versal{IBM}, Johnson \&
Johnson, Microsoft, P\&G, Pfizer e Sodexo}, dentre outras, são empresas
engajadas na \emph{\versal{ONG WE} Connect}, que contribui para o acesso a
mercado para negócios liderados por mulheres. A \emph{\versal{IBM}} tem uma
pessoa exclusiva em compras para verificar o trabalho dos diversos
fornecedores e fornecedoras, avaliando metas anuais para aumentar as
compras de empresas fornecedoras lideradas por grupos historicamente
discriminados.

O \emph{\versal{SESI} Paraná} disponibiliza desde 2011 a publicação ``Relações de
Gênero na Indústria: Metodologia \versal{SESI"-PR} em Prol da Equidade para
indústrias parceiras'', que tem o objetivo de compartilhar conhecimento
para a implantação de procedimentos de gestão comprometidos com a
igualdade de gênero. Foi incluída a ``lente de gênero'' no Manual de
Compras Sustentável do Conselho Empresarial Brasileiro para o
Desenvolvimento Sustentável (\versal{CEBD}s), patrocinado por \emph{Banco do
Brasil, Braskem, Caixa Econômica Federal, Coca"-Cola Brasil, Eco Frotas,
Eco Benefícios, Expers, Itaú, Mapfre Seguros} e o governo federal. Há
ainda a promoção de pesquisas e estudos sobre os temas relacionados à
igualdade de gênero e sua relação com o desenvolvimento econômico por
empresas como \emph{Bain \& Company, \versal{EY}, \versal{KPMG}, McKinsey \& Co e PwC}
(p.\,15).

\section{Liderança comunitária e envolvimento}

Quanto ao sexto princípio, ``Liderança comunitária e envolvimento'', o
documento da \versal{ONU} cita as seguintes iniciativas empresariais: criado com
o objetivo de empoderar e gerar renda para jovens de 15 a 25 anos
através de formação técnica, comportamental e encaminhamento ao mercado
de trabalho, o \emph{Coletivo Coca"-Cola} já formou mais de 100 mil
pessoas, sendo a maioria mulheres. O \emph{Instituto Avon} tem suas
ações destinadas ao combate ao câncer de mama e a violência contra as
mulheres há mais de 12 anos, contando com investimentos próprios e dos
consumidores que apoiam suas diferentes causas. O \emph{Instituto Lojas
Renner}, por sua vez, investe em projetos para o empoderamento das
mulheres desde 2008, com iniciativas que fomentam o empreendedorismo
feminino por meio de capacitação profissional e acesso a mercado.

A \emph{FoxTime} exerce uma função protagonista em projeto piloto que
apoia o recomeço da vida profissional e pessoal de 20 mulheres
refugiadas em São Paulo, em conjunto com a Agência da \versal{ONU} para
Refugiados (\versal{ACNUR}), o Pacto Global, o Programa de Apoio para Recolocação
dos Refugiados (\versal{PARR}) e a \emph{\versal{ONU} Mulheres}.

Várias das empresas signatárias dos \emph{Princípios de Empoderamento
das Mulheres} têm projetos de conscientização comunitária que incentivam
os funcionários e funcionárias a dividir igualmente com as companheiras
e os companheiros as tarefas domésticas e também promovem a paternidade
responsável, como \emph{Nogueira, Elias, Laskowisk e Matias Advogados,
PwC} e \emph{Walmart}.

\emph{Accenture, Coca"-Cola Brasil} e \emph{\versal{IBM}} mensuram o percentual de
mulheres impactadas por seus programas. A \emph{Schneider Electric}
possui um projeto pelo qual suas executivas visitam escolas técnicas e
de ensino médio para estimular as estudantes a escolherem profissões do
mercado de tecnologia. A \emph{\versal{EY}} criou no Brasil o projeto
\emph{Winning Woman} (Mulheres Vencedoras), que busca aconselhar e
reconhecer empreendedoras brasileiras de sucesso para que possam estar
preparadas para superar seus desafios. Também conecta as empreendedoras
com outras organizações e pessoas relevantes que possam apoiar suas
empresas na jornada para o crescimento. A \emph{Embrapa} desenvolve
projetos sociais que promovem o empoderamento das mulheres que trabalham
com agricultura em comunidades rurais, como quebradeiras de coco e
catadoras de mangaba.

Algumas empresas já têm ações comunitárias destinadas a combater
diretamente a violência doméstica contra mulheres e crianças, como
\emph{Accenture, Avon, Coca"-Cola Brasil, Eletronorte, \versal{EY}, \versal{KPMG},
Nogueira, Elias, Laskowisk e Matias Advogados e Vale}. A \emph{Caixa
Econômica Federal} tem uma linha de crédito voltada a pequenas
empreendedoras, que faz parte da Política Nacional de Microcrédito
Produtivo Orientado, para fomentar o empreendedorismo feminino. O
\emph{Itaú}, por sua vez, tem uma plataforma online e presencial de
orientação para mulheres empreendedoras, o \emph{Itaú Mulher
Empreendedora}. O portal \emph{Tempo de Mulher}, o \emph{Banco
Interamericano de Desenvolvimento} (\versal{BID}) e a \emph{Unilever} investiram
na criação de uma plataforma online de orientação profissional e pessoal
para mulheres, a Escola de Você. O \emph{Instituto Consulado da Mulher},
da \emph{Whirlpool}, trabalha na transformação social por meio do
incentivo ao empreendedorismo para mulheres de baixa renda e
escolaridade que vivem em comunidades vulneráveis na periferia das
grandes cidades ou em áreas rurais de todo o Brasil.

\section{Transparência, medição e relatórios}

Por fim, quanto ao sétimo Princípio de Empoderamento das Mulheres
desenvolvido no documento da \versal{ONU}, ``Transparência, medição e
relatórios'', é citada a publicação de
indicadores de gênero no \emph{Global Reporting Initiative} (\versal{GRI}) por
empresas como \emph{Avon, Boticário, Caixa Econômica Federal, Dow,
Eletronorte, \versal{KPMG}, Petrobras} e \emph{Vale}. \emph{Dow, \versal{EY}, Itaú} e
\emph{Walmart} adotam a versão do \versal{GRI} que inclui o indicador que mensura
a proporção média de salários entre homens e mulheres (\versal{LA}13). O
\emph{\versal{BNDES}} criou uma base de dados histórica que servirá de apoio para
a análise da progressão das mulheres na carreira para fortalecer as
ações em prol da igualdade de gênero nos cargos de decisão. A
\emph{Caixa Econômica Federal} criou categorias de diversidade na
ouvidoria interna como forma de identificar toda manifestação de
discriminação de gênero, raça/cor e etnia, orientação sexual e
identidade de gênero, pessoas com deficiência, geração e religião. A
área de recursos humanos da \emph{Coca"-Cola Brasil} tem indicadores para
monitorar questões bastante específicas, como a evasão de mulheres após
a licença"-maternidade, que atualmente está próxima de zero. A
\emph{Eletronorte} publica, em seus relatórios de sustentabilidade, um
banco de dados sobre a diversidade de gênero entre a mão de obra
terceirizada para acompanhar seus compromissos públicos com mais
igualdade de gênero (p.\,19).

\looseness=-1
Em 2012 a União Europeia aprovou uma medida estabelecendo que os
conselhos de administração das empresas tenham 40\% de mulheres até
2020. No Brasil o \emph{Instituto Ethos} apoia o Projeto de Lei do
Senado (\versal{PLS}) 112/2010, da senadora Maria do Carmo Alves, o qual reserva
para as mulheres 40\% das vagas nos conselhos de administração das
empresas públicas, sociedades de economia mista e demais empresas em que
a União detenha a maioria do capital social com direito a voto.

\looseness=-1
Qual destas bandeiras, princípios e práticas enumeradas acima não
encontraria apoio nas organizações da esquerda feminista? A convergência
e/ou contradição entre coletivos de esquerda organizados em torno de
pautas identitárias e organizações empresariais e governamentais que
buscam manejar lucrativamente essas pautas é algo a se explorar. O
objetivo da \versal{ONU} é promover o desenvolvimento econômico do capitalismo,
com uma aposta na igualdade de gênero enquanto um caminho. Até aí nada
de surpreendente. É melhor um capitalismo com a igualdade entre homens e
mulheres, tanto para o desenvolvimento do capitalismo quanto para a vida
cotidiana das mulheres, negros e demais grupos que sofrem com a
discriminação. A questão é que isso evidencia que a ocupação de espaços
dentro do sistema não constrói a sociedade que queremos, em termos de
horizontalidade, justiça e igualdade substantiva. Pelo contrário,
reforça as bases políticas, econômicas, culturais e ideológicas do
sistema hierárquico de exploração e opressão a que chamamos capitalismo.
Nesse sentido, a política identitária levada a cabo por movimentos
antiopressão reforça o próprio sistema de relações sociais que cria e
recoloca as opressões, estruturalmente.


Enquanto a esquerda anticapitalista não forjar na prática respostas
satisfatórias às demandas concretas de negros, mulheres, \versal{LGBT}s e outras
``minorias'' a plataforma teórica e prática do identitarismo seguirá
norteando as lutas por caminhos nebulosos que pouco têm a contribuir com
as condições de vida das franjas negra, feminina e \versal{LGBT}s da classe
trabalhadora.

\chapter[Uma convergência de interesses entre feminismo e\\ instituições capitalistas?]{Uma convergência de interesses entre\break feminismo e instituições capitalistas?}
\hedramarkboth{Uma convergência de interesses entre feminismo\ldots{}}{}

Por conta da convergência de interesses entre práticas de segmentos da
esquerda e de segmentos empresariais e de governos, nos últimos anos
vemos uma crescente integração de mulheres e, em um ritmo mais lento, de
negros, nos cargos superiores e de chefia das empresas, bem como nos
cargos de comando das instituições estatais.

Isso nos coloca diante da divergência acerca do caráter atrelado ou
independente entre machismo, racismo e capitalismo. Noutros termos, será
possível um capitalismo sem machismo e sem racismo? Os dados levantados
até aqui nos indicam que não há uma correlação necessária entre
exploração do tempo de trabalho e racismo ou machismo, muito embora o
sistema capitalista consiga, muitas vezes, manejar lucrativamente os
elementos políticos, ideológicos e identitários de discriminação a fim
de potencializar a exploração da classe trabalhadora como um todo. Ainda
assim, dúvidas persistem.

Será mesmo que o desenvolvimento capitalista caminha para um capitalismo
colorido? Os dados levantados a seguir buscam indicar outros elementos
para pensarmos essa questão, mas antes de qualquer coisa vale lembrar
que há movimentos feministas e antirracistas e \versal{LGBT}s que defendem que
não há capitalismo sem machismo ou sem racismo e que, portanto, a luta
feminista e antirracista seria em si mesma danosa ao sistema e, por
isso, anticapitalista. Será?

Em diversos países do globo, a luta feminista, negra e de direitos às
chamadas minorias tem garantido um movimento irrefreável de reversão
deste quadro de elites masculinas e brancas, como o mostram não apenas
os dados positivos com relação aos anos anteriores, mas a própria
repercussão automática gerada por qualquer posicionamento contrário à
tendência de integração multiculturalista e identitária, as chamadas
``agendas de diversidade''.

Como exemplo do grau de legitimidade desse movimento vale lembrar o
rechaço sofrido pela \emph{Apple} por ter, em janeiro de 2016, feito
circular um documento onde recomendava a seus acionistas o veto de uma
cláusula que previa a adoção de medidas para garantir ``maior
diversidade na empresa'': mais mulheres, negros e pessoas pertencentes a
outras minorias nos cargos de administração e gestão de topo.\footnote{Cf.
  ``Apple Formally Opposes Shareholder Call For Increased Diversity''
  (\textless{}\emph{https://bit.ly/3f1QRD1}\textgreater{})
  e \versal{CASTRO} (2016).}

Em 2015 a pressão do ``\emph{Rainbow Push Coalition}'', grupo de
ativistas dos direitos civis encabeçado pelo líder negro Reverendo Jesse
Jackson, já havia garantido que a \emph{Apple} nomeasse duas novas
vice"-presidentes negras, Lisa Jackson e Denise Young Smith. A empresa
não teria diretores negros ou latinos, e 12 dos 15 principais executivos
seriam homens brancos. \emph{Google}\footnote{Cf.
  \textless{}\emph{https://bit.ly/3d3RGcH}\textgreater{}.},
\emph{Facebook}\footnote{Cf.
  \textless{}\emph{https://bit.ly/3aJNMno}\textgreater{}.},
\emph{Yahoo}, \emph{Twitter} e \emph{Intel}, além das demais empresas do
Vale do Silício, têm sido forçadas pelo Congresso estadunidense e por
líderes ativistas e consumidores a publicar relatórios onde expõem a
composição de gênero e origem étnica de seus funcionários, o que por sua
vez tem levado ao aumento da contratação e nomeação de minorias para os
cargos de chefia, visando a promoção pública de uma empresa com quadro
gestor colorido.

Em 2014, por exemplo, a \emph{Apple} mais do que duplicou o número de
mulheres, negros e hispânicos contratados. Segundo o \versal{CEO} da empresa, Tim
Cook, em 2014 a empresa contratou ``mais de 11 mil mulheres a nível
mundial, o que representa 65\% a mais do que no ano anterior. Nos
Estados Unidos, nós contratamos mais de 2.200 empregados negros, um
aumento de 50\% sobre o ano passado, e 2.700 funcionários
latino"-americanos, um aumento de 66\%''. O gestor ainda se defende de
possíveis críticas à empresa e garante que o movimento de contratação
prioritária de minorias continuará: ``Algumas pessoas vão ler esta
página e ver o nosso progresso. Outros irão reconhecer que ainda há
muito trabalho a fazer. Nós vemos ambos''.\footnote{Cf.
  \textless{}\emph{https://glo.bo/2SdIhHo}\textgreater{}.}

Em 2014 o \emph{Twitter} foi forçado a divulgar a composição de gênero e
etnia de seus empregados e a adotar iniciativas voltadas para ``um maior
equilíbrio e diversidade entre seus funcionários''. Além de patrocinar
conferências voltadas para grupos sub"-representados como ``\emph{Out for
Tech}'' e ``\emph{Grace Hopper}'' a empresa apoia a iniciativa
``\emph{Girls Who Code}'', e hospedou dois programas de imersão do
projeto em San Francisco, Nova York e Boston, assim como o ``\emph{Girl
Geek Dinners}'' e o ``\emph{sf.girls}'', que incentivam meninas do
ensino médio a seguir carreira na área de tecnologia.

Os investimentos atuais das empresas atentas à agenda da diversidade
concentram"-se nos alunos do ensino fundamental e médio, muito embora os
estágios posteriores sejam eficazes para captar mulheres e meninas. As
empresas de tecnologia, por exemplo, concentram 66\% de seu
financiamento filantrópico em programas do ensino fundamental, em
comparação com 3\% em programas de nível superior. Embora muitas
empresas invistam em esforços de recrutamento no final da faculdade,
poucas investem filantropicamente no ensino superior a fim de construir
a mediação a partir da qual acabarão por recrutar pessoal qualificado.
Isso aponta para uma oportunidade perdida pelas empresas de tecnologia,
que poderiam aprimorar, no curto prazo, a composição de gênero em todos
os níveis da carreira (pipeline).

O \emph{Twitter} apoia iniciativas destinadas às mulheres como
\emph{Technovation}, \emph{Techwomen}, \emph{\versal{LEAD} Instituto Ciência da
Computação}, \emph{PyLadies} e \emph{Black Girls \versal{CODE}}. A
``\emph{Rainbow Push Coalition}'' conseguiu mobilizar mais de 25 mil
petições para que o \emph{Twitter} liberasse seus dados de diversidade
entre os funcionários. O grupo, em parceria com a organização
\emph{ColorOfChange.org}, criticou a dificuldade de empresas de
tecnologia em contratar e manter mulheres e minorias em seus quadros de
funcionários, defendendo que ``não há déficit de talentos, há um déficit
de oportunidades''.\footnote{Cf.
  \textless{}\emph{https://bit.ly/2W24r0r}\textgreater{}.
  Ver também \versal{ADAMS} \& \versal{FERREIRA} (2009), acerca dos impactos na economia e
  governança de se ter mulheres em cargos de gestão e o artigo de \versal{ADLER}
  (2009) intitulado ``Lucro, seu nome é Mulher?''. Ver \versal{ASHCRAFT}, C. \&
  \versal{BREITZMAN}, A (2007) para informações sobre a participação feminina em
  patentes na área de informática.}

Vale lembrar que essa estrutura explicativa é a mesma usada pelo
\emph{Banco Mundial} e \emph{\versal{ONU}} ao tratar de diversas desigualdades
sociais como fruto não da própria estrutura do capitalismo, mas da
``falta de oportunidades'', em âmbito individual, a partir de onde
passam a defender o empreendedorismo como solução para o problema da
pobreza e desigualdade social.

Em janeiro de 2015 a \emph{Intel} anunciou um investimento de três anos
no valor de \versal{US}\$ 300 milhões de dólares, a serem aplicados em medidas
para ``melhorar a diversidade da força de trabalho da empresa, atraindo
mais mulheres e minorias para o setor da tecnologia e tornando a
indústria mais receptiva a eles quando forem contratados''. Segundo a
empresa o dinheiro será usado para financiar bolsas de estudo em
engenharia e para apoiar universidades de público majoritariamente
negro.

Estas medidas visam combater uma tendência negativa do setor, apontada
por pesquisas recentes:

\begin{quote}
O setor de tecnologia precisa inovar para expandir sua força de trabalho
técnica --- e rapidamente. De acordo com um recente \emph{paper} posto em
discussão pelo \emph{McKinsey Global Institute}, a demanda por
habilidades avançadas de programação e \versal{TI} aumentará em até 90\% nos
próximos 15 anos. Os líderes de negócios em todos os setores já estão
relatando uma escassez esperada de habilidades técnicas em suas empresas
nos próximos três anos. E a competição por talentos técnicos está
prestes a se tornar muito mais feroz em todos os setores, à medida que
empresas de todos os tipos aumentam suas capacidades técnicas. Para
ficar à frente, o setor de tecnologia precisa expandir rapidamente seu
banco de talentos, investindo e atraindo talentos historicamente
subutilizados, principalmente as mulheres. \footnote{Cf.
  \textless{}\emph{https://mck.co/35bvv1t}\textgreater{}
  e \textless{}\emph{https://mck.co/2SeHu93}\textgreater{}.}
\end{quote}

O relatório conclui ainda que se as empresas de tecnologia se mostrarem
capazes de criar caminhos para mulheres e meninas,
``\emph{particularmente mulheres e meninas de cor e mais
marginalizadas}'', pessoas ``que enfrentam o maior número de
barreiras'', a fim de incentivá"-las a buscar carreiras em tecnologia, a
indústria ``se beneficiará de um conjunto de talentos muito mais amplo e
realizará novas oportunidades''. A receita é dada em detalhes:

\begin{quote}
Como parte de um esforço mais amplo de diversidade, é importante que as
empresas apoiem programas somente para meninas ou programas
coeducacionais que se concentrem em alcançar pelo menos 40\% de
representação de meninas por meio de etapas proativas de recrutamento e
retenção. Manter o foco na representação igual das mulheres, com metas
declaradas no nível do programa, é a única maneira de evitar replicar as
mesmas proporções de gênero que vemos hoje na tecnologia. Ajude os que
enfrentam mais barreiras --- mulheres e meninas de cor sub"-representadas.
As mulheres experimentam diferentes tipos de barreiras e preconceitos
quando estudam computação e buscam uma carreira em tecnologia devido à
sua raça ou etnia, status socioeconômico, sexualidade e outros elementos
de suas identidades e origens. As empresas podem apoiar estratégias e
programas que atendam aos desafios específicos enfrentados pelos
subsegmentos de mulheres que enfrentam múltiplas formas de
marginalização. Concentrar"-se nas experiências daqueles que enfrentam o
maior número de barreiras irá estimular soluções que, em última análise,
melhoram a inclusividade do setor de tecnologia para todos os grupos
sub"-representados. {[}\ldots{}{]} As empresas de tecnologia devem trabalhar
com seus parceiros para garantir que esses fatores de sucesso sejam
implementados para maximizar o impacto de seus investimentos: 1. Ofereça
rampas de acesso para iniciantes. 2. Crie um sentimento de pertença. 3.
Construa a confiança {[}das meninas{]} em suas habilidades. 4. Cultive
uma comunidade de pares de apoio. 5. Garanta que os apoiadores adultos
(família, professores, conselheiros) sejam encorajadores e inclusivos.
6. Fomente o interesse em carreiras de computação. 7. Crie continuidade
entre as experiências de computação. 8. Forneça o acesso à tecnologia e
experiências de computação. (op.\,cit)
\end{quote}

Com respeito ao item 7, o relatório da pesquisa realizada pela Pivotal
Ventures (uma empresa de investimento e incubação criada por Melinda
Gates) e McKinsey se explica:

\begin{quote}
A maioria dos programas tem como alvo apenas um estágio específico da
jornada tecnológica. No entanto, se a experiência de mulheres e meninas
na tecnologia é única, elas são menos propensas a permanecer engajadas
na computação. As empresas podem encorajar os programas que apoiam para
se conectarem uns com os outros e fazer a transição suave das mulheres
jovens de uma experiência para outra --- e investir para preencher as
lacunas nas ofertas de programas. Desenvolver esse ``tecido conjuntivo''
aumenta a probabilidade de que as experiências nas quais uma empresa
investe levem as mulheres a entrar no setor. (op.\,cit)
\end{quote}

A relação entre estas políticas e a lucratividade da empresa não é
apenas indireta, no plano do apoio dos consumidores a uma empresa
``socialmente responsável'' em termos de inclusão de ``diversidades'',
mas sim diretamente econômica: em notícia de 2015 a empresa anunciou que
investirá em medidas para trazer mais mulheres para o ramo dos jogos
eletrônicos. A diretora executiva da empresa, Kate Edwards, afirmou que
a \emph{Intel} planeja criar e sustentar uma equipe feminina de
\emph{gamers} profissionais, já tendo estabelecido uma parceria com a
Associação Internacional dos Desenvolvedores de Jogos, entidade sem fins
lucrativos que vai enviar 20 universitárias a uma conferência de
desenvolvedores de jogos com o apoio da \emph{Intel}.\footnote{Cf.
  \textless{}\emph{https://bit.ly/35c7pDF}\textgreater{}.}
A iniciativa da \emph{Intel} de criação de bolsas de estudo para
universidades e faculdades tradicionalmente negras foi adotada pela
\emph{Google} e \emph{Apple}.\footnote{Cf.
  \textless{}\emph{https://bit.ly/3eVYiLJ}\textgreater{}.}

Renee Richardson Gosline, da \emph{Mit Sloan School of Management}
observa que ``Uma geração mais jovem de consumidores está buscando
produtos que estejam alinhados com suas causas''. Patti Williams, da
Universidade da Pensilvânia, pontua que os consumidores de hoje em dia
``esperam que as marcas compartilhem seus valores e não apenas
representem o melhor valor e utilidade''. Da mesma forma, as empresas
mais atentas aos ventos do nosso tempo querem recrutar trabalhadores da
mesma geração de seus consumidores, o que também significa apelar para
seus valores. ``Os jovens não querem trabalhar para uma empresa se isso
for considerado prejudicial ao meio ambiente ou à sociedade'', diz
Jaideep Prabhu, da Judge Business School, da Universidade de Cambridge.
``Eles querem ter orgulho de dizer onde trabalham''.

Já no passado algumas grandes empresas demonstraram uma ``consciência
social''. Em 1969, por exemplo, em meio a uma época de altas tensões
raciais, a Coca"-Cola publicou um anúncio chamado ``Boys on a Bench'' em
que jovens negros e brancos estavam sentados juntos, degustando a bebida
gasosa.\footnote{Cf.
  \textless{}\emph{https://bit.ly/3eVFHQ8}\textgreater{}.
  Há ainda o esforço anterior de propaganda, pós"-Segunda Guerra, voltado
  para o mercado consumidor afro"-americano, encabeçado por Moss Kendrix.
  Ver: ``Consuming America: Moss Kendrix, Coca"-Cola and the Identity of
  the Black American Consumer''
  e sobre a primeira mulher negra a aparecer em uma propaganda da
  Coca"-Cola, ver:
  \textless{}\emph{https://bit.ly/3cSTmW4}\textgreater{}.}

Segundo o colunista Bartleby, da \emph{The Economist}, em seus
primórdios os anúncios eram um pouco mais sutis do que os exemplos
modernos, como por exemplo o da empresa de lâminas de barbear
\emph{Gillete}, que retrata criticamente a ``masculinidade tóxica'', mas
``as empresas estão mais uma vez sendo empurradas para o fórum político
porque as `guerras culturais' dos \versal{EUA} cobrem muitos problemas que afetam
o local de trabalho''.\footnote{Cf. \textless{}\emph{https://econ.st/2Sc9HgB}\textgreater{}.}

Larry Fink, gestor da empresa administradora de ativos \emph{BlackRock,}
escreveu em sua Carta anual a executivos"-chefes que ``a sociedade está
cada vez mais procurando empresas, públicas e privadas, para lidar com
questões sociais e econômicas prementes. Essas questões vão desde
proteger o meio ambiente até a aposentadoria, passando pela desigualdade
de gênero e racial''.

Iris Bohnet, professora de políticas públicas na \emph{John F. Kennedy
School of Government} da \emph{Harvard University}, lembra algumas das
vantagens organizacionais inerentes a uma maior diversidade no corpo
gestorial de uma empresa:

\begin{quote}
A evidência é muito forte de que equipes diversas superam equipes
homogêneas, sejam elas masculinas ou equipes exclusivamente femininas.
Isso ocorre em todos os tipos de variáveis dependentes diferentes, desde
a solução criativa de problemas até as tarefas analíticas e as
habilidades de comunicação. A diversidade ajuda porque temos uma
complementaridade de diferentes perspectivas, ou o que chamamos de
``inteligência coletiva''.\footnote{Cf. \textless{}\emph{https://mck.co/2YcVnIC}\textgreater{}.}
\end{quote}

Tal como o \emph{Instituto Ethos} e as maiores transnacionais, em
especial as do setor de tecnologia, também empresas brasileiras estão
atentas às oportunidades de negócios no plano da integração da
diversidade. Em um site voltado para o treinamento empresarial e à
educação corporativa podemos ler:

\begin{quote}
a participação de negros no primeiro escalão das empresas nem sequer
superou um ponto percentual. Mesmo no quadro funcional, no pé da
pirâmide hierárquica, a igualdade de raças ainda está em 1,3\%.
Aparentemente, o empresariado brasileiro continua refém do ``olhar
acostumado'', em vez de avançar rumo a uma postura mais agressiva. É
importante entender que não se trata apenas de um problema social. O
perfil predominante de brancos entre os colaboradores bate de frente com
a demanda por diversidade cultural, que veio junto com a globalização.
As empresas deveriam entender a diversidade cultural como uma oxigenação
do ambiente organizacional e como oportunidade de negócios. O Brasil, um
país abençoado com tanta diversidade cultural, simplesmente deixa de
ganhar com mais dinamismo, mais debate, mais pluralidade e mais
interação. Não faz sentido deixar espaço para os concorrentes terem
políticas multiculturais, enquanto empresas brasileiras seguem
monoculturais. {[}\ldots{}{]} O negro está consciente de que pode continuar
contribuindo para criar riquezas financeiras --- o branco não pode, pelo
menos, entrar num jogo ganha x ganha mais equilibrado com o negro?
\footnote{Cf. \textless{}\emph{https://bit.ly/3cK9SYy}\textgreater{}.}
\end{quote}

Hank Williams, empresário nova"-iorquino fundador e presidente da empresa
de serviços na nuvem \emph{Kloudco} esclarece que trazer mais
diversidade ao Vale do Silício, o principal polo tecnológico dos Estados
Unidos, não é apenas uma questão moral ou econômica: ``O problema é que
nossa economia é baseada em inovação'', ``Se todo nosso crescimento vier
de setores que excluem vários grupos demográficos, isso pode gerar uma
crise''. Esta importante fala conflui com diversas pesquisas citadas até
aqui.

A pressão por maior diversidade e inclusão nos postos de trabalho do
ramo tecnológico estadunidense levou à elaboração, pelo \emph{Wall
Street Journal}, de um infográfico onde os consumidores e ativistas
podem checar o grau de diversidade de cada empresa.\footnote{Cf.
  \textless{}\emph{https://bit.ly/2yMXiJq}\textgreater{}.}

Atualmente a lucratividade das empresas está totalmente articulada ao
\emph{politicamente correto}. Uma campanha difamatória de uma empresa
sem responsabilidade social, ambiental ou cultural pode levar uma
empresa a perdas significativas e, inclusive, à falência, assim como o
engajamento próximo às demandas dos movimentos sociais pode garantir uma
maior lucratividade. A pressão dos movimentos identitários já levou à
criação de uma série de organizações, empresas especializadas e
iniciativas, tanto corporativas quanto civis e não lucrativas, voltadas
para a redução da disparidade étnica e de gênero. Só nos \versal{EUA} e na área
tecnológica podemos citar como exemplo as iniciativas
\emph{\#YesWeCode,} o \emph{Code 2040,} a \emph{Girls Who Code,} a
\emph{All Star Code,} o \emph{Digital Undivided, a \versal{BUILDUP},} a
\emph{Black Girls Code,} a \emph{Silicon Harlem's Apps Youth Leadership
Academy,} a \emph{Latino Startup Alliance,} a \emph{Black Founders,} o
\emph{Culture Shift Labs.}\footnote{Cf.
  \textless{}\emph{https://bit.ly/2yKiQ9B}\textgreater{}.}

Bartleby, colunista da \emph{The Economist} citado acima, comentando a
inclusão de mulheres nos conselhos gestores conclui:

\begin{quote}
há um milhão de razões pelas quais as empresas podem ou não ter sucesso,
do clima econômico à tecnologia, e a composição do conselho
provavelmente é apenas uma pequena influência. Mas a nomeação de mais
mulheres poderia enviar um sinal para as funcionárias e clientes de que
a empresa é um empregador inclusivo. E há uma chance de que diretores do
sexo feminino ajudem as empresas a evitar o tipo de desastre de
marketing, como as canetas da Bic para mulheres.\footnote{Cf.
  \textless{}\emph{https://econ.st/3eVsVRo}\textgreater{}.
  O caso da fracassada ``homenagem'' da empresa de canetas Bic, com a
  linha ``Bic Para Elas'', pode ser conferido aqui:
  \textless{}\emph{https://bit.ly/2KFckU4}\textgreater{}.}
\end{quote}

A despeito das iniciativas estatais e do interessado apoio empresarial a
diversos elementos da ``agenda da diversidade'' a situação de
disparidade de salários é ainda particularmente notável quando se
observa o caso das mulheres negras, que sofrem com discriminações de
gênero e de cor, tendo no Brasil uma renda média 41,5\% menor do que
ganham as mulheres brancas e recebendo um salário 25,5\% menor que o de
homens negros. Milko Matijascic, pesquisador do \emph{Instituto de
Pesquisas Econômicas Aplicadas} (\versal{IPEA}), conclui a partir destes dados
que ``É preciso trabalhar políticas de gênero e raça em paralelo. Não
adianta pensar que se melhorar a educação de um modo geral, tudo vai se
equiparar. As mulheres já têm mais anos de educação que homens e
continuam ganhando menos''. Ora, se a qualificação é maior e a
remuneração é menor, estamos diante de uma potencialização da exploração
por meio de mecanismos de mais"-valia absoluta, ou seja, utiliza"-se a
discriminação de gênero e de cor para sub"-remunerar o trabalhador
qualificado.

Como a Lei das Cotas só entrou em vigor em 2012 o \versal{IPEA} aponta que a
política de cotas é ainda ``muito recente e pontual para ter reflexos
relevantes nos indicadores macroeconômicos''. Frei David Santos,
especialista em ações afirmativas e fundador da \emph{\versal{ONG} Educafro},
afirmou em entrevista que não há dúvidas de que esta política provoca
reflexos positivos na ``classe negra'': ``A comunidade negra está se
empoderando, especialmente por meio do acesso à universidade'', afirma.
Também em entrevista, Nelson Pires, estudante cotista da \versal{UERJ}, conta que
frequentou cursinhos comunitários e conseguiu uma vaga no curso de
Engenharia Civil, incentivado pela irmã, que já havia conseguido vaga no
curso de Direito. O futuro engenheiro conta que seus pais sequer os
incentivavam a prosseguir com os estudos após o Ensino Médio: ``Meu pai
falava para minha irmã que ela teria um diploma, mas não teria um
emprego. Hoje, ele mudou de postura, nos ajuda sempre que precisamos e
incentiva meus outros dois irmãos a estudar''. Convicto de que há um
``cenário promissor para o futuro'', Nelson conta que a irmã acaba de
passar no exame da Ordem dos Advogados do Brasil: ``A gente quebrou um
ciclo. Também vou exigir que meus filhos estudem também''. Depois da
implantação da Lei de Cotas, outras políticas afirmativas foram
implantadas no Brasil. Em 2014 foi aprovada uma lei que reserva vagas
para negros em concursos públicos, e em 2015 foi sancionada no Rio de
Janeiro uma lei que garante cotas em cursos de pós"-graduação.\footnote{Cf.
  ``Renda de negros cresce 56,3\%''
  \textless{}\emph{https://bit.ly/2SfaEFb}\textgreater{}
  e ``Negros quase triplicam no ensino superior no Brasil em 10 anos'',
  Agencia Brasil, 2015.}

Era uma prática comum, até pouco tempo atrás, a justificativa
empresarial de que o baixo recrutamento de mulheres para empregos e
promoções para cargos de chefia ocorria em virtude da falta de
candidatas mulheres qualificadas. De acordo com a pesquisa da Mckinsey
\& Co, \emph{Women in the workplace 2017}, ``essa desculpa não cola
mais'', pois as pessoas estariam compreendendo que os preconceitos podem
minar o sucesso das mulheres:

\begin{quote}
O viés de avaliação de desempenho, por exemplo, significa que os homens
tendem a ser mais avaliados quanto a seu potencial e as mulheres quanto
a suas realizações até o momento. As mulheres também tendem a receber
menos crédito do que os homens pelo sucesso e mais críticas pelo
fracasso. Além disso, as mulheres são menos assertivas que os homens e
subestimam suas próprias contribuições. O viés materno desencadeia
suposições de que as mães têm menos comprometimento com suas carreiras,
portanto, elas são avaliadas segundo padrões mais elevados e recebem
menos oportunidades de liderança.
\end{quote}

O tema das cotas em cargos de comando surge quando o relatório tematiza
as ações que as empresas adotam, incluindo a McKinsey, para
promover a igualdade no trabalho:

\begin{quote}
Estes incluem a oferta de cursos de formação sobre o preconceito; tomar
medidas para garantir que os processos de recrutamento, desempenho e
promoção sejam justos; trabalhar para ajudar os funcionários a
equilibrar o trabalho e a vida em casa, como oferecer licença parental
prolongada, condições de trabalho flexíveis e apoio à assistência
infantil; e olhar com atenção os dados para entender onde as mulheres
ficam presas, na estrutura da empresa. Todas essas ações são importantes
para promover o tipo de cultura inclusiva em que as empresas prosperam,
embora as prioridades possam diferir em diferentes geografias,
dependendo do contexto sociocultural. {[}\ldots{}{]} Os resultados dos oito
países europeus que impõem quotas femininas a conselhos de administração
são instrutivos. Atualmente, a representação dos membros da diretoria
feminina varia entre 33 e 40\%, em comparação com uma média de 17\% nos
países do G"-20. Alguns observadores, no entanto, temem que as cotas
promovam o tokenismo e, portanto, não construam a capacidade de
liderança feminina. Outros passaram a ver as cotas como etapas
transitórias desconfortáveis, mas necessárias. Barbara Dalibard, \versal{CEO} da
empresa de tecnologia \versal{SITA}, nos disse que o progresso era inadequado:
``Em alguns ambientes técnicos, as mulheres ainda enfrentam as mesmas
dificuldades de 25 anos atrás. Quando eu era jovem, eu era totalmente
contra as cotas; minha crença agora é que, se você não tem cotas, as
coisas não mudam. A mudança na França está acontecendo em conselhos por
causa da lei. Isso não está acontecendo nos comitês executivos, porque
as cotas não se aplicam lá''. Na ausência de cotas, o progresso repousa
na medição da diversidade, divulgando os progressos realizados e
responsabilizando as pessoas. As empresas com os melhores registros para
representação feminina compartilham suas métricas com todos os
funcionários, mas essa transparência é rara. Embora nosso estudo mais
recente sobre mulheres no local de trabalho mostre que 85\% das empresas
acompanham a representação feminina em cada nível, menos da metade
afirma que responsabilizam gerentes seniores por melhorar as métricas de
gênero e menos ainda são ousados ​​o suficiente para definir metas de
qualquer descrição.\footnote{Cf.
  \textless{}\emph{https://cutt.ly/Pysu9H4}\textgreater{}.}
\end{quote}

De acordo com estudos do Banco Mundial as barreiras à inclusão econômica
das mulheres vêm caindo nos últimos 50 anos em todo o mundo. Atualmente
as leis que restringem a atividade econômica das mulheres são
prevalecentes no Oriente Médio e Norte da África, África Subsaariana e
Sul da Ásia, ou seja, em regiões de pouco desenvolvimento econômico e,
consequentemente, de predomínio dos mecanismos menos refinados de
mais"-valia absoluta. Vale pontuar que, aqui, o insultado
``eurocentrismo'' não pode ser invocado a fim de explicar a situação de
intensa discriminação de gênero.

Na maioria dos países os salários das mulheres representam entre 70 e
90\% do recebido pelos homens, com taxas ainda mais baixas em alguns
países asiáticos e latino"-americanos; segundo dados de 2011 sobre a
economia global, coletados pela \versal{ONU} Mulheres, 50,5\% das mulheres que
trabalham estão em empregos vulneráveis, muitas vezes sem proteção da
legislação trabalhista, em comparação com 48,2\% para os homens. As
mulheres estão mais propensas do que os homens a estar em empregos
vulneráveis ​​na África do Norte (55\% \emph{versus} 32\%), Oriente
Médio (42\% \emph{versus} 27\%) e África Subsaariana (quase 85\%
\emph{versus} 70\%).\footnote{Ver o relatório da \versal{ONU} Mulheres ``Economic
  empowerment of women'', de 2012.}

Um relatório do \emph{Banco Mundial} indica que, entre 2012 e 2013, 44
economias fizeram alterações jurídicas visando aumentar as oportunidades
das mulheres. Em Costa do Marfim e Mali, por exemplo, os maridos não
podem mais impedir as mulheres de trabalhar, enquanto as Filipinas
retiraram as restrições ao trabalho noturno e a República Eslovaca
aumentou o percentual dos salários pagos durante a licença"-maternidade.
As economias do Leste Europeu e Ásia Central são as que têm mais
profissões que não podem ser executadas por mulheres, nalguns casos
devido a uma costumeira ``vontade de proteção das mulheres'', o que nem
por isso deixa de significar menos postos de trabalho disponíveis para
as mulheres, portanto, no limite, uma maior dificuldade de independência
econômica face aos homens. Em 2012, na Federação Russa, por exemplo, as
mulheres não podiam dirigir caminhões no setor agrícola, na Bielorrússia
não podiam ser carpinteiras e no Cazaquistão não podiam ser
soldadoras.\footnote{Cf.
  \textless{}\emph{https://cutt.ly/Eysu3OM}\textgreater{}.}
Uma notícia de 2018 nos informa a Rússia proíbe a atuação de mulheres em
456 funções consideradas muito difíceis, perigosas ou arriscadas, mas
que estas restrições nem sempre funcionam na vida real, pois os
empregadores contratam mulheres ao garantir condições seguras de
trabalho e, nalguns casos, são as próprias candidatas às vagas que fazem
esforços extraordinários para conseguir o emprego.\footnote{Cf. 3 russas
  que desafiaram as proibições de postos de trabalho para mulheres.
  \textless{}\emph{https://cutt.ly/Vysu8zP}\textgreater{}.}

Esses tipos de restrição são prejudiciais ao desenvolvimento
capitalista, a começar pelo fato de que quando estas barreiras caírem
haverá naturalmente uma ampliação da oferta de força de trabalho e, por
conseguinte, uma redução dos níveis de salário. Quando os gestores não
percebem por si mesmos estas vantagens relativas, as lutas sociais os
pressionam a perceber e levam o capitalismo a se dinamizar.

O relatório da \versal{ONU} Mulheres aponta que de 1960 a 2010 foram retiradas
mais da metade das restrições aos direitos de propriedade das mulheres e
sua capacidade para realizar transações nas 100 economias estudadas.
Também foram notados avanços quanto à legislação a respeito das formas
de violência contra as mulheres, como por exemplo, o assédio sexual e a
violência doméstica. Atualmente, em 78\% daquelas economias há ampla
difusão das proibições contra o assédio sexual no local de
trabalho.\footnote{Cf. \textless{}\emph{https://cutt.ly/sysu4g9}\textgreater{}.}

\chapter{A pressão modernizante das lutas identitárias}

As políticas de cotas e demais bandeiras inclusivas da política
identitária minimizam a eficácia dos métodos de exploração da mais"-valia
absoluta, forçando os capitalistas a se modernizarem, a desenvolverem as
forças produtivas e, assim, a recolocarem a exploração prioritariamente
sobre as bases da mais"-valia relativa, capacitando as empresas a atender
às demandas identitárias sem que isso represente perdas na
lucratividade, pelo contrário.

Além disso, ao conquistar a proteção legal do poder público, os agentes
mais ativos das organizações pautadas em políticas identitárias e
multiculturalistas adquirem o direito de se relacionarem com o capital
sob novas \emph{condições} de exploração, tanto de baixo para cima
quanto de cima para baixo.

A política de cotas destina"-se a inserir negros e mulheres na elite,
portanto é uma política de promoção de novas elites, como o demonstra a
experiência, que já vem de décadas, da política de cotas nos Estados
Unidos. Aqui podemos atrelar o interesse material à ideia, hegemônica
dentre os identitários, de que os negros e mulheres são vítimas
seculares, e que por isso constituem ``minorias'' (quando numericamente
são maiorias) aptas a serem ressarcidas com retroativos.

A pretensão dos identitários de se integrarem às elites se legitima
ideologicamente por esta via, carregando consigo outro elemento bastante
problemático: o \emph{ressentimento.} Este se constitui enquanto
ressentimento face ao racismo e ao machismo, portanto entendido não
enquanto desejo de superar o capitalismo e sim como aspiração de
ascensão social dentro do sistema. O objetivo central dos identitarismos
é sempre o estabelecimento de ações afirmativas, em especial cotas (ou a
versão militante das cotas, os ``espaços exclusivos'' e de
``protagonismo'' de sujeitos). Vale lembrar que não há pressão por cotas
em cargos ou posições rebaixadas e mal remuneradas da escala econômica e
política, por exemplo, para os cargos de gari, servente de pedreiro etc.

A política identitária pode ser identificada com o que João Bernardo
chamou, em \emph{Labirintos do fascismo}, de uma modalidade de fascismo
pós"-fascista, de caráter racista. Se observarmos de perto a política
identitária veremos uma diluição do conceito de \emph{exploração} em uma
noção ambígua e difusa de \emph{poder,} que abarca a tudo. Além disso,
há a apologia da \emph{vontade} em detrimento do determinismo (que é
negado), o primado atribuído à política por sobre a economia e, por fim,
a conversão de uma nação ou micronações e etnias em postulados
ideológicos (2018: 1364). O identitarismo, ainda,\,relaciona"-se
com o cultural e o natural/biológico de modo oportunista, num hábil
pêndulo tático que ora ressalta os aspectos históricos que conformam as
opressões de raça e gênero, ora ressalta supostos elementos naturais
delimitadores da posição do indivíduo como oprimido ou opressor. Esse
pêndulo aparece de modo muito nítido quando as políticas identitárias
são levadas ao extremo e vemos escrachos pautados em pré"-julgamentos
baseados apenas na visão da suposta vítima, quando as feministas
identitárias buscam silenciar ou sobreporem"-se, politicamente, às
mulheres transgênero, pelo fato de terem pênis e nascido ``dentro dos
privilégios'' de ser homem, quando seus adeptos afirmam os homens como
``potencialmente estupradores'', quando recriminam uma pessoa negra que
se relaciona com pessoas brancas (os negros ``palmiteiros''), quando
estabelecem hierarquias políticas e de lugar de fala baseadas em níveis
de melanina na pele, etc.

O identitarismo também tem uma relação ambígua com os mecanismos de
mais"-valia, pois enquanto \emph{movimentos sociais} identitários ---
movimento feminista, movimento negro, movimento \versal{LGBTIQ}+ --- eles
relacionam"-se mais diretamente com a mais"-valia absoluta e seus
mecanismos de reserva de mercado de trabalho, postos de gerência etc. Já
enquanto \emph{instrumentos de pressão} sobre os grandes capitalistas
estes movimentos podem ser considerados como parte da engrenagem
propiciadora da mais"-valia relativa, ou seja, do desenvolvimento
econômico. Dada a importância gigantesca das pautas da ``agenda da
diversidade'' (em especial o combate ao machismo, homofobia e racismo)
podemos notar o quão trágica é a situação de se perceber semelhanças
entre aspectos destes movimentos de esquerda identitária e o que de pior
surgiu no campo ideológico"-político da extrema direita. O importante
para este tópico, contudo, é observar que as pautas identitárias,
justamente por conta da legitimidade inquestionável e da necessidade
urgente de respostas às opressões de gênero e raça, leva a uma resposta
modernizante por parte do Estado e das empresas. Por operarem tal
resposta, as empresas se reforçam política e economicamente, o que
renova as elites e faz assentar em bases mais sólidas o domínio
capitalista, pois a hegemonia se torna estável conforme se estreitam os
laços políticos, ideológicos e econômicos entre patrões e trabalhadores.
O antagonismo estrutural entre capital e trabalho dá lugar a um consenso
tácito de que um \emph{capitalismo colorido} garantidor de
``oportunidades iguais'' (asseguradas por meio de políticas públicas de
fomento à diversidade) é bom para todos, cabendo a cada indivíduo
inovar, ir à luta, empreender e conquistar seu lugar ao Sol (cargos de
comando, posições de destaque e \emph{status} etc).

Outro elemento importante para observarmos algumas das armadilhas da
política identitária diz respeito ao fato de que a promoção de uma
identidade automaticamente promove a identidade negada, num processo de
reação. Não por acaso o principal alicerce ideológico do presidente
Trump, nos \versal{EUA}, é a noção de uma identidade branca. Lá, o identitarismo
de direita é mais poderoso que os de esquerda, tendo força também o
identitarismo cristão, especialmente evangélico, e também o
identitarismo heterossexual. O mesmo fenômeno ocorre no Brasil, o que
nos ajuda a entender a vitória eleitoral de Bolsonaro e o delineamento,
cada vez mais nítido, de um confronto de identitarismos. Por serem
igualmente identitários ambos os lados deste confronto são fascistas,
portanto trata"-se de um embate entre vertentes do fascismo pós"-fascista.
O fato de termos preferência por um lado ou outro não muda o caráter
fascista do grupo ou da política: uma ``epistemologia do sul'' é tão
fascista quanto a ``física ariana''.

Com a facilitação do acesso aos cargos de gestão as conquistas das lutas
identitárias resultam na renovação das classes capitalistas e na
progressiva eliminação das formas de discriminação enquanto fator de
incremento da mais"-valia absoluta. Ao mesmo tempo, esse processo garante
a legitimidade do próprio sistema capitalista enquanto sistema em que
todos, independentemente de raça, gênero ou o que for, têm oportunidade
de ascensão social pela via do trabalho e direitos de cidadania:
enquanto, na reportagem mencionada acima, o pai do futuro engenheiro
cotista não acreditava na garantia de emprego da filha sequer depois
dela obter o diploma, os filhos cotistas veem dentro do capitalismo um
``futuro promissor'', cientes de que ``a gente quebrou um ciclo''.

Assim, as lutas identitárias tensionam para que o capitalismo
homogeneíze as formas de exploração independentemente de critérios de
raça, gênero, etnia e sexualidade, e por isso são um dos meios pelos
quais o capitalismo se autolegitima e se dinamiza, fortalecendo a
capilaridade do poder de modo a encontrar novo ânimo de desenvolvimento
econômico. O identitarismo exerce uma pressão modernizante.

Não é por acaso que o pastor ativista do movimento negro estadunidense,
Jesse Jackson, localiza justamente no setor de maior dinamicidade em
pesquisa e desenvolvimento tecnológico o alvo prioritário das lutas
negras: ``Diversificar esse setor é o próximo passo do movimento de
direitos civis'', disse em conferência no \versal{MIT}, o \emph{Instituto
Tecnológico de Massachusetts}. No mesmo evento, o empresário Hank
Williams demonstra sua capacidade de visão estratégica ao dizer sem
rodeios que um argumento fundamental a favor da diversificação de gênero
e etnia dos trabalhadores das empresas é o risco de estagnação econômica
e de protestos populares: ``É impossível excluir uma parte significativa
da população do setor da economia que mais cresce no país e não
compreender que, mais cedo ou mais tarde, isso resultará em protestos
populares --- ou coisa pior'', arrematando que ``não dá para tirar todas
as oportunidades econômicas de uma população e esperar que ela não se
revolte''.\footnote{Cf. \textless{}\emph{https://cutt.ly/eysu7em}\textgreater{}.}

\chapter{Periferia empoderada}

\section{Geopolítica}

Vimos no começo deste livro que as empresas têm inovado nas formas de
recrutamento dos gestores e do corpo ``tecnocrático'' que administra as
empresas. Essas formas de recrutamento envolvem a assimilação
estratégica de algumas das bandeiras identitárias por integração e
melhores condições de vida. Atualmente, uma questão importante a
ressaltar quanto às empresas transnacionais diz respeito ao fato de que
essas empresas criam centros de \emph{Pesquisa e Desenvolvimento} junto
às filiais da companhia, o que repõe em outro patamar a questão do
colonialismo, das antigas formas de imperialismo e da ``periferia do
sistema''. Se a princípio esse processo ocorria tão somente quando havia
necessidade de adaptação local de tecnologias vindas da sede, com o
aprofundamento da transnacionalização ele passa a ser um elemento comum,
que não apenas amplifica o espectro daquele processo, mas o muda em
termos qualitativos, uma vez que agora as empresas filiais, localizadas
nas periferias do sistema, criam centros de pesquisa e desenvolvimento
que não se limitam a adaptar uma tecnologia desenvolvida na matriz e
passam a consolidar plataformas de investigação científica voltadas para
a \emph{inovação,} no sentido schumpeteriano da ``destruição
criativa''.\footnote{Para estudos de caso com respeito à relação entre
  multiculturalismo e inovação ver, por exemplo, \versal{BOSETTI}, V. et. all ,
  2012. e \versal{PARROTTA}, et~al., 2012.}

Se notarmos o impacto do ramo de \emph{Pesquisa e Desenvolvimento} na
geração de inovações que resultem em mercadorias com valor agregado
perceberemos o quanto esse aspecto da transnacionalização do capital é
importante e o quanto traz implicações no âmbito das antigas teorias
anti"-imperialistas. Não é à toa que as empresas transnacionais de sede
brasileira se articularam com o Estado e, durante as incursões do
imperialismo brasileiro em África, certificaram"-se de empolgar a ideia
de criação da \versal{CAPES}"-África, o que visa aprofundar o enraizamento
cultural dos governos e empresas interessadas em investir na região
(portanto interessadas em que seja desenvolvido conhecimento técnico e
científico a ser convertido em elementos de desenvolvimento das forças
produtivas).\footnote{Cf. Presidente da Capes recebe homenagem e anuncia
  Capes África.
  \textless{}\emph{https://cutt.ly/Sysu7XZ}\textgreater{}.
  De acordo com o presidente da Coordenação de Aperfeiçoamento de
  Pessoal de Nível Superior, (Capes) Jorge Guimarães, a Capes África
  atenderá 19 centros de pesquisa que serão montados pelo Banco Mundial
  em dez países africanos, entre eles Nigéria, Quênia e Moçambique.
  ``Será um `clone' da nossa Capes ou algo semelhante com o mesmo
  modelo'', declarou em cerimônia de lançamento do projeto.} À época,
uma vertente do movimento negro brasileiro considerou a iniciativa uma
bela forma do Brasil reatar suas ``origens'' africanas. Cabe notar,
ainda, que se trata de programas institucionais, que persistem --- ou
tentam persistir --- mesmo após o desmantelamento dos governos petistas e
da forma como o Estado se articulava com as empresas imperialistas
brasileiras.\footnote{Ver, por exemplo, o ``Programa de cooperação
  estratégica com o sul global \versal{COOPBRASS}''
  \textless{}\emph{https://www.capes.gov.br/cooperacao-internacional/multinacional/pve/programa-de-cooperacao-brasil-sul-sul-coopbrass}\textgreater{};
  ``\versal{CAPES} e Moçambique estabelecem diálogo sobre fomento educacional''
  \textless{}\emph{https://www.capes.gov.br/sala-de-imprensa/noticias/8913-capes-e-mocambique-estabelecem-dialogo-sobre-fomento-educacional}\textgreater{}.}

Em julho de 2009 a \emph{The Economist} informava que a \emph{Petrobras}
``está entre as petrolíferas com o crescimento mais acelerado do planeta
e caminha para se tornar um polo mundial do setor'', sendo que àquela
altura extraía gás natural em Angola, Argentina, Bolívia, Colômbia,
Equador, Peru, Venezuela e \versal{EUA}. A \emph{The Economist} de 24 de Maio de
2014 informava que a empresa \emph{Vale} estava construindo uma ferrovia
que ia de Moçambique até o Malawi. Em 05 de fevereiro do mesmo ano a
presidente Dilma Rousseff inaugurou em Cuba o Porto de Mariel, um dos
mais modernos do Caribe, construído pela \emph{Odebrecht} com o aporte
de R\$ 682 milhões do \versal{BNDES} (Banco Nacional de Desenvolvimento Econômico
e Social). Mesmo frente a notícias como essas, que se tornaram
frequentes na última década, a esquerda brasileira continua a
identificar imperialismo com Estados Unidos, o que lhe permite ignorar a
expansão imperialista do capitalismo brasileiro na América Latina e
África.

Segundo dados de 2018 as empresas brasileiras estão presentes em 82
países, sendo que a grande maioria delas atua somente por meio de
unidades próprias, em alguns casos atuam por meio de unidades próprias e
franquias, e em poucos casos atuam apenas por meio de franquias. Como
esperado, a América do Norte e América do Sul são as regiões com maior
concentração de empresas transnacionais de sede brasileira, seguido de
Europa, Ásia, África, América Central, Oriente Médio e Oceania.

\pagebreak

\begin{center}
\versal{PAÍSES COM MAIOR PRESENÇA DE EMPRESAS BRASILEIRAS}
\end{center}

\begin{center}
%\resizebox{6.7cm}{!}{
\begin{tabular}{|c|c|c|}
\hline
\textbf{Posição} & \textbf{País} & \textbf{Núm. de empresas} \\ \hline
\textbf{1} & Estados Unidos & 45 \\ \hline
\textbf{2} & Argentina & 29 \\ \hline
\textbf{3} & Chile & 28 \\ \hline
\textbf{4} & México & 26 \\ \hline
\textbf{5} & Colômbia & 23 \\ \hline
\textbf{6} & Paraguai & 19 \\ \hline
\textbf{6} & Uruguai & 19 \\ \hline
\textbf{6} & Peru & 19 \\ \hline
\textbf{7} & China & 18 \\ \hline
\textbf{8} & Reino Unido & 17 \\ \hline
\textbf{9} & Alemanha & 15 \\ \hline
\textbf{10} & Portugal & 12 \\ \hline
\textbf{11} & Índia & 11 \\ \hline
\textbf{11} & Bolívia & 11 \\ \hline
\textbf{11} & Emirados Árabes & 11 \\ \hline
\end{tabular}
%}
\end{center}

\begin{center}
{\scriptsize{Fonte: \versal{FDC} 2018}}
\end{center}

\medskip

Já com respeito ao continente africano vale pontuar que de acordo com o
Ranking \versal{FDC} de 2014, 12 empresas brasileiras presentes na África atuavam
em Angola, que era o único país do continente a ter franquias
brasileiras. A dispersão geográfica das multinacionais brasileiras é a
seguinte: \emph{América do Norte}: Canadá, Estados Unidos e México.
\emph{América Central e Caribe}: Antígua e Barbuda, Bahamas, Barbados,
Bermudas, Costa Rica, Cuba, El Salvador, Guatemala, Honduras, Ilhas
Cayman, Nicarágua, Panamá, Porto Rico, República Dominicana e Trinidad e
Tobago. \emph{América do Sul}: Argentina, Bolívia, Chile, Colômbia,
Equador, Paraguai, Peru, Uruguai, Venezuela. \emph{África}: África do
Sul, Angola, Argélia, Benin, Cabo Verde, Camarões, Congo, Egito, Gabão,
Gana, Guiné, Guiné Equatorial, Líbia, Malaui, Marrocos, Moçambique,
Namíbia, Nigéria, Quênia, Tanzânia, Tunísia, Zâmbia. \emph{Europa}:
Alemanha, Áustria, Bélgica, Dinamarca, Espanha, França, Holanda,
Hungria, Irlanda, Itália, Luxemburgo, Polônia, Portugal, Reino Unido,
Romênia, Suécia, Suíça, Turquia. \emph{Ásia}: China, Coréia do Sul,
Filipinas, Hong Kong (China), Índia, Indonésia, Japão, Malásia, Papua
Nova Guiné, Rússia, Singapura, Tailândia, Taiwan. \emph{Oriente
Médio}: Arábia Saudita, Catar, Emirados Árabes Unidos, Irã, Israel,
Kuwait, Líbano, Omã. \emph{Oceania}: Austrália, Nova Caledônia.
Empresas transnacionais com sede brasileira estão presentes, portanto,
em todos os centros dinâmicos de produção capitalista do planeta, o que
nos permite repensar a visão segundo a qual ``o Brasil'' ocupa uma
posição subalterna na teia de relações imperialistas. Essa teia só pode
ser adequadamente compreendida se o foco da análise residir na dinâmica
expansionista das empresas, ancoradas em Estados, e não na lógica da
disputa entre nações.

Em 2014 a \versal{FAPESP} (Fundação de Amparo à Pesquisa de São Paulo) informou
que pesquisadores da Universidade Estadual de Campinas (Unicamp -- \versal{SP})
estavam desenvolvendo, no âmbito do Programa \versal{FAPESP} de Pesquisa em
Bioenergia (\versal{BIOEN}), o projeto \versal{GSB}"-Lacaf"-Cana"-I (\emph{Bioenergy
Contribution of Latin America \& Caribbean and Africa to the Global
Sustainable Bioenergy Project}). Trata"-se de uma parceria entre a \versal{FAPESP}
e o Nepad (Nova Parceria para o Desenvolvimento da África). O projeto
\versal{GSB} é apoiado pelo \emph{Oak Ridge National Lab}, dos Estados Unidos, e
pelo consórcio público"-privado holandês Be"-\versal{BASIC}, voltado para a criação
de conhecimento e tecnologias para estimular a química industrial
sustentável. O papel da \versal{FAPESP} reside em analisar as possibilidades de
produção de etanol de cana"-de"-açúcar na Colômbia, Guatemala, Moçambique
e África do Sul, sendo que a meta é ``incentivar o desenvolvimento
sustentável dos biocombustíveis e analisar a possibilidade de substituir
25\% da energia usada hoje no planeta por bioenergia''.\footnote{Cf.
  \textless{}\emph{https://cutt.ly/Oysie44}\textgreater{}
  e
  \textless{}\emph{https://bit.ly/3fdlV2U}\textgreater{}.}

Também em 2014 vimos a saber que a consultoria de negócios \emph{Bain \&
Company} e o escritório de advocacia Machado, Meyer, Sendacz e Opice
Advogados concluíram um amplo estudo sobre o setor sucroalcooleiro no
oeste da África. Esse estudo foi encomendado pelo \versal{BNDES} e concluiu que
existe na União Econômica e Monetária do Oeste Africano (Uemoa) um
potencial para a produção de 950 mil toneladas de açúcar, o que poderia
resultar em \versal{US}\$ 1 bilhão para a economia local, além da possibilidade
de implantação de polos agroenergéticos capazes de produzir 600 mil
metros cúbicos de etanol e 310 \versal{GW} de energia elétrica, gerando \versal{US}\$ 437
milhões. O impacto do setor seria de algo em torno de 1\% a 5\% do \versal{PIB}
da região, formada por Senegal, Mali, Níger, Burkina Faso, Benin, Togo,
Costa do Marfim e Guiné Bissau.\footnote{Cf.
  \textless{}\emph{https://bit.ly/3lISgkx}\textgreater{}.}

Não constitui mera casualidade que um dos sites a veicular as notícias
relativas ao avanço de empresas e agências de pesquisas brasileiras em
continente africano se chame ``Cultura ambiental nas escolas'': os
avanços no plano econômico só se tornam duráveis se houver um suporte
ideológico que reafirme e legitime os tipos de atividade econômica
empreendidos. Uma notícia de junho de 2014 nos informa que o ritmo de
crescimento do comércio brasileiro com a Nigéria, Gana, Senegal e África
do Sul diminuiu, estagnou ou recuou entre 2010 e 2013, devido a
restrições fiscais sobre produtos e a maior concorrência da China,
contudo, nesse mesmo período ``o comércio com Angola continua a
melhorar, em virtude dos fortes laços culturais e da língua
portuguesa''. Ora, são estes laços fortes que garantem a capilaridade do
poder das empresas, e é aqui que entra, com peso, a assimilação das
pautas identitárias por parte das empresas mais modernas e dinâmicas.

A incorporação de mulheres, de negros e de nativos nas instituições
voltadas para o \versal{P\&D} e nos espaços de comando (de baixo patamar) das
empresas constitui uma faceta da estratégia de legitimação e reforço da
empresa na região onde se está a investir. Já em 2003 a promoção da
diversidade racial no setor privado foi objeto de estudo (cf. \versal{MYERS},
2003) onde se concluía que ``apostar na diversidade, e especificamente
na diversidade racial, é contribuir para uma sociedade mais justa e uma
economia mais competitiva'' e se ``recomendava'' às empresas ``abraçarem
o \emph{valor}, tanto ético como econômico, da diversidade'', passo a
passo:

\begin{enumerate}
\def\labelenumi{\alph{enumi}.}
\item criar um comitê/conselho de diversidade,

\item realizar treinamentos internos e externos (com fornecedores),

\item iniciar um diálogo/criar parcerias com entidades do terceiro setor e
com outras empresas sobre e para a promoção de diversidade,

\item buscar alinhamento entre ações externas e ações internas na promoção
da diversidade,

\item mensurar o número de minorias no quadro de funcionários,

\item estabelecer metas específicas de longo prazo para aumentar a presença
dessas minorias na empresa em todos os níveis,

\item oferecer incentivos financeiros aos gestores para cumprirem essas
metas,

\item adotar ações afirmativas. (\versal{MYERS}, 2003: 20)
\end{enumerate}

Comentando as relações comerciais entre Brasil e Angola o gerente Dakalo
Mboyi, do \emph{Grupo Safmarine Container Lines} (especializada no
transporte marítimo de cargas em contêineres entre a África, o Oriente
Médio e a Índia), afirma que ``O setor agrícola local é inexistente, por
isto os produtos vindos do Brasil não são fáceis de substituir
localmente. Levará um longo tempo até que Angola consiga desenvolver sua
própria indústria agrícola e isto proporciona oportunidades para o nosso
País oferecer mais produtos além do açúcar''.

Segundo Dirk Van Hoomissen, diretor"-gerente da \emph{Safmarine} no
Brasil, ``a África representa uma nova fronteira de crescimento para o
Brasil. Há oportunidades significativas para expandir, mas somente as
maiores empresas do Brasil estão começando a enxergá"-las, tais como
explorar a proteína, as bebidas, maquinários ou o mercado de
fertilizantes em todo o continente africano''. O gestor aponta
``oportunidades reais para o aumento de dois dígitos, mas eles dependem
do forte comércio e dos laços culturais e políticos entre o Brasil e os
principais mercados da África. Quando se trata de África, considera"-se a
Ásia e Europa como principais parceiros comerciais e não a América
Latina''.\footnote{Cf. \textless{}\emph{https://bit.ly/3f9tZlo}\textgreater{}}

Entre 2003 e 2012 as exportações brasileiras para o mercado africano
passaram de \versal{US}\$ 2,9 bilhões em 2003 para \versal{US}\$ 12,2 bilhões em 2012,
portanto cresceram 327\%. Os principais produtos exportados em 2012
foram o açúcar (33,3\% do total), carnes (14,6\%), cereais (12,9\%),
veículos e autopeças (5,4\%), minérios (4,9\%), máquinas e aparelhos
mecânicos (4,8\%) e aviões (2,2\%), portanto se destacaram as vendas de
produtos intensivos em tecnologia, como aeronaves da Embraer. Já quanto
às importações, nos últimos dez anos o Brasil importou 334\% mais da
África, passando de \versal{US}\$ 3,291 bilhões em 2003 para \versal{US}\$ 14,265 bilhões
em 2012. Na pauta importadora predominam os produtos naturais não
renováveis, com destaque para o petróleo (61\% do total) e gás e
derivados, seguidos pelos adubos e fertilizantes, cacau e ferro
fundido.\footnote{Cf. \textless{}\emph{https://bit.ly/36PIqXU}\textgreater{}.}

Uma notícia de fevereiro de 2013 nos informa que a fabricante americana
de máquinas agrícolas \versal{AGCO} inaugurou em 2012 a primeira
\emph{fazenda"-modelo} na Zâmbia, com o objetivo de incentivar a
mecanização no continente. A empresa instalou na Argélia, em 2013, uma
planta de produção de tratores da marca \emph{Massey Ferguson}, em
parceria com o governo local e já aventando a possibilidade de expansão
das fazendas"-modelo para Moçambique, Nigéria, Senegal e Etiópia, em
parceria com governos e empresas. A empresa avalia que o continente
africano carece de assistência técnica e funcionários preparados para
operar suas máquinas, que muitas vezes ficam ``encostadas'' por falta de
peças e de manutenção adequada, mas que a própria presença da empresa em
território africano mostra que há uma expectativa de crescimento do
mercado local. ``De repente, daqui a dez anos aumenta o mercado, como
{[}aconteceu{]} no Brasil'', disse André Carioba, vice"-presidente sênior
da \versal{AGCO} para a América do Sul. Ora vejam a amplitude estratégica da
operação, que visa ``de repente'' ter lucratividade ampliada daqui a 10
anos e, por isso, mantém sua ``presença'' na região.\footnote{Cf.
  \textless{}\emph{https://bit.ly/36TUE1Y}\textgreater{}.}

Trata"-se de gestores competentes, que mantém os olhos voltados para o
longo prazo e, por isso, não deixam passar as oportunidades de lucro
inerentes à adoção da agenda de diversidade em suas mais variadas
formas, como vimos ao longo deste livrinho.

\section{Populações}

Além do sentido geopolítico e de divisão internacional do trabalho a
relação de empoderamento da ``periferia'' tem ainda um segundo sentido,
relacionado às próprias populações que residem nas periferias das
grandes cidades. Também neste segundo sentido a política identitária
atua, contribuindo com o sucesso da função estatal (feita em parceria
estreita com as empresas enquanto aparelhos de poder) de administração
lucrativa das crises econômicas.

O economista belga Ernst Mandel caracteriza a função estatal de
administrar as crises do capital da seguinte forma:

\begin{quote}
\emph{Economicamente} falando, essa ``administração das crises''
inclui todo o arsenal das políticas governamentais anticíclicas, cujo
objetivo é evitar, ou pelo menos adiar tanto quanto possível, o retorno
de quedas bruscas e catastróficas como a de 1929/1932.

\emph{Socialmente} falando, ela envolve esforço permanente para
impedir a crise cada vez mais grave das relações de produção
capitalistas por meio de um ataque sistemático à consciência de classe
do proletariado. (1982: 341)
\end{quote}

O pensador belga pontua ainda os aspectos político"-ideológicos dessa
função econômica direta desempenhada cada vez mais pelo Estado
capitalista: afirma que a \emph{vasta maquinaria de manipulação
ideológica} voltada para a ``integração'' do trabalhador à sociedade
capitalista enquanto consumidor, ``parceiro social'' ou ``cidadão'' está
intimamente relacionada à função econômica estatal de administração das
crises do capital. Na função de administrador das crises pode"-se ver
claramente o caráter de \emph{todo orgânico} das funções estatais
(função coercitiva, função integrativa, função de garantir as condições
gerais de produção e função de administrar as crises) e quanto a este
aspecto de totalidade do processo vale citar o exemplo das Unidades de
Polícia Pacificadora implantadas nos territórios cariocas.

As \versal{UPP}s brasileiras são um óbvio exemplo de atuação repressiva do
Estado, uma vez que promovem uma regulação armada de territórios
considerados estratégicos para a realização de um \emph{modelo
empresarial de cidade}, ou seja, são impulsionadas por fatores
diretamente econômicos, tanto em sentido amplo e de larga escala
temporal, quanto no que diz respeito ao curto prazo: a garantia de
realização dos megaeventos na cidade do Rio de Janeiro e de todas as
obras e mudanças urbanas aí implicadas. Desse modo, as \versal{UPP}s intercalam
modalidades estatais de intervenção urbana e de segurança pública num
contexto de queda nos índices de crescimento econômico e, no limite,
crise capitalista.

Nesse percurso há uma crescente tendência de redução da política a
``política de segurança'', onde há um ``agigantamento do aparato estatal
vigilante, coercitivo e repressivo'', de modo que outras áreas
importantes da atuação do poder público passam a ser perpassadas pelo
vetor de ``segurança''. A escassa presença ``social'' do Estado convive
com a disseminação de um modelo de cidadania mediado pelo consumo, com o
``requentamento e o requintamento'' da ideologia neoliberal da livre
iniciativa, do empreendedorismo empresarial e do ``empresariamento de si
mesmo'', tudo isso somado às ``doses cavalares de onguismo'', que vemos
nas periferias das grandes cidades. (cf. \versal{BRITO} \& \versal{OLIVEIRA}, 2013)

Nas periferias brasileiras um dos resultados mais expressivos do
identitarismo é o elogio à ``identidade periférica'' ou ``identidade do
favelado''. A fim de ilustrar o modo como a política identitária se
associa à ideologia neoliberal do empreendedorismo de si mesmo, se
entranhando nas periferias das grandes cidades, vamos partir de alguns
exemplos emblemáticos, dados por Lívia de Tommasi (2013).

O primeiro deles é o caso de Eleilson Leite. Formado na Escola da
Pastoral da Juventude e do Movimento dos Sem Terra, Eleilson tornou"-se
diretor de uma \versal{ONG} paulista que desde 2007 produz a \emph{Agenda
Cultural da Periferia}, promovendo e divulgando eventos culturais que
acontecem nas periferias da cidade. O guia tem periodicidade mensal e
tiragem de 10 mil exemplares, que são distribuídos em bares, escolas,
casas de cultura, mercados, lanchonetes, feiras etc., e conta com um
programa semanal na Rádio Heliópolis. Segundo Eleilson

\begin{quote}
é por meio da música, literatura, cinema, teatro, dança e de outras
linguagens artísticas que o povo pobre dos fundões da metrópole vem
dando um novo significado à expressão periferia. Uma denominação que até
os anos 80 designava genericamente o amontoado de precárias residências
que ocupam as franjas das grandes cidades. Hoje, a cultura de periferia
vem na contramão da história, desmontando os estigmas e os estereótipos
que pesam sobre os arrabaldes. Nos becos e vielas do subúrbio pulsa uma
arte original, criativa e vibrante. {[}\ldots{}{]} A cultura na periferia
surge como elemento aglutinador da comunidade. {[}\ldots{}{]} E a comunidade,
diferente do movimento de tipo reivindicatório, se expressa em função do
que tem e não por aquilo de que carece. Sendo assim sua manifestação tem
um potencial mobilizador de grande intensidade. A cultura gera
movimentação social, desperta consciências, embrenha processos
políticos, promove transformações. {[}\ldots{}{]} Na periferia, sem arte não
há transformação. E para transformar há que se produzir uma cultura
própria, porque a ``arte que liberta não pode vir da mão que
escraviza'', como disse o poeta Sérgio Vaz no brilhante manifesto da
Semana de Arte Moderna da Periferia, evento emblemático realizado em
novembro de 2007 em São Paulo. (E. Leite, Boletim Juventude em Cena,
Ação Educativa, 2008).
\end{quote}

A \emph{Agenda Cultural da Periferia} está organizada em cinco sessões:
literatura, Hip Hop, cinema e vídeo, samba, teatro, além de reservar
espaço para a divulgação de ``manifestações periféricas'' que acontecem
no centro de São Paulo. A literatura ocupa lugar de destaque, com uma
influência importante da cultura Hip Hop. Enquanto a hegemonia musical
passou do Rap para o Funk, e o grafite se tornou a linguagem artística
``oficial'' da periferia, tendo inclusive ``conquistado'' (será uma
conquista ou uma assimilação capitalista?) espaços em museus e
exposições, a literatura e a poesia produzida nas periferias paulistas
se tornou um novo produto dos rappers, que a converteram numa espécie de
produto artístico autônomo em relação à música.

A capitalização da cultura periférica não apareceu de fora para dentro e
sim a partir de dentro, brotando no solo fertilizado pela racionalidade
neoliberal do empresariamento de si mesmo. A periferia não esperou que
os produtores culturais do ``centro'' valorizassem suas produções: os
artistas periféricos viraram produtores de si mesmos, produzindo e
vendendo seus próprios livros, organizando saraus e outros eventos para
difundir suas obras. Aqui o empreendedorismo de si mesmo encontra o
identitarismo, numa confluência ideológica e política assombrosa.

Lívia de Tommasi explica que os produtores periféricos também ``criam
empreendimentos que produzem riqueza, material e simbólica, para e na
periferia'' e cita outros exemplos: Ferréz criou uma empresa e abriu
duas lojas que vendem os produtos de sua grife (a \emph{1DaSul}); Allan
de Rosa (que fez mestrado em educação na \versal{USP}) fundou a editora
``\emph{Edições Toró}'' e uma \versal{ONG} para viabilizar a realização de cursos
de formação sobre a temática indígena e racial e de criação literária.
Sérgio Vaz, Binho e muitos outros animam espaços culturais na periferia
e mantêm site ou blog na internet para divulgar suas obras e suas
opiniões (\versal{TOMMASI}, 2013: 16).

Lívia explica que essas iniciativas artísticas e culturais representam
também um meio de sobrevivência de grupos sociais subalternos que
geralmente têm acesso a postos de trabalho caracterizados pela baixa
qualificação, remuneração e gratificação pessoal. ``Assim, a dimensão
econômica dessas iniciativas se entrelaça com a dimensão política''
(ibid: 17). Para ilustrar a consciência do processo, Lívia cita Sérgio
Vaz: ``O único espaço público que tem na favela é o bar. Você imaginou
que a gente ia se acabar tomando cachaça? E a gente transformou os bares
em centros culturais''. Essas iniciativas ganham fôlego ao fornecer aos
moradores espaços legítimos de manifestação de ideias etc., ou seja,
espaços de garantia do direito à expressão, em uma sociabilidade que os
têm silenciado desde sempre.

Embora o processo de empoderamento, como um todo, resulte na
mercadorização da cultura ``da periferia'', muitas vezes as iniciativas
se apresentam como sendo ``de resistência'', ou seja, voltadas para a
contestação da mercadorização da cultura, o que constitui um paradoxo:
as rodas de samba, por exemplo, tal como o \emph{Samba da Vela} em Santo
Amaro, ``representam uma forma de voltar a valorizar o samba de raiz e
escapar, dessa forma, das amarras artísticas, políticas e mercadológicas
das escolas de samba. Nesse âmbito é significativa a valorização da
cultura negra e indígena brasileira.'' (ibid: 18). Esse exemplo também é
conhecido no Rio de Janeiro, onde se tornou prática costumeira, nos
últimos anos, a realização de Rodas se samba sem divulgação além da
``boca miúda'', o que visa não só garantir que estarão presentes apenas
``amigos'' e a comunidade do bairro, como ainda se preservar da
assimilação capitalista dessas manifestações culturais. Aqui o
identitarismo assume feições microrregionais e bairristas, garantindo o
status de sujeitos e grupos culturais locais que não conseguiram (ou
mesmo não quiseram) alavancar a carreira para além do âmbito do bairro.

Lívia de Tommasi comenta que o compromisso político de Eleilson o leva a
se esforçar pela organização de um movimento cultural da periferia.
Entretanto o próprio empreendedor cultural observa as dificuldades da
empreitada: os diferentes segmentos periféricos se assemelhariam mais a
``tribos'' onde ``os integrantes de uma dificilmente se misturam e
dialogam com os da outra, ou realizam projetos em conjunto''.

A questão não é meramente ``antropológica'': no momento em que concorrem
para os editais e financiamentos, públicos e privados, todos,
inevitavelmente, competem pelo acesso a recursos. (ibid: 18) Chegamos
então num ponto"-chave da questão: os recursos. Um fator que contribuiu
para produzir a efervescência cultural da periferia paulista foi o \versal{VAI}
(Programa de Valorização das Iniciativas Culturais), um fundo de apoio
instituído por uma lei municipal criada durante o governo petista de
Marta Suplicy e ampliado durante os governos Serra e Kassab. Lívia
salienta que os projetos que o \versal{VAI} financia nas periferias ``são em
número muito reduzido, em comparação com os financiados nas regiões
ricas da cidade'' (ibid.: 19).

O sucesso das iniciativas empresariais das periferias está erradicado na
confluência funcional entre neoliberalismo e identitarismo, o que
significa que esse processo assenta no âmbito mesmo da função estatal
geral de apassivamento das classes trabalhadoras. Por esse motivo a
confluência, na medida em que se põe de um lado como rebeldia e
``direito a voz'' dos ``excluídos do sistema'', é na verdade uma
confluência perversa, uma trágica assimilação capitalista dos produtos
culturais da classe trabalhadora que, submersa na racionalidade
empreendedora, se esforça por converter em mercadoria suas expressões
culturais ``periféricas'' originalmente concebidas como expressão
antagônica justamente à sociedade da mercadoria.

Uma vez que se mostra lucrativo, o processo passa a ser incentivado pelo
Estado e empresas. Como disse o secretário de segurança
pública\footnote{Beltrame em \emph{Polícia e Comunidade} Canal Multi
  Show, 2009. Disponível em:
  \textless{}\emph{https://www.youtube.com/watch?v=s9EVSoA0xQ0\&feature=relmfu}\textgreater{}.} do \versal{RJ}:

\begin{quote}
Há que existir um tsunami de ações sociais. Essa guerra ela só será
vencida se os outros projetos forem comigo, e boto aí uma dose grande na
iniciativa privada. Que a iniciativa privadas ela não pode mais ver o
Dona Marta e o Cidade de Deus como a Favela Dona Marta e a Favela Cidade
de Deus.
\end{quote}

Muitas vezes essa atuação estatal e empresarial se dá sob a roupagem de
\versal{ONG}s, e o resultado final do processo tem sido a assimilação, por parte
dos trabalhadores, da ideologia identitária e da racionalidade
neoliberal, as quais ganham expressividade no próprio âmbito da classe.
Isso tem implicações seríssimas para as lutas sociais na medida em que a
politica identitária amplia as cisões entre os trabalhadores, ancorados
em matrizes de raça, gênero, sexualidade, cultura e, neste caso, local
de moradia (identidade territorial).

A cisão no próprio bojo das organizações da extrema esquerda, por
exemplo, entre quem tem direito de fala e quem não tem, dentro das
próprias fileiras da classe trabalhadora, se apresenta inicialmente como
rebeldia contra o silenciamento histórico das opressões de gênero, raça
e sexualidade, mas termina pela neutralização de todo e qualquer
militante que não se encaixe no padrão biologizante de ``sujeito
periférico historicamente oprimido'' ou que, partindo de um ``lugar de
fala opressor'' (por exemplo, se for homem, branco e heterossexual) ouse
discordar das ideias ou da estratégia política sendo posta em prática.
Para estes é reservada a posição de apoiador da luta, repondo na relação
entre protagonista e apoiador da luta identitária uma modalidade mais
desenvolvida da nefasta polaridade entre base militante e direção, que
historicamente impregnou as lutas classistas.

\begin{quote}
``Ser'' e ``estar'' na periferia é argumento sempre colocado para
legitimar a fala. Expressões como ``nasci e cresci na comunidade tal'',
proferidas no começo de uma fala pública, colocam naturalmente um
divisor de águas entre os presentes. Como disse Allan de Rosa no
seminário supracitado: ``Cep é vivência''. Dessa forma, esses artistas
se contrapõem a toda uma herança histórica de desapropriação da fala, a
quem continua a falar ``em nome de'' mesmo ocupando um outro lugar
social e geográfico. As falas que, no seminário, colocavam a ideia de
superação e de fronteiras borradas entre centro e periferia, alguns
participantes (especificamente entre o público) responderam convidando
os presentes a ir pegar um ônibus lotado, pela manhã cedo, para
constatar que ainda há, e bem evidente, segregação e desigualdades nas
condições de vida dos moradores da cidade. ``A gente não reside,
resiste'', disse o \versal{GOG} no seminário. (\versal{TOMMASI}, 2013: 19)
\end{quote}

O que se apresenta como legítimo e progressista termina por servir aos
interesses políticos e econômicos da ordem, revigorando e gerando lucros
para a indústria da cultura. Já que o produto ``periférico'' ``traz em
si as marcas fortes da identidade territorial, reivindicada e positivada
por essa produção, a venda do produto implica valorização do lugar'', e
é por isso que setores do governo e empresariado patrocinam a
valorização da periferia como ``lugar onde se produz cultura, e não
somente violência e marginalidade''. Lívia comenta que há um número
significativo de produtores culturais e jornalistas egressos das escolas
da elite (como por exemplo, a \versal{FGV}) dispostos a investir na difusão
dessas manifestações, e cita como exemplo o site ``\emph{DoLadoDeCá}''
da jornalista Tati Ivanovici.\footnote{Cf.
  \textless{}\emph{www.doladodeca.com.br}\textgreater{}.} A
confluência de movimentos e o crescente interesse do mercado da arte
pelo produto ``favela'' ou ``periferia'', não ocorre por acaso:

\begin{quote}
A produção do quadro ``Central da Periferia'' no programa Fantástico da
Rede Globo, em 2006, por Regina Casé e o antropólogo Hermano Vianna; a
organização da exposição ``Estética da Periferia --- inclusão cultural e
cidadania'' em 2005 pelo produtor Gringo Cardia e a escritora e
professora Heloisa Buarque de Holanda; a realização do Festival Visões
Periféricas --- Audiovisual, Educação e Tecnologia (que em 2012 realizou
a 6ª edição); a recente organização da exposição ``O Design da Favela''
no Centro Carioca de Design na Praça Tiradentes, são exemplos
significativos desse interesse. As falas de Gringo Cardia no seminário
de São Paulo não deixaram dúvida sobre o valor mercadológico dessa
operação: ``Tira uma peça do camelódromo, tira pela excelência, coloca
aquilo isolado numa sala e você passa a ver o que, de outra forma,
ficaria invisível.'' (ibid: 21)
\end{quote}

E damos mais uma volta no parafuso do identitarismo se observarmos que o
movimento de mercadorização da ``cultura periférica'' ganhou corpo
depois das instalações das Unidades de Polícia Pacificadora (\versal{UPP}) nas
favelas cariocas, a partir de 2008.

Foi a partir da implantação das \versal{UPP}s em 2008 que as confluências de
interesses entre Estado, empresas e indivíduos foram canalizadas num
objetivo de lucro e apassivamento assentado na capacidade criativa das
populações faveladas, um movimento irresistível, em especial se
pensarmos a ausência de oportunidades e o estigma que sempre pesou sobre
os moradores das favelas, em especial sobre a população negra.

Acostumados à criminalização, como poderiam os favelados recusar o leque
de oportunidades políticas e econômicas desse processo? ``Favela é
potência'' diz o produtor cultural Marco Faustini, criador da ``Agência
de Redes para a Juventude'', um programa social que estimula jovens das
``comunidades pacificadas'' a realizar empreendimentos socioculturais
(\versal{VELAZCO}, 2012). Como pondera Livia de Tommasi: ``mais um exemplo da
instrumentalização da cultura para resolver problemas sociais''. (ibid:
21).

Ao comandar, desde a formação dos policiais das \versal{UPP}s, até os mecanismos
de financiamento, o poder público, em associação com as empresas,
controla todo o processo, incentivando a proliferação de identitarismos
atrelados à expressão criativa de produtos artísticos e culturais ao
mesmo tempo em que barram as iniciativas potencialmente antissistêmicas
e dão vazão às mais lucrativas.

\begin{quote}
O teor e a quantidade das informações que circulam na mídia, e em
particular na mídia eletrônica, não deixam dúvida: no Rio foi inventado
um produto, a favela pacificada, lugar de criatividade, inovação,
produção artísticas das ``pessoas do bem'', para se contrapor à imagem
de perigo, violência e marginalidade que durante muito tempo foi
divulgada na mídia. Projetos, programas, empreendimentos turísticos
visam vender ``o encanto das favelas'' (nome de um concurso de
fotografia organizado pela \versal{ONG} Viva Rio, no âmbito do programa ``Viva
Favela''). {[}\ldots{}{]} Paralelo a isso, há iniciativas mais antigas, como
o movimento cineclubista, os grupos de rock independentes da baixada
fluminense e os anarco"-punk das ocupações. Há uma miríade de grupos e
indivíduos que fazem arte, música, poesia, teatro, vídeo, promovem
iniciativas de forma independente, dentro e fora das favelas. Alguns
mais, outros menos, tentam dialogar com o poder público, procurando
encontrar algum financiamento. Hoje tanto o Ministério da Cultura como
as secretarias estaduais e municipais de cultura ampliaram bastante a
oferta de editais. Mas, como lembram meus interlocutores, os editais,
com suas regras, colocam um filtro: para concorrer é preciso saber que
existem e é preciso estar preparado, regularizado, dominar as técnicas,
os códigos e as regras de conduta. (\versal{TOMMASI}, 2013: 22)
\end{quote}

Apoiadas no identitarismo do sujeito periférico, negro, mulher da favela
etc. as chamadas ``culturas de periferia'' estão sendo valorizadas, no
Rio de Janeiro, como parte da produção de um novo regime discursivo que
visa promover o ``encontro'' da favela com o asfalto\footnote{Dois
  exemplos sintomáticos das manobras: o ``Museu do Encontro'', uma
  proposta do antropólogo Hermano Viana, da artista Regina Casé e do
  produtor cultural Gringo Cardia, para ``celebrar o encontro'' entre a
  favela e o asfalto. Cf.
  \textless{}\emph{http://riodeencontros.wordpress.com/2010/10/28/ummuseu-para-celebrar-o-encontro}\textgreater{}.
  E um monumento, colocado no Largo da Carioca em 2010, por ocasião do
  ``dia da favela'' (4 de novembro) e que vinha com os dizeres
  ``Favela'' e ``Rio'', entremeados por um enorme coração.},
proporcionando uma cidade ``pacificada'' e em vias de ser ``integrada'',
a suposta superação da ``cidade partida'', premissas sobre as quais se
apoia a proposta do programa \emph{\versal{UPP} Social}.

\begin{quote}
A referência direta é a ideia de ``cidade partida'' do jornalista Zuenir
Ventura: graças à ocupação policial e à ``libertação'' dos territórios
retirados do poder do ``tráfico'', a separação pode ser agora superada
por meio de dispositivos de promoção da ``integração'' e do
``encontro''. Nessa operação, as práticas discursivas sobre a cidade são
reconfiguradas. A difusão de imagens positivas na mídia, sobretudo na
digital, espaço privilegiado para a divulgação dos projetos de \versal{ONG}s e
governos, nomeiam a favela não mais como lugar do tráfico, da violência,
do perigo, do medo e sim, como lugar da solidariedade, da riqueza
cultural, artística e estética, num discurso que exalta a capacidade
empreendedora e criativa da população local. Como exemplo, podemos citar
o concurso fotográfico ``Encantos da favela'', promovido pelo portal
Viva Favela, um projeto da \versal{ONG} Viva Rio. Artistas, intelectuais,
curadores e promotores turísticos estão ajudando a construir e promover
um produto, a favela pacificada, lugar de criatividade, inovação e
produção artísticas das ``pessoas do bem''; lugar, inclusive, onde é
possível fazer turismo e desfrutar das lindas vistas sobre a ``cidade
maravilhosa'' que oferecem as favelas situadas na Zona Sul da cidade.
(\versal{TOMMASI} \& \versal{VELAZCO}, 2013: 20)
\end{quote}

Apassivamento da classe e, de brinde, lucros extras de onde não se
esperava. Não por acaso as gestões petistas, assentadas na propalada
``participação'', que caracterizou a estratégia democrático"-popular,
foram aquelas que melhor se adaptaram a todo o processo decorrente desse
modelo de gestão da cidade e das contradições urbanas, intensificando o
diálogo com a periferia e levando muitos artistas e produtores da ``cena
independente'' a assumiram cargos públicos.

Tratando da particularidade pernambucana, Lívia comenta que lá a
articulação dos artistas periféricos locais não se concretizou ``pela
falta de recursos e capacidade de organização'', e finaliza: ``faltaram,
provavelmente, empreendedores culturais periféricos como os que animam a
cena paulista''. No caso de Recife ao que parece predomina uma clivagem
de classe onde os coletivos que ocupam a cena da produção musical são
formados por jovens universitários de classe média que divulgam, nos
festivais independentes que organizam, produtos produzidos por eles
mesmos, dificultando que bandas da periferia ultrapassem o cerco e
conquistem espaço no mercado (ibid: 24). Tal como noutros casos, aqui a
política identitária serve a propósitos de reserva de mercado. Numa
plataforma identitária intimamente articulada com a mercadorização da
cultura, é feita a crítica de que haveria a imposição de um padrão
estético dominante:

\begin{quote}
é aquele esquema das artes visuais, das artes plásticas, das artes
contemporâneas: você primeiro precisa ter um curador para dizer para
você e para todo mundo que aquele trabalho que você faz é de boa
qualidade, é um trabalho expressivo que merece ser respeitado como
expressão artística de verdade. Mas a questão é: quem são os curadores?
Os curadores são os caras que vêm da classe média, naturalmente, por
quê? São os caras que estudaram. {[}\ldots{}{]} hoje o grafite precisa ser
respeitado como arte e para isso no discurso deles tinha que estar na
galeria, tinha que ser vendido caro. {[}\ldots{}{]} tem um discurso que
valoriza contanto que seja aquele cara que eu tire daquele contexto lá e
diga ``olhe que coisa, esse é um representante daquilo e por ser
representante ele é o melhor''. O cara que é o melhor é o cara que se
destaca porque faz um trabalho que consegue agradar, ou estar dentro de
um padrão que já está estabelecido, então tudo que for fora daquilo não
vai ser tirado como destaque. (João Lin, apud: \versal{TOMMASI}, 2013: 26)
\end{quote}

Como se vê, a questão de classe não aparece enquanto questionamento à
inserção na lógica capitalista, e sim enquanto desigualdade de
oportunidades na corrida pelo reconhecimento e ascensão artística etc.,
além de acesso às verbas dos financiamentos em editais estatais. No
limite, a crítica incide na desigualdade de acesso ao mercado consumidor
da arte produzida na periferia, ou seja, ainda se trata de galgar a
escada do empreendedorismo de si mesmo, onde a ``vitória'' coincide com
o enriquecimento ``pela via de sua própria arte e trabalho''. Nessa
chave, qualquer potencialidade antissistêmica fica de antemão obstruída.
Ainda assim, por seu próprio caráter e \emph{habitat}, o processo não
deixa de carregar em si contradições:

\begin{quote}
O ``fazer por nós mesmos'' em vez de esperar que o poder público supre a
falta de equipamentos e de serviços culturais que caracterizam os
bairros de periferia; a ideia de resgate, a valorização da identidade
territorial periférica; a afirmação da autonomia, que se expressa
também, às vezes, na recusa a se submeter à normatização implícita nos
editais, a denúncia das condições precárias dos serviços públicos da
região, o tema do desemprego sempre presente, do trabalho precário, da
exploração, são conteúdos importantes da produção cultural periférica
paulista. Expressões como ``o mundo é diferente da ponte para cá'' (a
ponte é a que cruza a marginal, na zona sul da cidade), ou ``periferia é
periferia em qualquer lugar'', extraídas das letras dos Racionais Mc's,
são paradigmáticas dessa postura política. Afirmar com orgulho de ``ser
da periferia'' é uma experiência inédita. Nesse sentido, ser ou não da
periferia é um dado fundamental que legitima a fala. Questão central
para os chamados ``novos movimentos sociais'', a questão identitária (no
caso, territorial) deve ser interrogada e não naturalizada, ou colocada
simplesmente como uma conquista que supera a (supostamente velha)
questão de classe. (\versal{TOMMASI}, 2013: 27)
\end{quote}

Além da ideologia e políticas identitárias e de todo o arcabouço estatal
e empresarial voltado ao estrangulamento das potencialidades subversivas
e antissistêmicas do processo, temos ainda a questão da presença das
Igrejas, forjando a identidade do ``crente'', restringindo a consciência
de classe e levando as indagações para o plano individual, afinado com a
racionalidade neoliberal. E temos, como já mencionado, a presença
fragmentadora da ideologia identitária e suas implicações nefastas para
a unidade da classe trabalhadora. ``O \emph{acionamento identitário}
dos artistas periféricos paulistas opera uma afirmação política,
enquanto reivindicação do pertencimento territorial a uma periferia
simbolicamente unificada como alteridade, contraposta ao centro
dominante'', e por essa via ``apela para o reconhecimento político de
uma alteridade positivada'', representando"-se como um ``ato de
resistência''.

\begin{quote}
Quando esse acionamento identitário vira produto de mercado e é
capturado pelo discurso oficial, como no contexto da celebração do
talento artístico dos moradores das favelas (que seriam ``naturalmente
criativos'') operada pelo discurso que projeta a imagem de uma cidade
supostamente ``integrada'', vêm à tona seus limites políticos. Limites
que, me parecem, dizem respeito à chamada ``política de identidade''
(\versal{FRASER}, 2007). (ibid: 28)
\end{quote}

É através do acionamento identitário que tem se dado a disputa por um
lugar no mercado e pelas verbas dos editais. A inclusão social pelo
mercado da arte ocorre por meio da expressão da alteridade, territorial
e política. Nesse sentido os artistas periféricos convertem sua
localização periférica em valor agregado de suas mercadorias, o que é
aceito e estimulado por um mercado sedento de ``inovações'', mesmo
quando tal se transveste de ``descoberta"-afirmação'' de um ``lado bom da
periferia''.

Se a produção cultural periférica sempre existiu, o que provocou essa
recente explosão de visibilidade? Como bem coloca Lívia, ``foi a
capacidade de seus protagonistas, ou também uma conjuntura política
favorável?'' e como se perguntou Renildo Oliveira, do \emph{Movimento
Cultural d@s Guaianás}, \emph{Cine Campinho} e \emph{Arte Maloqueira da
Zona Leste}: ``Foi conquista ou concessão?''.

Lívia afirma que no caso do Rio de Janeiro essa visibilidade serve a
interesses políticos mais ou menos claros, como parte da venda da
``cidade maravilhosa'', agora finalmente ``pacificada'' para acolher os
grandes eventos esportivos mundiais (e os interesses econômicos que
movimentam) (ibid: 29). E com o ganho de expressividade da ``cultura de
periferia'' convertida em mercadoria temos a transformação social
convertida em projeção e afirmação pessoal, tudo conforme aos interesses
capitalistas e à correspondente racionalidade neoliberal. O processo é
percebido por alguns, que aos poucos vão questionando a lógica do jogo:

\begin{quote}
A inserção no mercado, a relação com os governos e as fontes de
financiamento são questões sempre abertas e presentes nos debates entre
os artistas periféricos. Como vender sem fazer desaparecer o conflito, a
carga de ruptura, a crítica à ordem vigente? Como negociar com políticos
e empresários sem virar palanque eleitoral dos poderosos? {[}\ldots{}{]}
Dessa forma, os moradores das periferias afirmam seu direito a fazer
arte, sair da invisibilidade e da criminalização e se afirmar enquanto
produtores de arte. A postura política se expressa no conteúdo veiculado
nas letras, na vontade de se expressar e falar da própria condição de
vida. Para alguns, sobretudo os mais jovens, é também a expressão da
vontade de fugir ao destino: nem bandido nem mão de obra barata, e sim
artistas. (ibid: 30 e 31)
\end{quote}

O problema é que a crise da esquerda e de seus órgãos de luta tem
colaborado para que as tentativas de saída da armadilha sejam modeladas
dentro da lógica fragmentária do identitarismo, o que afasta a
perspectiva da luta em unidade classista. Como explica Nancy Fraser, o
modelo da identidade é profundamente problemático:

\begin{quote}
Entendendo o não reconhecimento como um dano à identidade, ele enfatiza
a estrutura psíquica em detrimento das instituições sociais e da
interação social. Assim, ele arrisca substituir a mudança social por
formas intrusas de engenharia da consciência. O modelo agrava esses
riscos, ao posicionar a identidade de grupo como o objeto do
reconhecimento. Enfatizando a elaboração e a manifestação de uma
identidade coletiva autêntica, autoafirmativa e autopoiética, ele
submete os membros individuais a uma pressão moral a fim de se
conformarem à cultura do grupo. Muitas vezes, o resultado é a imposição
de uma identidade de grupo singular e drasticamente simplificada que
nega a complexidade das vidas dos indivíduos, a multiplicidade de suas
identificações e as interseções de suas várias filiações. Além disso, o
modelo reifica a cultura. Ignorando as interações transculturais, ele
trata as culturas como profundamente definidas, separadas e não
interativas, como se fosse óbvio onde uma termina e a outra começa. Como
resultado, ele tende a promover o separatismo e a enclausurar os grupos
ao invés de fomentar interações entre eles. Ademais, ao negar a
heterogeneidade interna, o modelo de identidade obscurece as disputas,
dentro dos grupos sociais, por autoridade para representá"-los, assim
como por poder. Consequentemente, isso encobre o poder das facções
dominantes e reforça a dominação interna. Então, em geral, o modelo da
identidade aproxima"-se muito facilmente de formas repressivas de
comunitarismo. (\versal{FRASER}, 2007: 106-107).
\end{quote}

Temos então a problemática política de identidade imbuída nas
resistências das periferias brasileiras, enquanto os mecanismos de poder
se articulam organicamente de modo a que a arte e a cultura sejam
instrumentalizadas para a gestão das populações periféricas. Essa gestão
é arquitetada a partir da própria mercadorização das resistências
sociais operadas por meio de empreendimentos culturais criados pela
população que vive nas periferias ou que, por diversos motivos, se
identifica e fala em nome da favela.

Trata"-se de uma faceta bastante complexa da atuação estatal com vistas à
administração das crises do capital, sintetizada por Maria Célia Paoli
quando a pesquisadora afirma que a lógica gestionária inerente ao modelo
neoliberal substituiu a \emph{política} pela \emph{gestão técnica de
territórios e populações}. A pesquisadora explica que a gestão técnica
das necessidades atua segundo o esquema ``problemas --- diagnósticos ---
soluções --- intervenções localizadas'' destruindo a política como
expressão de conflitos. ``A racionalidade técnica que se sobrepõe à
política visa tornar inoperantes as manifestações de contestação''.
Trata"-se de um modo de gestão da vida social que opera por meio da fusão
do Estado policial com o Estado gestor. (\versal{PAOLI}, 2007: 243).

Outros elementos da implantação das \versal{UPP}s são mais facilmente
reconhecíveis em seus vínculos com a atuação estatal visando a ampliação
dos lucros das empresas, a começar pelo próprio apassivamento e controle
territorial com vistas à legalização de serviços a serem adequadamente
cobrados (internet, gás, luz, abertura de bancos e de outras empresas
etc). Nesse amplo e complexo processo social um elemento identitário
ainda transparece enquanto fator importante para o sucesso da
empreitada. Pense"-se, por exemplo, no caso da favela Dona Marta, no
bairro de Botafogo. Depois de receber a primeira \versal{UPP}, alegadamente
enquanto uma administração armada voltada à ``pacificação'' de um
território ``dominado pelo tráfico'' etc., a empresa de luz \emph{Light}
viu"-se em condições de cortar as ligações irregulares de energia
elétrica, os ``gatos'', e 98\% das residências foram ligadas à rede
oficial de consumidores, que passaram a ``poder pagar'', ou melhor, a
``dever pagar'' as contas de luz à \emph{Light}. Não foram poucos os
moradores que ficaram felizes por poder se apresentar como ``cidadãos
honestos'' que pagam pela energia elétrica e pela \versal{TV} a cabo que
consomem, o que, em seus imaginários, requalifica e reafirma
positivamente suas identidades de ``favelado''.

Esses elementos são essenciais para pensarmos a função estatal de
administração das crises, uma vez que se veem vários segmentos
empresariais manifestando interesse em explorar o mercado consumidor em
potencial espalhado pelas mais de mil favelas existentes na capital do
Rio de Janeiro, um mercado que havia sido deixado de lado por parte do
circuito capitalista de produção e venda de mercadorias e serviços.

Na Comunidade pacificada da Dona Marta o governo do Estado abriu uma
linha de microcrédito para ``moradores empreendedores'' e
``microempresários'' locais, como parte do Programa \emph{Investe Rio},
ligado à Secretaria de Desenvolvimento Econômico, Energia, Indústria e
Serviços, que com tal linha especial de crédito visa, segundo
informações oficiais, ``aproveitar o potencial econômico da comunidade''
e, em outras favelas pacificadas na Zona sul da cidade, seu ``potencial
turístico'': a ``\emph{mise"-en"-scène} da favela S.A'' ``exibe a
comunidade e sua territorialização precária como uma mercadoria mais ou
menos exótica a ser vendida no nicho de mercado multiculturalista''
(\versal{BRITO}, ibid: 102) de modo que ``o que temos, no fundo, são as \versal{UPP}s como
suporte para um processo de instrumentalização da pobreza e da cultura
como alavancagem para a valorização imobiliária e fundiária'' (ibid:
103).

Valorização imobiliária e fundiária de um lado, capacidade de consumo e
capital inicial para iniciativas ``empreendedoras'' de outro,
autolegitimação e elogios ao sujeito morador das periferias enquanto
``sujeito periférico'': o vínculo econômico entre \versal{UPP}, administração
estatal"-empresarial das crises e políticas identitárias parece ser
inquestionável.

A pensadora indiana Ananya Roy tratou, em seu livro \emph{Capital
Pobreza} (2010), dos impactos dos programas de microcrédito no âmbito da
economia mundial, concluindo que se trata de um dispositivo importante
para a gestão da crise do sistema capitalista via financeirização da
pobreza. Não por acaso, Lívia de Tommasi e Dafne Velazco constataram que
em uma das comunidades geridas pela \versal{UPP} a chegada do Bradesco foi ``o
acontecimento mais significativo'' desde a ``pacificação''. E a seguir
as pesquisadoras expuseram o quadro em sua riqueza de elementos:

\begin{quote}
Na inauguração da agência, no dia 5 de janeiro de 2011, estava presente
o governador Sérgio Cabral. Como já mencionado, na agência só funcionam
os serviços de abertura de conta e empréstimos, além de dois caixas
eletrônicos. Os caixas presenciais são ``terceirizados'' e funcionam em
algumas lojas do comércio local. O gerente do Banco é pessoa muito ativa
que gosta muito do que faz e do lugar onde trabalha. Parece conhecer
todos os comerciantes e empreendedores locais pessoalmente. Segundo ele,
na comunidade tem cerca de quinhentos empreendimentos. Conta que, junto
com o empréstimo, oferece informalmente os serviços de consultor
financeiro: os comerciantes trazem seus livros de caixas e o gerente
ajuda na organização das contas, na projeção das despesas e dos
investimentos. Depois de menos de dois meses, já tinha conseguido abrir
cerca de mil contas individuais e cem de razão social. A política da
agência é não oferecer talão de cheque aos clientes, e sim apenas cartão
de crédito com teto baixo, ``para as pessoas irem se acostumando
devagarzinho'', diz o gerente. Mas muitas contas nem são movimentadas
(``muita gente abre só para abrir''). ``Integrar"-se'' à cidade é
tornar"-se correntista. Para abrir conta não é preciso trazer o
comprovante de residência nem de renda. O pessoal passa na associação de
moradores, que assina uma carta de garantia. Flexibilização dos
serviços, adequação à demanda. Os empréstimos concedidos são baixos (ao
redor de mil reais inicialmente), para ``educar'' os empreendedores a
lidar com esse tipo de situação. O aprendizado do manejo com o sistema
financeiro requer tempo, pedagogia e paciência. O valor dos empréstimos
é baixo também para que a agência não fique em vermelho. Porque,
descobrimos, uma agência de um banco (privado) funciona como uma filial
de uma franchising, ou seja, é o gerente que tem que encontrar os
recursos para reformar o local, alugar as máquinas dos caixas
eletrônicos junto à sede central, equacionar as contas com as folhas de
pagamento dos funcionários. Os lucros devem servir para financiar as
atividades, como qualquer outro empreendimento. Isso dá também uma certa
liberdade, diz o gerente, que por exemplo resolveu contratar
exclusivamente os serviços da mão de obra local para reformar a casa
sede da agência. Mas, como todo empreendimento, esse também comporta
riscos, e isso justifica a contenção de despesas com mão de obra para os
caixas presenciais. A ``descentralização'' desses é sem dúvida um
recurso importante nesse sentido. É o espírito do capitalismo moderno,
reduzir os custos da mão de obra, flexibilizar as formas de trabalho,
adequar o empreendimento às demandas locais. Mesmo assim, depois de um
tempo de euforia, a agência foi obrigada a colocar um freio à concessão
de empréstimos, por causa dos altos índices de inadimplência. (\versal{TOMMASI}
\& \versal{VELAZCO}, 2013: 32)
\end{quote}

O quadro descrito pelas pesquisadoras é bastante completo. Temos desde
as boas e velhas presenças do poder público em ``inaugurações'' até
modalidades inéditas de terceirização de serviços bancários para as mãos
de comerciantes locais, passando pelo fetiche da conta no Banco como
status de integração à ordem e, claro, por práticas aproveitadoras de
superexploração da mão de obra local, a qual certamente não exigirá do
Banco os direitos de auxílio"-transporte e, possivelmente, estará mais
propensa a aceitar salários abaixo da média, uma vez empolgada com um
trabalho que beneficiará ``a comunidade'' como um todo etc.

Diante do ``freio'' e da cautela do Bradesco em seguir com os
empréstimos, não há motivo para alarde, pois já há a concorrência a
beneficiar os moradores: outro programa de microcrédito, o \emph{Fundo
\versal{UPP} Empreendedor.} Operado por uma empresa privada por conta do governo
do Estado no âmbito da Agência Investe Rio, o Fundo compete com o
Bradesco no oferecimento de créditos, especialmente para a população
jovem, a taxas de juros bastante atraentes, somente 3\% ao ano e com
exigências mínimas para a concessão. A justificativa de existência do
Programa de Microcrédito não poderia ser mais nobre: visa oferecer
``oportunidades'' para jovens que poderiam ``se perder'' no envolvimento
com o tráfico de drogas. Aliado às linhas de crédito, há a oferta de
inúmeros cursos e cursinhos de curta duração que, segundo Tommasi e
Velazco, têm eficácia ``em termos de profissionalização''
``evidentemente duvidosa'', muito embora, e essa informação é
valiosíssima: essa ineficácia ``geralmente, não está na pauta dos
gestores, muito mais preocupados com `ocupar o tempo ocioso' dos
jovens'' (ibid: 33). Não poderia estar mais claro o intuito contra
insurgente e de garantia da manutenção da ordem de tais políticas.

Os projetos apoiados pelo governo e empresas são, via de regra, projetos
que visam ``ativar positivamente a população jovem, enquanto sujeito e
objeto de múltiplas formas de intervenção que visam fomentar,
fortalecer, ampliar suas capacidades ``empreendedoras'' nos mais
diferentes campos: cultural, social, econômico'' (ibid: 34). Um exemplo
de Projeto que deixa muito claro suas intenções políticas é o do
importante empreendedor cultural e social Marcos Faustini, a ``Agência
de Redes para a Juventude'', que conta com financiamento da Petrobrás e
apoio do governo. Conforme Tommasi e Velazco explicam:

\begin{quote}
A iniciativa consiste na formação de jovens para que eles elaborem um
projeto social de intervenção para ``melhoria das condições de vida na
comunidade''. Aqui, os jovens da comunidade são formados durante quatro
meses e no final devem elaborar um projeto; um dos quais, depois de
passar pelo crivo da avaliação de uma comissão de ``notáveis'' (muitos
dos quais são empresários), receberá um financiamento de 10 mil reais
para que possa ser realizado. Os educadores e ``mediadores culturais''
do projeto são todos jovens de ``comunidades'', jovens que já tiveram
uma trajetória em projetos sociais e se destacaram pelas habilidades
adquiridas. Jovens que, evidentemente, recebem salários ``adequados'' à
sua condição de jovens. A ideia do projeto é ``despertar o sentimento de
pertencimento à comunidade'' e ao mesmo tempo, promover a circulação na
cidade. A primeira pergunta à qual os jovens precisam responder na
preparação de seus projetos é se eles têm algum sonho. A segunda, é
sobre valores: que valores o projeto vai ajudar a difundir? A intenção é
clara: um bom empreendedor é o que persegue seus sonhos e ajuda a
difundir na comunidade os ``bons'' valores (de cidadão e empreendedor?
Ou cidadão"-empreendedor). (ibid: 35)
\end{quote}

A chegada da \versal{UPP} trouxe para os moradores, além de Bancos e contas de
luz, a imposição da regularização do empreendimento. Com a
\emph{pacificação}, os comerciantes locais se viram subitamente forçados
a abrir um \versal{CNPJ}, a entender como funciona a burocracia da prefeitura, a
declarar o imposto de renda e a temer a possível chegada dos
``fiscais''. A recusa das regras da legalidade muitas vezes é vista como
um verdadeiro ato de resistência: ``vou assinar a carteira do meu
marido, ou de minha cunhada? Não faz sentido!'' diz uma das operadoras
de caixa descentralizado do Bradesco, entrevistadas por Tommasi e
Velazco.

Quanto à legalização dos empreendimentos e a disseminação da
racionalidade neoliberal cabe destacar a atuação do Sebrae, que ajuda
nos aspectos burocráticos e oferece cursos de formação, por exemplo o
``Aprender a empreender serviços'' e a ``Oficina de finanças'', voltados
a empreendedores e comerciantes que queiram se familiarizar com as
exigências legais decorrentes de sua nova condição.

O Sebrae incentiva, ainda, a criação de uma associação de comerciantes
locais visando impedir a entrada das grandes cadeias de lojas de
departamentos, como as Casas Bahia, cujos preços competitivos
evidentemente prejudicariam as vendas do comércio local. Tommasi e
Velazco explicam que os cursos oferecidos pelo Sebrae são cursos
específicos para Empreendedores Individuais (\versal{EI}), uma figura jurídica
criada pelo governo federal para facilitar a legalização de algumas
categorias de trabalhadores informais através da redução dos impostos.
``Para convencer os empreendedores a se legalizar, os representantes do
Sebrae fazem apelo a dois atrativos: o desenvolvimento e a possibilidade
de utilizar o cartão de crédito'' (ibid: 29). À miríade de identidades
soma"-se, então, mais uma: a de empreendedor.

Os técnicos do Sebrae informam que um conjunto de dificuldades tem
inviabilizado seu trabalho de legalização dos empreendimentos, e afirmam
que se tornou comum, nas comunidades, a utilização da figura do \versal{EI}
(Empreendedor Individual) para driblar os direitos trabalhistas: donos
de restaurantes e cabeleireiros obrigam seus empregados a se tornar \versal{EI},
para não ter que assinar suas carteiras de trabalho e pagar os impostos
devidos. Pesquisadores do Ipea afirmam que a manobra vem sendo usada
especialmente no setor da construção civil, que já conta com alto índice
de trabalho informal.

Aprofundando a interiorização da racionalidade neoliberal no âmago da
classe trabalhadora que reside nas comunidades, as ``oportunidades'' de
ascensão pessoal são promovidas por diversas empresas transnacionais,
articuladas com o governo e \versal{ONG}s, numa confluência de interesses jamais
vista, o que tem levado à exportação do modelo brasileiro de gestão
lucrativa da pobreza. Veja"-se a riqueza de elementos que a narrativa a
seguir nos traz:

\begin{quote}
Numa outra lateral, já perto do cruzamento central, abriu em 2011 uma
pequena loja, toda reformada, de sabão e sabonete, loja na frente e
espaço para produção atrás, um empreendimento bem no espírito do moderno
``combate à pobreza'': é um grupo de mulheres egresso de um curso sobre
empreendedorismo e microcrédito, oferecido por uma \versal{ONG} de mulheres da
Zona Sul da cidade em quatro favelas, curso que resultou no
financiamento de um empreendimento em cada favela; curso e
empreendimentos financiados com recursos (meio milhão de dólares) de uma
multinacional americana (a Chevron), recursos transacionados por uma
fundação privada americana, a Fundação Kellogg, muito atuante no Brasil
na área da responsabilidade social empresarial. O sabão comum é feito
com óleo de cozinha reciclado, que as mulheres recolhem na comunidade.
Por enquanto, estão ``tirando'', como elas dizem, cerca de trezentos
reais por mês. No curso, segundo elas, tudo era decidido ``por
consenso''; mas de fato das 38 participantes, só seis se juntaram para
criar o grupo. A seleção se deu de forma ``natural'': o curso tinha que
ser frequentado todos os dias, inclusive aos sábados o dia inteiro. E
justamente por ter permanecido o curso inteiro, persistindo e gerando o
novo negócio, essas mulheres se veem como um pequeno grupo vitorioso,
que com força de vontade e determinação conseguiu o que queria. Esse é o
ideal constantemente enfatizado: ``O projeto me fez colocar os pés no
chão, erguer a minha cabeça e falei para mim mesma que venceria, ia para
a luta nessa oportunidade única da minha vida'', relata uma das
integrantes numa matéria feita em ocasião da inauguração da loja e
publicada no blog da ``pacificação''. O dinheiro para abrir o
empreendimento (50 mil reais financiados pela Chevron) é administrado
pela \versal{ONG}, que o ``libera'' gradualmente e acompanhará o grupo até sentir
que ``a gente está apta para ganhar o mercado sozinhas'', dizem. Quando
pergunto o que é preciso fazer para virar empreendedoras, respondem: ``é
preciso ter coragem, não adianta só querer''. Uma delas tem o sonho de
ganhar, algum dia, 5 mil reais, outra 2 mil reais. Coragem, ousadia,
confiança: ingredientes chaves do ``espírito empreendedor''. Mas,
recentemente o número das mulheres integrantes do grupo diminuiu, por
causa das dificuldades financeiras que o empreendimento enfrentou depois
que o financiamento da Chevron acabou; visivelmente, não é um exemplo de
sucesso. {[}\ldots{}{]} Entre outras coisas, as mulheres me contam dos roubos
que estão acontecendo na comunidade, na rua, nas casas, e que antes não
aconteciam. ``Agora você não pode nem deixar uma bicicleta no meio da
rua''. ``Efeito perverso'' da ``pacificação''. Como outros moradores,
elas asseguram que antes da \versal{UPP} viviam relativamente tranquilas na
comunidade, ``se você não devia nada a ninguém, se ficava na tua,
ninguém implicava com a gente''. Além das lojas, na rua principal
encontramos também algumas ofertas de serviços, como a venda dos planos
da Sky. A regularização do acesso aos canais fechados de \versal{TV} é um dos
primeiros acontecimentos nos territórios ``pacificados''; aliás, como me
contou um gestor de outro território, os moradores dizem que ``a Net
sobe o morro já no carro da polícia de ocupação''. Mas a proposta de
planos aparentemente baratos (cinquenta reais) esconde a armadilha: o
plano permite o acesso a muito poucos canais, enquanto os planos
``ilegais'' ofereciam, pelo mesmo preço, acesso a todos os canais
fechados. Numa lona improvisada na calçada, a Honda oferece planos para
parcelar a compra de uma moto. Numa outra, a \versal{TIM} oferece seus novos
serviços de telefone fixo pré"-pago. As ``melhorias'', ou seja a obra
Bairro Maravilha Cidade de Deus, já foram ``lançadas'' três vezes pelo
prefeito. Outro dia são os representantes dos serviços judiciários, numa
ação da Casa dos Direitos Itinerante, com a presença do ministro da
Justiça: uma feira de serviços jurídicos ambulantes, onde os cidadãos
podem fazer denúncias, emitir carteira de trabalho, utilizar os serviços
de cartórios etc. Depois, é a empresa de venda em domicílio Natura que
monta alguns estandes para ``juntar todas as forças sociais da
comunidade'' (ou seja, a própria Natura, duas ou três \versal{ONG}s locais que
vendem artesanato, e os policiais da \versal{UPP}, que distribuem ``santinhos''
para alertar a população sobre a necessidade de preservar o meio
ambiente) e dar publicidade a seus produtos. Nas barracas da Natura é
possível fazer uma maravilhosa massagem nas mãos ou uma linda
maquilagem, ou ouvir a música tocada pela banda dos policiais da \versal{UPP},
motivando as mulheres a trabalhar como vendedoras. Outra vez, é dia da
Feira de Economia Solidária. Em seguida, vem a gestão do Rock in Rio
apresentar seu programa social. Todos esses eventos acontecem com a
presença ostensiva e fortemente armada dos policiais da \versal{UPP}, que vigiam
o bom andamento das comemorações. É o ``tempo do evento'' (parafraseando
a feliz expressão de Moacir Palmeira e Beatriz Heredia, o ``tempo da
política'') no qual aparecem personagens, práticas, relações distintas
respeito ao cotidiano que os moradores habitam; tempo durante o qual as
adesões tornam"-se manifestas e as alianças são sacramentadas
(especialmente, no nosso caso, entre os que estão no palco). (\versal{TOMMASI} \&
\versal{VELAZCO}, 2013: 28-30)
\end{quote}

Sintomaticamente, as pesquisadoras descrevem um dia em que havia na
Cidade de Deus, simultaneamente, um Evento da \emph{Natura} e um Comício
do Ministro da Justiça, com preferência clara para o primeiro evento e
manifestações de descrédito da população ao passar ao lado do Comício:
``queremos não só carteira de trabalho: queremos carteira de trabalho
assinada!'', teria dito um transeunte (ibid: 30).

Tal como no caso da linha ``Quem disse, Berenice?'', da empresa \emph{O
Boticário}, o sucesso da \emph{Natura} toca sutilmente em elementos da
identidade do morador favelado: provém da conjugação do caráter mais ou
menos ``de luxo'' de suas mercadorias, que são voltadas para o cuidado
estético, e seu método de emprego, o qual garante certo status às
vendedoras a domicílio (o percentual de vendedores homens é mínimo). A
questão causa, de certo modo, indignação, já que as vendedoras da
\emph{Natura} não trabalham pelo sistema de consignação de venda, e sim
compram os produtos na empresa com 20\% de desconto e os revendem por
conta própria, restando às trabalhadoras, portanto, ganhos mínimos e uma
insegurança inerente ao modelo de venda, além, é claro, da ausência de
direitos trabalhistas.\footnote{Cf. Uma radiografia deste setor é feita
  por Ludmila Costhek Abílio (2014).}

O fato dos produtos da \emph{Natura} serem caros e ainda assim bastante
consumidos pela população de baixa renda é apontado por Tommasi como
mais um sinal da ``integração'' dos pobres pela via do consumo. Mas o
que já está ruim pode ficar pior: nas proximidades das favelas
pacificadas há contingentes enormes de moradores ``invisíveis'', que não
aparecem nas estatísticas do \versal{IBGE} e nem nos mapas da Secretaria de
Saúde: são moradores excluídos das iniciativas voltadas a incrementar o
consumo e o ``espírito empreendedor''.

\begin{quote}
Aqui, as cores da pintura não são vivas; ao contrário, é a escala de
cinza que prevalece. Desolação, abandono, muito lixo, muitas crianças
brincando com nada. Para eles, não tem academia nem quadra, só alguns
pneus velhos amontoados. Imagens fortes que fazem surgir novas
interrogações: quem pode se beneficiar desse modelo de ``integração''
via empreendedorismo e consumo? Qual é o destino reservado a quem fica
de fora? Como se determinam as clivagens? As fronteiras deslocam"-se, mas
não estão borradas. (ibid: 36)
\end{quote}

Quando os métodos de gestão da pobreza são o que há de mais ``humano''
em determinado território, não há outra palavra a se pronunciar senão
\emph{barbárie}. A situação a que está submetida a classe trabalhadora
das periferias das grandes cidades (mas não só) e seu manejo pelo poder
público e privado são um terrível pesadelo tornado real em um tempo de
vitória da contrarrevolução e de predomínio, no imaginário dos
trabalhadores, de ``expectativas decrescentes'' (Paulo Arantes). Como
bem pontuam as pesquisadoras Tommasi e Velazco, o que está acontecendo
nos territórios pacificados e periferias das grandes cidades ``não é
circunscrito e especifico a esses espaços'', e sim ``diz respeito a
todos nós, enquanto remete à forma como o governo neoliberal se
manifesta na atualidade'' e à forma como se concretiza, em práticas
sociais, aquilo que chamamos de ``cidadania''.

\begin{quote}
Referimo"-nos, em particular, aos incentivos ao chamado
``empreendedorismo'', que encontramos, na atualidade, em diferentes
âmbitos da vida econômica, social e cultural: dos programas de apoio às
micro e pequenas empresas, aos projetos sociais com jovens; das
atividades de ``responsabilidade social'' à produção cultural. (ibid:
39)
\end{quote}

O que temos é a ``conformação da subjetividade empreendedora como uma
estratégia de poder'':

\begin{quote}
O vocabulário do empreendedorismo une a retórica política e os programas
regulatórios às capacidades de ``autodireção'' das pessoas. {[}\ldots{}{]}
Refere"-se a uma série de regras para a conduta da existência diária de
uma pessoa: energia, iniciativa, ambição, cálculo e responsabilidade
pessoal. O self empreendedor fará da sua vida um empreendimento,
procurando maximizar seu próprio capital humano, projetando seu futuro e
buscando se moldar a fim de se tornar aquilo que deseja ser. {[}\ldots{}{]} O
empreendedorismo designa uma forma de governo que é intrinsecamente
``ética'': o bom governo deve ser baseado nas maneiras pelas quais as
pessoas governam a si próprias. (Nikolas Rose apud \versal{TOMMASI} \& \versal{VELAZCO},
2013)
\end{quote}

``Os indivíduos contemporâneos são incitados a viver como se fossem
projetos, a tornar"-se, cada um, um empresário de si mesmo'' (\versal{TOMMASI} \&
\versal{VELAZCO}, 2014: 40). As políticas estatais, e em particular as de gestão
da pobreza, se inserem em uma determinada lógica de organização social,
política e econômica centrada no fortalecimento e \emph{valorização do
indivíduo}, a qual se mostra de modo claro quando entendemos o
neoliberalismo como racionalidade onde há o ``empreender a vida como uma
escolha ativa'' em paralelo com uma ``de"-socialização da gestão
econômica'' (Rose, op.\,cit). Trata"-se de uma estratégia de governo que
tem como ``sujeito e objeto a população'', ou seja, um governo que se
realiza ``não somente sobre, mas também através da população'', ou seja,
uma forma de ``governo de si'' que não se limita a si e se expande para
o governo dos outros: governar significa ``agir de maneira a estruturar
o campo de ação possível dos outros'' (op.\,cit). A partir da chave
denominada ``liberal avançado'', praticamente idêntica à do
``neoliberalismo'' de que falam Dardot e Laval, Rose afirma o caráter
inovador das estratégias de governo que se desenvolvem na última década
do século \versal{XX}:\looseness=-1

\begin{quote}
Ao invés de governar o social em nome da economia nacional, se governam
zonas particulares --- regiões, cidades, setores, comunidades --- em vista
do interesse dos circuitos econômicos que correm entre as regiões e as
fronteiras nacionais. Os destinos econômicos dos cidadãos no interior de
um território nacional estão desatrelados, e agora estão compreendidos e
governados como uma função de seus particulares níveis de
empreendimento, habilidade, criatividade e flexibilidade. {[}\ldots{}{]} Essa
ênfase sobre o indivíduo como um agente ativo no governo de sua própria
economia através da capitalização de sua própria existência é paralelo a
uma série de novos vocabulários e conjuntos de dispositivos implantados
para gerir os indivíduos no interior dos postos de trabalho em termos de
reforço de suas próprias competências, capacidades e espírito
empreendedor. {[}\ldots{}{]} O trabalho, também, não é mais considerado como
uma obrigação social, nem sua eficiência deve ser reforçada através da
maximização dos benefícios sociais que o trabalhador encontra no local
de trabalho, nem o seu principal papel deve ser o de circunscrever o
indivíduo na coletividade através dos efeitos de socialização dos
hábitos de trabalho. Pelo contrário, o trabalho em si --- para os
trabalhadores e para os managers --- torna"-se um \emph{espaço de
autopromoção} e a gestão do trabalho é realizada em termos de
\emph{reforço das capacidades ativas dos empreendedores individuais}.
{[}\ldots{}{]} A gestão econômica está sendo de"-socializada em nome da
maximização do comportamento empreendedor de cada indivíduo. (Nikolas
Rose apud \versal{TOMMASI} \& \versal{VELAZCO}, 2013: 40)
\end{quote}

A partir da ideia de que os ``incluídos'' do sistema são aqueles
indivíduos que ``detêm os recursos financeiros, educacionais e morais
para assumir o papel de cidadãos ativos em comunidades responsáveis'',
Rose pontua que essa nova gestão econômica provoca uma transformação nas
formas de governo das condutas: ``empreender a vida, no âmbito de todas
as práticas quotidianas, como uma escolha ativa'', de modo que ``a
inclusão se faz através do fortalecimento da escolha, da autonomia e do
consumo''.

Como bem lembra Tommasi e Velazco, se nas décadas passadas a inclusão
era concebida como resultado da obtenção de um trabalho assalariado,
hoje é ``a figura do empreendedor que se coloca como modelo e
possibilidade de inclusão''. Como já havia sido notado por Dardot e
Laval, é o fantasma de Schumpeter que assombra o contemporâneo, na
medida em que para o economista a evolução econômica se dá por meio de
rupturas e descontinuidades parametradas por uma ``destruição criadora''
operada por indivíduos empreendedores, ou seja, criativos, com a ousadia
de inovar.

\begin{quote}
Como escrevem Pierre Dardot e Christian Laval, que estudam o
neoliberalismo enquanto sistema de normas que orientam as práticas de
governo, hoje a concorrência não se faz somente através dos preços, e
sim através da inovação, operada por sujeitos que interiorizam a pressão
concorrencial ``de forma a torná"-la a norma da subjetividade''. Assim,
``são todas as atividades humanas, até as mais distantes do mercado
mundial, que precisam funcionar de forma homogênea segundo a lógica da
concorrência'' (ibid: 46).
\end{quote}

Não é por acaso que todo esse panorama crítico encontra correspondência
no exemplo carioca das implantações das \versal{UPP}s e todo o Projeto de
políticas sociais e econômicas aí envolvidas, as quais têm como pano de
fundo a relação íntima entre, de um lado, a valorização do indivíduo
inerente ao neoliberalismo e ao identitarismo e, de outro, as múltiplas
estratégias empresariais e estatais voltadas para o desenvolvimento
capitalista.

A ``pacificação'' no Rio de Janeiro tornou"-se a ocasião perfeita para a
instalação de dispositivos de governo que visam ampliar o mercado
consumidor interno e promover, no próprio âmbito da classe trabalhadora,
o ``espírito empreendedor''. Se por um lado os moradores tornam"-se
``cidadãos'' pela via do consumo, por outro essa suposta ``inclusão''
pressupõe que eles virem gerentes de algum empreendimento, nem que tais
empreendimentos sejam eles próprios. Como a entrada na legalidade e a
suposta ``conquista da cidadania'' vêm com certo \emph{preço}, os
moradores se defrontam \emph{de modo ambivalente} com o \emph{projeto
estatal"-empresarial,} em uma oscilação de aceitação e resistência,
proximidade e recusa.

Se para os moradores o Programa da Pacificação aparece como recheado de
contradições e paradoxos, para as classes dominantes a implantação das
\versal{UPP}s no Rio de Janeiro teve resultados econômicos imediatos (e mesmo
prévios). No campo da especulação imobiliária, por exemplo, viu"-se
significativa valorização tanto dos imóveis situados nas favelas
``pacificadas'' ou a serem ``pacificadas'' quanto naqueles do
``asfalto'' em torno delas. Na Comunidade Cidade de Deus, por exemplo,
houve uma elevação de até 400\% no preço dos imóveis.

Como se vê, as políticas são implementadas ora por empresas, ora por
instituições estatais, ora por \versal{ONG}s, ora por entidades da sociedade
civil e até mesmo por movimentos sociais. De fato temos uma fusão de
atividades, o que corrobora o modelo analítico de João Bernardo (2009),
que coloca Estado Amplo (empresas) e Estado Restrito (Estado Nacional)
como sendo ambos ``Estado''. Neste modelo teórico os ``gestores'' são
vistos como a principal classe capitalista, que atua tanto em espaços
estatais quanto empresariais. Tratando do tema das \versal{UPP}s, Lívia de
Tommasi observou que ``Setores públicos e privados, autóctones e
estrangeiros, atividades econômicas, sociais, culturais, de segurança,
educativas, políticas, nos parecem estar completamente imbricados.
Separar (o que é Estado do que é sociedade civil, o que é organizado do
que não, privado do público, legal e ilegal) não somente empobrece a
análise como é parte da lógica gestionária que precisa ser interrogada.
Curiosamente, inclusive, observamos uma ``dança dos papéis'': policiais
que realizam atividades de educadores ou animadores sociais, oferecendo
atividades esportivas, recreativas e de reforço escolar às crianças;
gerentes de banco que funcionam como conselheiros de negócios e
empreendimentos; comerciantes que viram caixa de banco; líderes
comunitários que gerenciam programas de governo; gestores públicos que
transacionam empreendimentos privados'' (\versal{TOMMASI}, 2014: 19).

A estratégia estatal e capitalista de implantação das \versal{UPP}s dá um curto
fôlego econômico para algumas empresas envolvidas no processo,
permitindo"-lhes auferir lucros de espaços até então não capitalizados,
que agora passam a estar inseridos nos esquemas de acumulação por
espoliação (Harvey). Mas esses lucros se dão ao preço do deslocamento
temporário de algumas contradições urbanas e raciais há séculos não
resolvidas e, por isso, alimenta a semente do conflito social em níveis
mais agudos, num futuro próximo.

Não por acaso tivemos as revoltas sociais de meados de 2013, quando
houve intensa participação de pessoas que residem nessas favelas
``pacificadas'' e que demonstraram nas ruas que o \emph{grau de
insuportabilidade} da vida social nas grandes cidades atingiu o teto.

As \versal{UPP}s, além de não resolverem o problema urbano de moradia, levaram a
uma expulsão dos moradores mais pobres, uma vez que com a
``pacificação'' os aluguéis e o custo de vida em geral subiram muito nas
favelas, levando a um ``branqueamento'' das comunidades conforme mudaram
para a periferia setores das classes trabalhadoras e estudantes que até
então viviam no ``asfalto'' (onde passaram a viger valores exorbitantes
de aluguel e custo de vida). Não é por acaso, então, que Felipe Brito
conclui que as \versal{UPP}s tendem a provocar ``novas rodadas de distensão da
conflitualidade socioespacial, ampliando as fronteiras da favelização,
de maneira a reatualizar o vaivém histórico das `resoluções não
resolvidas' dos nossos problemas sociais'' (ibid: 105). Tanto quanto as
contradições, os interesses estatais e os montantes de capital
envolvidos no processo são grandes:

\begin{quote}
A relação das \versal{UPP}s com o mercado é mais extensa e profunda. Mediante uma
heterodoxa parceria público"-privada, um pool formado por Coca"-Cola,
Souza Cruz, Light, Metrô, Bradesco e outras empresas comprometeu"-se a
criar um fundo destinado às \versal{UPP}s como reconhecimento às garantias e
salvaguardas que estas fornecem e fornecerão aos grandes investimentos.
{[}\ldots{}{]} Além desse pool, a Confederação Brasileira de Futebol (\versal{CBF})
também prometeu doar recursos ao fundo. A Bradesco Seguros, A Coca"-Cola
e a Souza Cruz comprometeram"-se, respectivamente, com R\$ 2 milhões, R\$
900 mil e R\$ 400 mil. Contudo, a parceria não se restringe à criação de
um fundo: na Ladeira dos Tabajaras, a Souza Cruz e a Coca"-Cola estão
construindo a sede de uma \versal{UPP}. A fabricante de cigarros também doou um
terreno em Manguinhos para a Construção da Cidade da Polícia, local que
concentrará todas as sedes de delegacias especializadas do Rio de
Janeiro. A \versal{CBF}, por seu turno, está participando da construção da \versal{UPP} na
Cidade de Deus. No fim de outubro de 2011, Eike Batista reforçou a
intenção de comprar a refinaria de Manguinhos (que, além da localização
estratégica, obteve recentemente licenciamento ambiental), mas
condicionou a compra à instalação de uma \versal{UPP} na região. O fato é que,
além de investimentos destinados à Copa e às Olimpíadas na ordem de R\$
55 bilhões, o Estado do Rio de Janeiro deverá receber cerca de R\$ 181,4
bilhões em investimentos entre 2011 e 2013. Por isso, o grande capital
tem fortes expectativas em relação à atuação das \versal{UPP}s. Nesse sentido,
não foi fortuita a instalação da transnacional Procter \& Gamble na
Cidade de Deus há cerca de dois anos, mediante reduções e isenções
fiscais. (\versal{BRITO}, 2013: 105 e 107)
\end{quote}

O Estado nacional procura constantemente articular"-se com as empresas
\emph{multi} e transnacionais de modo a garantir a eficácia dos
mecanismos da mais"-valia relativa. Essa conjunção de esforços visa
restringir as rebeliões dos trabalhadores ao âmbito reivindicativo
reformista, ou seja, a demandas que o sistema possa absorver
lucrativamente. O método de tal contenção das lutas é o solapar da
solidariedade nos locais de trabalho e moradia, onde atuam não apenas os
mecanismos próprios da disciplina capitalista nas empresas privadas, mas
também atua fortemente o Estado, referendando a introdução de novos
métodos para calcular e pagar os salários, promovendo a rivalidade entre
trabalhadores nacionais e imigrantes, pela promulgação de políticas
salariais ou ``contratos sociais'', pelas políticas monetárias e demais
veículos da política econômica do governo, que incidem diretamente no
poder de compra dos salários e na desvalorização da força de trabalho,
etc.

Nesse processo de promoção da rivalidade entre trabalhadores há a forte
presença do identitarismo enquanto elemento de fragmentação da classe.

A administração das crises via articulação entre Estado e empresas
cumpre, portanto, a função dupla de, no âmbito \emph{político}, evitar
as ameaças potencialmente revolucionárias ao sistema e, do ponto de
vista \emph{econômico}, fornecer saídas econômicas rentáveis para o
capital (estando ou não enfrentando uma crise). Muitas vezes, inclusive,
o processo de salvaguarda estatal das empresas se dá por meio da ajuda
financeira direta e até mesmo pela aquisição das empresas em falência,
como vimos na última manifestação mais visível de aguçamento da crise
capitalista, nos \versal{EUA} em 2008-2009. Tanto num caso quanto no outro o
sucesso ou fracasso da missão conservadora do Estado é mútuo, ou seja, o
Estado precisa ter sucesso em suas frentes de atuação para que o sistema
capitalista não se veja em apuros e tendo de lidar com situações de
colapso e potencialmente ``pré"-revolucionárias''. Nessas tarefas, ele
encontrou na assimilação das pautas identitárias um poderoso aliado, por
isso o identitarismo pode ser visto, do ponto de vista da esquerda
anticapitalista, como um inimigo interno.

\chapter*{Modernizam as empresas e\break arcaízam os trabalhadores?}
\addcontentsline{toc}{chapter}{Modernizam as empresas e arcaízam os trabalhadores?
\medskip}
\hedramarkboth{Modernizam as empresas e arcaízam os trabalhadores?}{}

Além de contribuir com a dinamização e legitimação do capitalismo, a
política identitária também fortalece o sistema ao contribuir para
enraizar, ou até para ressuscitar, diferenciações biologizantes (de
cunho racista) e tradições culturais que servem apenas para fragmentar
ainda mais a classe trabalhadora. Aqui reside o nó górdio que articula
as implicações políticas e econômicas das políticas identitárias. O
racismo e machismo são problemas reais que afligem os trabalhadores, por
isso trata"-se de enfrentá"-los no campo das lutas sociais imediatas. Em
contextos de crise econômica e acirramento das competições entre
trabalhadores as alternativas identitárias são ainda mais atrativas
àqueles que podem se beneficiar delas no curto prazo, que é, em geral, o
\emph{tempo histórico} que mais importa a sujeitos individualmente
considerados.

Por um lado, o primado das lutas imediatas constitui um ponto de
consenso tácito para a maioria do campo político de esquerda, em
especial aquele ciente dos danos causados pela prática
stalinista segundo a qual as opressões seriam enfrentadas
\emph{apenas depois} da revolução contra o capitalismo. Por
outro lado, as formas de enfrentamento à opressão
\emph{descoladas} da perspectiva de enfrentamento das relações sociais
de exploração resultam não apenas na manutenção do capitalismo, como
ainda permitem que o sistema se dinamize e se enraíze em bases políticas
e econômicas mais sólidas.

Por isso a extrema esquerda se vê diante da constrangedora situação na
qual as velhas ortodoxias não conseguem responder satisfatoriamente às
demandas identitárias dos trabalhadores e as iniciativas mais
sistemáticas de enfrentamento das opressões, postas efetivamente em
prática, se mostram perniciosas em termos de luta contra o capitalismo,
afinal resultam no reforço daquilo que outros pretendem superar.
Constatar isso nos mostra o quanto é importante a crítica ao
identitarismo e suas implicações nefastas no âmbito das demandas e
formas de luta, como por exemplo aquelas que caracterizam algumas
vertentes do feminismo enquanto \emph{feminismo excludente} e as
correlatas variedades de sectarismo no campo da luta contra o racismo e
a homofobia, mas o quê colocar no lugar destas modalidades de luta? Como
atender às demandas por diversidade sem que a mobilização em torno de
tais bandeiras e o atendimento a tais pautas terminem por reforçar o
sistema que se pretende negar?

A dinâmica do capitalismo é capaz de recuperar qualquer demanda
ou forma de luta, exceto aquela que contesta --- ou que simultaneamente
contesta --- o fundamento da mais"-valia, ou seja, a relação social de
exploração do trabalho. As novas
modalidades de sociabilidade autogestionária, quando ensaiadas pela classe trabalhadora, não são assimiláveis pelo capital por conta de sua organização
completamente distinta daquela que formata as relações capitalistas.
Entretanto, se não se generalizam por todo o tecido social, mesmo esses
ensaios e experiências históricas acabam sendo assimilados, no longo
prazo, uma vez que o sistema a eles contraposto não é substituído em
escala global.

O capitalismo possui uma tendência centrífuga que resulta na absorção e
englobamento de todo e qualquer experiência ou espaço organizado sob
outra lógica que não esteja em acordo com a lei do valor. Essas
experiências são desnaturadas e assimiladas, passando a renovar a
próprio sistema de relações sociais capitalistas ao qual inicialmente
estavam contrapostas. Ter em mente este e outros limites expostos ao
longo deste livro não implica secundarizar as lutas contra as opressões
em prol das lutas classistas, mas tão somente desvendar alguns dos
mecanismos assimilatórios operantes e indicar as \emph{condições} para
que as lutas feministas, as lutas raciais e étnicas e as lutas \versal{LGBTIQ}+
deixem de ser assimiladas resultando no fortalecimento do capitalismo.

Ademais, a própria formulação do problema em termos\,de ``opressão''
e ``classe'' em ``lados'' distintos gera confusão, pois as opressões
existentes no capitalismo resultam da exploração, na medida em que cada
tipo ou modalidade específica de exploração do trabalho de pessoas
corporifica"-se, no campo político, em formas também específicas de
opressão. Isso necessariamente ocorre devido ao fato de que as
sociedades assentadas em regimes de exploração se apresentam como
organismos políticos necessariamente complexos --- do contrário as
relações sociais de exploração apareceriam de modo simplório, por assim
dizer \emph{cru} e, por isso, uma vez expostas as suas contradições e
antagonismos estruturais, sua organização hierárquica e seus polos de
dominação seriam mais facilmente contestados e desmantelados pelos
explorados.

As estruturas e relações sociais de opressão, portanto, tanto ocultam
quanto reforçam o elemento \emph{econômico} da organização social. Nesse
sentido, para os anticapitalistas a luta contra as opressões e
discriminações só deve assumir a forma de uma luta no interior da classe
trabalhadora a fim de educar ou forçar os trabalhadores racistas,
machistas e homofóbicos a deixarem de sê"-lo.

Isso deve ocorrer não por conta de qualquer dívida histórica, privilégio
ou qualquer outra querela da gramática identitária, mas tão somente
porque a sociedade igualitária que se almeja construir deve ser
prefigurada \emph{desde as lutas} em relações sociais que estejam em
conformidade com os novos valores que se busca impor: o caráter
revolucionário de uma luta social emerge quando estamos diante de uma
luta pela construção de relações sociais novas não mais racistas,
homofóbicas, machistas e exploradoras.

Se a luta no interior da classe trabalhadora organizada deixa de ser
uma luta de solidariedade entre pares e assume roupagens conflituosas
que aprofundam a fragmentação da classe, trata"-se de um elemento
ideológico"-político intruso que deve ser combatido.\,É o caso da
política identitária, com\,o agravante de que esta se configura
não só como algo nocivo para a organização dos trabalhadores e
trabalhadoras, mas como uma teoria e prática assentada em valores
radicalmente opostos aos do anticapitalismo. De contrabando, o
identitarismo traz, para o bojo das franjas mais arrebentadas da classe,
os temas, valores e formas organizativas que historicamente embasaram as
teorias e práticas da extrema direita.

Entender o modo como o capitalismo recupera as conquistas das lutas por
diversidade não significa um antídoto à assimilação, mas ao saber quais
são, como funcionam e onde encontram limites as armas à disposição de
ambas as partes, podemos repensar nossas formas de luta e as teorias e
práticas que lhes dão sustentação ideológica e material. É nesse sentido
que este livro busca dar alguma contribuição.

\chapter{Considerações finais}

Buscamos demonstrar os modos pelos quais as empresas capitalistas
assimilam as pressões sociais decorrentes das lutas identitárias, em
especial as lutas feminista e negra. Por meio dos mecanismos de
mais"-valia relativa, de desenvolvimento da produtividade e concessões
estratégicas para os trabalhadores, as empresas se tornam aptas a
integrar as demandas das lutas contra as opressões de raça, gênero e
sexualidade. Ao assim proceder, reforçam suas próprias raízes políticas,
ideológicas e culturais nos locais onde atuam, estreitando os laços
políticos, ideológicos e econômicos entre patrões e trabalhadores. Com a
dinamização das elites empresariais, decorrente das pressões das lutas
identitárias, ganha novo fôlego o desenvolvimento capitalista. O livro
tratou, portanto, de alguns dos modos como o capitalismo converte a luta
contra o machismo, o racismo e a homofobia em algo lucrativo.

Em 2015 a \emph{McKinsey Global Institute} publicou um relatório chamado
``Diversity Matters'' onde analisava os dados de 366 empresas, de
diversos setores, que atuavam no Canadá, América Latina, Reino Unido e
Estados Unidos. Por meio da comparação métrica de resultados financeiros
e composição da alta administração e dos conselhos, o instituto concluía
o seguinte:

\begin{enumerate}
\def\labelenumi{\arabic{enumi}.}
\item
  Empresas no quartil superior para a diversidade racial e étnica são
  35\% mais propensas a ter retornos financeiros acima de suas
  respectivas medianas da indústria nacional.
\item
  Empresas no primeiro quartil para diversidade de gênero têm 15\% mais
  chances de obter retornos financeiros acima de suas respectivas
  medianas da indústria nacional.
\item
  Empresas no quartil inferior, tanto para gênero quanto para etnia e
  raça, têm estatisticamente menos probabilidade de obter retornos
  financeiros acima da média do que as empresas médias no conjunto de
  dados (ou seja, as empresas de quartil inferior estão ficando mais do
  que simplesmente não liderando).
\item
  Nos \versal{EUA}, há uma relação linear entre diversidade racial e étnica e
  melhor desempenho financeiro: para cada 10\% de aumento na diversidade
  racial e étnica na equipe executiva sênior, o lucro antes de juros e
  impostos (\versal{EBIT}) aumenta 0,8\%.
\item
  A diversidade racial e étnica tem um impacto mais forte sobre o
  desempenho financeiro nos Estados Unidos do que a diversidade de
  gênero, talvez porque os esforços anteriores para aumentar a
  representação das mulheres nos níveis mais altos de negócios já
  produziram resultados positivos.
\item
  No Reino Unido, a maior diversidade de gêneros na equipe executiva
  sênior correspondeu ao aumento de desempenho mais alto em nosso
  conjunto de dados: para cada aumento de 10\% na diversidade de
  gêneros, o \versal{EBIT} aumentou 3,5\%.
\item
  Embora certas indústrias tenham um melhor desempenho na diversidade de
  gênero e em outras indústrias quanto à diversidade étnica e racial,
  nenhuma indústria ou empresa está no quartil superior em ambas as
  dimensões.
\item
  O desempenho desigual de empresas no mesmo setor e no mesmo país
  implica que a diversidade é um diferencial competitivo que muda a
  participação de mercado para empresas mais diversificadas. (\versal{HUNT}, V.;
  \versal{LAYTON}, D. \& \versal{PRINCE}, 2015)
\end{enumerate}

As conclusões, baseadas numa quantidade enorme de dados, não deixavam
margem para dúvidas e, além disso, de modo precursor, apontavam um
impacto maior da diversidade racial e étnica em comparação com a de
gênero.

Realizando pesquisas voltadas à temática da diversidade desde 2007, a
\emph{McKinsey Global Institute} se tornaria um referencial obrigatório
para todos os interessados no tema, em especial as empresas, pois além
de diagnósticos e análises o Instituto passou a formular propostas de
ação e em 2018 enumeraria quais são os dez atributos de uma empresa ou
organização inclusiva:

\begin{enumerate}
\def\labelenumi{\arabic{enumi}.}
\item
  Não ortodoxa. Políticas, regras, normas e práticas são constantemente
  desafiadas para levar em conta as necessidades de todos, não apenas um
  grupo dominante.
\item
  Polimórfica. Diversos estilos de liderança são usados, reconhecendo
  que a efetividade vem em muitas formas.
\item
  Empoderada. Em vez de ``comandar e controlar'' todos são capacitados e
  têm a capacidade de moldar o futuro.
\item
  Multifacetada. A organização espelha a sociedade em que vivemos ---
  multicultural e reflete uma ampla gama de religiões, culturas e
  etnias.
\item
  Meritocrática e justa. Os processos são justos e todos são tratados
  igualmente, em ambientes livres de preconceitos.
\item
  Cuidadosa e segura. O ambiente é sem medo, não hierárquico e não
  violento.
\item
  Respeitosa. As mulheres são consideradas pares; todos têm a mesma voz
  e podem ser ouvidos por todos.
\item
  Equilibrada. A organização permite o equilíbrio entre vida pessoal e
  profissional, o que significa que não há mais horas longas de trabalho
  e compreende"-se que o desempenho não está vinculado à presença física
  e ao comprometimento de tempo.
\item
  Global e ágil. Há conectividade total, em escala global, e
  flexibilidade --- alavancando a tecnologia.
\item
  Inventiva. Um \versal{CEO} com visão de futuro é cercado por \emph{millennials}
  arrojados e criativos.\footnote{Cf. ``Women Matter: Time to accelerate
    --- Ten years of insights into gender diversity''. Disponível em:
    \textless{}\emph{https://cutt.ly/uysigUL}\textgreater{}.}
\end{enumerate}

Enquanto aparelho de poder, as empresas transnacionais não se limitam a
ditar os métodos de exploração dos trabalhadores que estão em sua folha
de pagamentos. Elas são soberanas e exercerem sua autoridade sobre a
grande maioria da população, a começar pelo ordenamento da vida das
famílias de toda a classe dos trabalhadores (local de moradia, leque de
necessidades e forma de satisfazê"-las etc.), por isso João Bernardo
(2009) qualifica as empresas enquanto Estado Amplo, e é enquanto
aparelhos de poder que elas precisam assimilar eficazmente as lutas
identitárias, recuperando conquistas e antecipando conflitos sociais de
modo estrategicamente lucrativo. O exemplo a seguir nos traz algumas
lições:

\begin{quote}
A instauração do colonialismo moderno em África deveu"-se à iniciativa de
capitalistas privados, que durante algum tempo mantiveram o exclusivo
das operações, inaugurando a primeira experiência de soberania integral
das empresas. Detentoras de exércitos próprios, as empresas conduziam
guerras ou assinavam acordos diplomáticos com os potentados autóctones,
e nos territórios que passaram a controlar foram elas quem estabeleceu
os órgãos do sistema administrativo, judiciário e repressivo. As armas,
o chicote e a palmatória deixaram nos corpos traços indeléveis, mas para
implantarem o capitalismo em África, portugueses, ingleses, franceses e
belgas usaram como principal instrumento a cobrança do imposto de
palhota. Tratava"-se simplesmente de obrigar as famílias nativas a pagar
um imposto em dinheiro. Por vezes, como medida de excepção e apenas
durante um período transitório, o imposto podia ser pago em géneros, mas
o objectivo era a cobrança monetária. Como só se aceitava a moeda
emitida pela potência colonizadora, mesmo em regiões onde existiam
tradicionalmente outros instrumentos pecuniários, e como só através do
mercado os negros tinham acesso a essa moeda, eles viam"-se na
necessidade de vender géneros agrícolas ou a própria força de trabalho.
Quanto aos géneros agrícolas, os colonialistas estavam interessados
apenas nos que pudessem servir de matéria"-prima às suas indústrias ou,
de modo geral, à actividade económica das metrópoles, o que levou a
população autóctone a alterar drasticamente as suas plantações, com
efeitos desastrosos sobre as colectividades locais. Mas o que as
empresas coloniais acima de tudo desejavam era comprar aos nativos a
força de trabalho, ou seja, proletarizá"-los. {[}\ldots{}{]} o relatório de
uma das subcomissões de um congresso reunido em Lisboa em 1911 e 1912
por iniciativa da Sociedade de Geografia havia afirmado com notável
concisão: «Obrigar, pelos impostos directos, os indígenas nas colónias a
trabalhar, para poderem pagar o imposto criando"-lhes quanto possível
necessidades que só pelo trabalho assíduo possam satisfazer».
Contrariamente ao que sucedera em África durante a época do
mercantilismo, quando os comerciantes europeus mantiveram relações com
os sistemas sócio"-económicos tradicionais, o colonialismo moderno
destruiu os sistemas tradicionais para os substituir pelo capitalismo. A
exportação de capitais é acima de tudo uma exportação das relações de
trabalho proletárias, e sob a moeda sonante do imposto de palhota era o
assalariamento que progredia. Deste modo, a influência das empresas
coloniais exerceu"-se, não só indirecta mas directamente, sobre a
globalidade da população colonizada. (\versal{BERNARDO}, 2005: 7).
\end{quote}

É por levar em conta o caráter ampliado e sistêmico das estratégias de
crescimento econômico que a expansão das empresas imperialistas
brasileiras no continente africano não se limita ao plano econômico e
busca lançar profundas raízes sociais no continente como, por exemplo,
por meio da implantação da Capes"-África e de centenas de projetos e
parcerias estratégicas estatais e empresariais costuradas por
brasileiros e africanos. Nesse mesmo sentido é interessante observar a
política de concessão de bolsas de graduação e pós"-graduação a
estudantes oriundos dos países africanos de língua portuguesa, o que
leva ao estreitamento das relações entre as elites destes países e as
classes dominantes brasileiras, aperfeiçoando os mecanismos de
recrutamento de gestores e consolidando as vias de penetração do
imperialismo/colonialismo brasileiro. Compreende"-se, então, como é
importante para a expansão deste imperialismo em África a aplicação da
política de cotas para futuros diplomatas, no Instituto Rio
Branco/Itamaraty.

Como a esquerda brasileira tem reagido a todo esse processo? Em geral há
uma indiferença com respeito ao imperialismo brasileiro, como se se
tratasse de temas alheios à prática anticapitalista. Outra postura
típica é a de reafirmação dogmática do caráter dependente do capitalismo
no Brasil. Por fim, há a postura complacente de se analisar as cotas e
programas de estreitamento de relações com a África como simples
conquista que resultou de anos de luta contra o racismo, assim como
assimilações e apropriações de bandeiras da luta feminista são vistas
não como um fortalecimento estratégico do capitalismo, mas como uma
imposição de pautas por parte das lutas feministas.

Não por acaso, assim como figuras de algumas vertentes do movimento
feminista aplaudem a presença de mulheres em postos de comando de
empresas e cargos públicos, figuras de algumas vertentes do movimento
negro aplaudem a expansão brasileira em África dizendo que assim o
Brasil ``reencontrava suas verdadeiras raízes''. Ou seja, as novas
elites já surgirão munidas de legitimação ideológica e
política.\footnote{Cf. \versal{ARANTES}, D. (2015).} Os limites da integração de
negros nos cargos de comando já foram apontados por alguns. Em \emph{As
faces do empreendedorismo negro} (\versal{BORGES}, 2016) lemos, por exemplo, o
seguinte:

\begin{quote}
A natureza do capitalismo brasileiro e o processo de acumulação de
riquezas no país estão fundados na exploração do povo negro e nos 350
anos de escravidão, de acordo com Juninho Jr. Segundo o membro do
Círculo Palmarino, esse processo chegou ao ponto de 0,2\% da população
deter 50\% da riqueza no país. Por isso, ``ou você de fato garante um
processo de distribuição de riqueza, você garante um processo de
empoderamento real dessa população, ou nós vamos ter uma parcela
ascendendo, formando uma espécie de uma elite negra, em detrimento de
uma grande parcela da população pauperizada'' (2016: p.\,10).
\end{quote}

Não é por acaso que o Movimento Black Money\footnote{Cf.
  \textless{}\emph{https://movimentoblackmoney.com.br/}\textgreater{}.} vem ganhando repercussão
por meio da defesa do fortalecimento econômico de empreendedores negros,
tendo desenvolvido a \emph{StartBlackUp}, uma comunidade de
\emph{networking} que visa colocar em contato afroempreendedores
alinhados à hashtag \emph{\#invistapreto}.

Nina Silva, Executiva de Tecnologia da Informação e fundadora do D'Black
Bank (\versal{DBB}), que tem como \emph{slogan} ``O banco feito de negro para
negro'', afirma que

\begin{quote}
não faltam exemplos de que criatividade, coletivismo e iniciativa são
pilares de sobrevivência da população negra, mesmo sem visibilidade e
reconhecimento da história pelo sistema educacional e mídias. Durante e
no pós"-período escravocrata a população afrodescendente lutou e tem
lutado para garantir sua subsistência, onde o que denominamos ``Black
Money'' tem sido a garantia de que podemos ser o nosso próprio mercado,
em combate à marginalização e subalternação dos empreendedores e
profissionais negros. Black Money também é resistência ao genocídio
histórico da população negra.
\end{quote}

Em matéria\footnote{Cf.
  \textless{}\emph{https://cutt.ly/4ysijcg}\textgreater{}.}
de dezembro de 2017 intitulada ``Ascensão econômica, Crédito e Black
Money: a contramão da realidade dos empreendedores negros no Brasil''
podemos nos informar de que

\begin{quote}
não faltam exemplos de projetos e empresas atuantes do ecossistema do
Empreendedorismo Negro em todo o Brasil: o crowndfunding da parceria
entre a Ganbatte e o Capital Herdeiro, onde uma formação de Mercado
Financeiro será financiada para capacitar uma turma de jovens de baixa
renda na intenção de inseri"-los no mercado de trabalho (contribua aqui:
\textless{}\emph{http://juntos.com.vc/pt/impulsionando-talentos}\textgreater{}); iniciativas
transformadoras como o D´Black Bank, uma fintech brasileira de negros
para negros que será lançada em 2018 com o intuito do fomento à
comunidade afrodescendente na manutenção do consumo de produtos e
serviços para circulação de riqueza dentro de seu grupo étnico racial
(\textless{}\emph{http://www.dblackbank.com.br}\textgreater{}); ainda temos diversas plataformas
de serviços para Negros como a Diáspora Black, plataforma digital que
conecta viajantes e anfitriões interessados em vivenciar experiências de
viagens focadas na história e cultura da comunidade negra em diferentes
cidades do mundo, que está sendo incubada pelo Estação Hack, projeto do
Facebook; na moda temos os Afrocriadores, coletivo que nasceu entre
conversas nos intervalos de oficinas do projeto Sebrae Moda/\versal{RJ}; e o Vale
do Dendê que visa criar uma plataforma de atração de investimentos
sociais e econômicos para a revitalização do centro de Salvador,
restaurando espaços públicos e privados, formando mão de obra
qualificada e criando um novo branding para a cidade.
\end{quote}

Na sequência da reportagem, Débora Santos, Gerente de Projetos da \versal{IBM},
líder do grupo \versal{BRGA}fro e Advisor do Movimento Black Money diz que

\begin{quote}
A cor da pele não define a capacidade de ninguém. Contudo, o negro
enfrenta tantas dificuldades para conquistar o seu espaço desde criança,
que acaba desenvolvendo soft skills como resiliência, empatia e
persuasão que são características essenciais para qualquer empreendedor.
\end{quote}

O elogio às \emph{``soft skills''} do negro e sua conversão em algo
lucrativo sintetiza, simbolicamente, todo um conjunto de contradições
inerentes às políticas identitárias e suas absorções lucrativas por
parte das empresas. Uma visão crítica, comprometida com a política
anticapitalista, talvez visse esse elogio como tendo, em si mesmo, algo
de perverso.

Os movimentos feministas e antirracistas e os que lutam pelas bandeiras
\versal{LGBTIQ}+ poderão seguir na defesa de sua luta e de suas justas demandas
de integração e de combate à desigualdade de gênero e raça, mesmo depois
de compreender como o capitalismo se fortalece a partir delas, afinal
podem muito bem defender que a mera existência de um patrão
negro\footnote{Sobre o empreendedorismo negro ver \versal{RIBEIRO}, Djamila
  (2015); \versal{SEBRAE} (2013); \versal{SPITZ}, C. (2013); \versal{INSTITUTO ADOLPHO BAUER}
  (2016); \versal{JAIME} (2013); \versal{MORAES E SILVA} (2016); \versal{VOROS} (2015); \versal{AGUIAR}
  (2014); \versal{EXAME} (2015); bem como todas as notícias e pesquisas do \versal{CEERT}
  (Centro de Estudos das Relações de Trabalho e Desigualdades).} ou uma
patroa mulher ou gay pode vir a trazer consequências estratégicas no
âmbito de redução das opressões de gênero, raça e sexualidade.

É uma opção política e, como vimos, vem dando resultados, inclusive para
o capital.

Só não se queira defender esse tipo de lutas (suas demandas e formas
organizativas) como se elas tivessem algo que ver com a esquerda e com
as lutas anticapitalistas, uma vez que estas, com sua ênfase nas
relações sociais entre as classes, lutam não por patrões coloridos e
diversificados, mas por um mundo sem patrões ou patroas de qualquer
tipo.

Antes da esquerda pós"-moderna e da política identitária adquirirem
hegemonia nas lutas sociais o combate ao racismo, machismo e à homo ou
``lgbtiqfobia'' tinha por objetivo que a cor de pele, o sexo/gênero e a
sexualidade passassem a ser um elemento pessoal indiferente, assim como
a cor do cabelo, o tamanho do nariz ou se uma pessoa é careca ou
cabeluda.

Assim, buscava"-se combater a discriminação com vistas ao asseguramento
de uma igualdade entre as pessoas, começando pelo aspecto econômico. As
lutas negras (antirracistas) e o feminismo atualmente hegemônico tiraram
o foco da busca pela igualdade e passaram a pretender uma inversão das
hierarquias históricas, priorizando políticas compensatórias.

A luta contra as discriminações é conduzida, pelos identitários, de
forma supraclassista, confundindo, nos mesmos movimentos, as
discriminações que existem no âmbito dos capitalistas e aquelas que
existem no âmbito dos trabalhadores. Além disso, a luta contra as
opressões e discriminações é conduzida como um movimento de ascensão de
novas elites, o que fica claro se observarmos que a preocupação dos
movimentos identitários se centra no acesso às altas esferas decisórias,
às administrações das empresas, aos governos e parlamentos, e não, por
exemplo, com o acesso das mulheres a cargos de baixa remuneração e nulo
poder decisório, como o de operário da construção civil, onde não estão
representadas. Assim, a luta contra as discriminações não leva a uma
nova consciência de classe, servindo, pelo contrário, para fragmentar e
diluir essa consciência numa miríade de identidades excludentes.

Em termos de teorização a política identitária legitima suas bandeiras e
formas de luta acentuando as clivagens e os elementos da
diferença/desigualdade entre as ``pessoas''. O elemento de classe é
deixado de lado e só mobilizado discursiva e oportunisticamente. Ora se
recorre ao natural e biológico, ora se recorre ao cultural e histórico.
A articulação entre identidade e hierarquia fica bastante evidente não
apenas quando observamos a criação dos ``espaços exclusivos'', de
``espaços seguros'', a gramática do ``lugar de fala'' e do
``protagonismo'', ou a auto atribuição do movimento negro enquanto
julgador, no âmbito da política de cotas, das modalidades de decisão
acerca de quem é e de quem não é negro, mas se evidencia também quando
observamos os ataques do movimento negro aos mestiços e aos negros e
negras que se envolvem amorosamente com brancos e brancas (os ``negros
palmiteiros'') ou quando os movimentos feministas discriminam as
transgênero enquanto não mulheres, ou, ainda, quando os movimentos
\versal{LGBTIQ}+ estabelecem hierarquias entre as letras e advogam formas de
moralismo e de comportamento social aceitável ou não, bem como
discriminações contra mulheres feministas hétero que se envolvem
amorosamente com homens etc. Em todas estas situações vemos ecoar
nitidamente os ecos do fascismo, mas há ainda outros.

A busca pela igualdade e pelo universalismo, que durante séculos
caracterizou a esquerda, foi deixada de lado e o que norteia as
esquerdas hoje em dia é o próprio elogio à fragmentação e a formas
nefastas de hierarquização entre trabalhadores e trabalhadoras. O
pós"-modernismo, com a crítica ao ``eurocentrismo'' e à ``grande
narrativa'', criou as condições ideológicas para o triunfo do
identitarismo. Os repetidos fracassos da esquerda clássica em oferecer
respostas satisfatórias às opressões e discriminações garantiram as
condições políticas para o triunfo do identitarismo, numa conjuntura
longa de rebaixamento das expectativas anticapitalistas.

O feminismo identitário, o movimento negro identitário e os movimentos
\versal{LGBT}s identitários constituem modos de ocultar as cisões de classe que
existem no interior das presumidas identidades, reforçando as
autocriadas cisões identitárias existentes no interior das classes. A
política identitária é uma forma de mobilizar não a classe trabalhadora,
mas pessoas, pessoas oprimidas, pessoas periféricas, historicamente
``silenciadas'', pessoas discriminadas por um ou outro aspecto físico ou
social. Nesse sentido, os motivos de insatisfação e contestação podem e,
como vimos, costumam ser, legítimos, mas a forma de encará"-los e o modo
de organização da luta social pretendem, no âmbito do identitarismo,
estar acima das divisões entre classes, apagando o que une a classe e
reforçando o que a fragmenta, numa miríade quase sem fim de identidades
que se sobrepõem e se somam em cada pessoa.

É comum, nos movimentos identitários, qualquer intervenção começar com
uma auto apresentação, p.\,ex.: ``meu nome é Fulana, sou mulher, sou
negra, sou favelada, sou lésbica, estou desempregada, sempre fui
silenciada, nunca tive voz, mas sou brasileira, sou batalhadora e hoje
eu vou falar\ldots{}''. O apelo emotivo visa garantir de antemão uma
modalidade de solidariedade baseada na vivência individual e não na
comum inserção estrutural em tal ou tal esfera da produção e suas formas
dinâmicas de sociabilidade. Visa, igualmente, desarmar e silenciar
qualquer intervenção crítica que venha de algum sujeito que não
carregue, em seu corpo ou em sua bagagem de vida, as mesmas experiências
individuais daquela pessoa a quem se pretende interpor uma crítica por
conta de algo que foi dito ou defendido. Com isso o identitarismo
bloqueia a prática política, delimita o livre curso do debate de ideias
e obstrui o confronto de ideologias, estabelecendo uma hierarquia
discursiva de cariz fascista.

Em termos de desenvolvimento da ciência e do pensamento crítico as
implicações do identitarismo são ainda mais nefastas, pois há uma
hierarquização teórica de acordo com a identidade de quem as proferiu, e
não de acordo com critérios objetivamente comparáveis etc. Assim, por
exemplo, um debate sobre cotas raciais, dentro da Academia, é organizado
e levado a cabo por negros, mesmo se naquele campus há um branco que
conhece ou é especialista no tema. O mesmo quanto a textos ou debates
acerca do machismo, lgbtfobia e etc.: confunde"-se o fazer científico com
o mero fato de que a visão de um sujeito oprimido acerca da própria
opressão muitas vezes traz ou ressalta elementos importantes para a
compreensão de facetas da temática.

Esquece"-se, inclusive, que em termos de elaboração teórica e política a
\emph{vivência}, assim como pode ser um trunfo, pode ser uma
desvantagem, pois o fato do sujeito sofrer na pele a opressão pode até
mesmo levá"-lo a \emph{compreender menos} o caráter e estrutura da
opressão, confundido por sentimentos de rancor, ódio, nojo, repulsa etc.
Ou seja, mesmo em termos de comparação de vivências a questão se
distancia de algo como a garantia de um maior saber científico ou
``falar com propriedade'': assim como uma mulher pode mostrar a um homem
facetas do machismo, um homem pode ensinar à mulher outras facetas,
como, por exemplo, as minúcias do processo histórico e cultural da
formação de um machista desde a infância etc. A regra básica a se seguir
é a de que é científico aquilo que encontra correspondência objetiva nos
fatos: ``o que foi dito'', e não aquilo que se garante subjetivamente,
por conta de ``quem o disse''. Não atoa o elogio pós"-moderno ao
discurso, em detrimento da ciência e da razão modernas, está na base
metodológica de muitos dos teóricos precursores das políticas
identitárias, como por exemplo, Michel Foucault, Michel de Certeau, os
autores pós"-colonialistas etc.

Se em termos teóricos o desastre identitário se anuncia, em termos
políticos ele se completa: o resultado primário básico do identitarismo
é a fragmentação da classe e seu enfraquecimento enquanto classe, pois a
atuação de seus membros se dá em termos de identidades, resultando,
inclusive, em alianças entre indivíduos que em termos de classe deveriam
estar em guerra ou então resulta, por outro lado, em batalhas
fratricidas entre indivíduos que deveriam, enquanto membros de uma mesma
classe, estar unidos na luta contra inimigos em comum. A própria
colocação estrutural do trabalhador enquanto oposto ao capitalista se
converte em apenas mais uma identidade como qualquer outra, a identidade
de se ser trabalhador e trabalhadora, o que na prática significa a
recusa da noção de classe.

Em artigo de 2009, intitulado ``Entre a luta de classes e o
ressentimento'', João Bernardo afirmou que

\begin{quote}
Numa época em que, perante a concentração transnacional do grande
capital, os trabalhadores se encontram fragmentados, quando foram em boa
medida dissolvidas as suas antigas relações de solidariedade e atenuado
ou extinto o seu sentimento de classe, mais fácil se torna que eles
encontrem nos pequenos patrões os leaders ou os modelos. No plano
ideológico e psicológico, trata"-se de substituir o espírito de classe
pelo ressentimento, ou seja, o desejo de acabar com o capitalismo pela
aspiração de subir dentro do capitalismo. O fascismo, na face que
apresentou às massas populares, foi exactamente isto. \footnote{Cf.
  \textless{}\emph{https://cutt.ly/JysizmP}\textgreater{}.}
\end{quote}

Ao que parece, atualmente poderiam ser incluídas na lista dos ``pequenos
patrões'' os chefes e as chefas das organizações identitárias. Retomando
as reflexões daquele artigo, em \emph{Labirintos do Fascismo} (2018)
João Bernardo escreveu que ``com o abandono da esperança revolucionária,
a hostilidade de classe passava a assumir a forma degenerada do
ressentimento''.

\begin{quote}
Diluídas as redes de solidariedade, os trabalhadores já não apareciam
como membros de uma classe e apresentavam"-se como elementos das massas.
Uma massa agitada pelo descontentamento, mas sem nenhuma expectativa que
não se cingisse à sociedade existente --- eis a base popular da revolta
dentro da ordem. Foi nessa gente que o fascismo se apoiou para eliminar
as chefias operárias tradicionais, isolar as vanguardas combativas e
reorganizar o Estado consoante um novo modelo ditatorial. E fê"-lo tanto
mais facilmente quanto o refluxo do movimento revolucionário havia
fragilizado a base de sustentação de socialistas e comunistas, e a
repressão conduzida contra os trabalhadores mais ousados comprometera
qualquer prestígio de que os governos liberais tivessem podido gozar
entre a população humilde. (2018: 26)
\end{quote}

Segundo este autor, o mesmo ressentimento que moveu e move os fascismos
alimenta hoje em dia os identitarismos, o que expressa um ponto de
convergência entre as formas clássicas de fascismo e o fascismo
pós"-fascista. O antagonismo existente entre o ressentimento e o espírito
de classe é similar àquele que existe entre o identitarismo e a luta
contra o capitalismo.

A pesquisa que levei a cabo me fez concluir que havendo modernização, ou
seja, vencendo as empresas mais modernas e seus métodos de recrutamento
de gestores com vistas à consolidação da infraestrutura social da
mais"-valia, as lutas pela igualdade de sexos, pelos direitos \versal{LGBTIQ}+ e
contra o racismo sairão ``vitoriosas''. Essas bandeiras já estão bem
posicionadas e em uma sociedade profundamente modernizada o atendimento
das pautas identitárias está assegurado, sendo apenas questão de tempo
até as resistências de elites nacionais retrógradas serem quebradas pela
força modernizante das empresas transnacionais. Estas estão priorizando
estratégias voltadas não apenas para a \versal{P\&D} (Pesquisa e
Desenvolvimento), mas também à \versal{I\&D}: inclusão e diversidade. A adoção
empresarial da agenda da diversidade é vista como uma fonte de vantagem
competitiva\footnote{``Embora a correlação não seja igual à causalidade
  (maior diversidade étnica e de gênero na liderança corporativa não se
  traduz automaticamente em mais lucro), a correlação indica que, quando
  as empresas se comprometem com uma liderança diversificada, elas são
  mais bem"-sucedidas. Acreditamos que as empresas mais diversificadas
  são mais capazes de conquistar os melhores talentos e melhorar a
  orientação para o cliente, a satisfação dos funcionários e a tomada de
  decisões, e tudo isso leva a um ciclo virtuoso de retornos crescentes.
  Isso, por sua vez, sugere que outros tipos de diversidade --- por
  exemplo, em idade, orientação sexual e experiência (como uma
  mentalidade global e fluência cultural) --- também podem trazer algum
  nível de vantagem competitiva para empresas que podem atrair e reter
  tal talento diverso.'' Cf.
  \textless{}\emph{https://cutt.ly/eysixYb}\textgreater{}.}
e fator"-chave para o crescimento econômico.

Embora a relação entre maior diversidade e crescimento econômico seja
irrefutável\footnote{Quanto a isso, inclusive, os dados têm crescido:
  ``usando dados de diversidade de 2014, descobrimos que as empresas no
  primeiro quartil para a diversidade de gênero em suas equipes
  executivas eram 15\% mais propensas a ter uma lucratividade acima da
  média do que as empresas no quarto quartil. Em nosso conjunto de dados
  expandido de 2017, esse número subiu para 21\% e continuou sendo
  estatisticamente significativo. Para a diversidade étnica e cultural,
  a descoberta de 2014 foi uma probabilidade de 35 por cento de
  desempenho superior, comparável à descoberta de 2017 de uma
  probabilidade de 33 por cento de maior desempenho na margem do \versal{EBIT}''
  (o lucro empresarial ``puro'', antes das deduções de juros, impostos
  etc.). cf.
  \textless{}\emph{https://cutt.ly/MysicJc}\textgreater{}.},
de acordo com uma das pesquisas da McKinsey o progresso nas
implementações práticas de medidas que garantam a mobilidade social
ascendente de mulheres, negros e \versal{LGBT}s em todos os níveis da empresa tem
sido demasiado lento. Os resultados, nalguns casos, apresentaram uma
relativa desaceleração: ``A grande maioria das empresas diz que está
altamente comprometida com a diversidade racial e de gênero --- mas as
evidências indicam que muitas ainda não estão tratando a diversidade
como o imperativo comercial. Isso é evidente na falta de progresso no
\emph{pipeline} {[}a presença da diversidade na estrutura hierárquica
das carreiras dentro de cada empresa{]} nos últimos quatro
anos''.\footnote{Cf. \textless{}\emph{https://cutt.ly/LysivMI}\textgreater{}.}

A lenta inserção de mais mulheres, negros e \versal{LGBT}s em todos os níveis das
carreiras e especialmente nos mais altos cargos de comando não tem
mantido o grau de sucesso que se esperaria tendo em vista os dados de
2007 a 2015, o que pode indicar que demorará um tempo maior até que
sejam extirpadas, nas grandes empresas, as disparidades de representação
de gênero, raça e sexualidade nos cargos de comando. Em todo caso, os
dados mais recentes não indicam, de modo algum, uma mudança da
tendência.

O mais provável é que essa desaceleração e relativa estagnação dos
avanços da agenda da diversidade tenham ocorrido por conta de uma
mudança na política e nos ares dos \versal{EUA} sob Trump, o que impacta
decisivamente nas pesquisas, pois a maioria dos dados toma por base e se
refere a empresas transnacionais de base estadunidense. Além disso, as
pesquisas mais recentes têm expandido significativamente o arcabouço e
volume de dados analisados, o que implica cada vez mais tomar como base
também os dados de empresas menores, as quais, obviamente, estão menos
aptas a ter sucesso na implementação de medidas voltadas para um quadro
de comando empresarial mais colorido.

A pesquisa exposta neste livro tratou principalmente de \emph{uma} das
linhas tendenciais do capitalismo contemporâneo: aquela apresentada
pelas maiores empresas transnacionais, e não pela média de empresas de
todos os portes. A nosso ver o método marxista implica analisar os
\emph{casos mais desenvolvidos} e tomá"-los como referência para
vislumbrar as tendências históricas, que podem ou não se confirmar, pois
a própria existência de uma tendência implica contratendências. É comum
que os casos \emph{típicos} mais desenvolvidos arrastem os casos menos
desenvolvidos, ditando"-lhes as linhas e caminhos a seguir. Ou, como é
dito nas últimas linhas de um dos relatórios da \emph{McKinsey Global
Institute}: ``Dados os retornos mais altos que a diversidade deve
trazer, acreditamos que é melhor investir agora, já que os vencedores
irão se expandir mais e os retardatários ficarão mais para trás''.

O fato de esta tendência ser forte, o fato de a estratégia de absorção
das pautas identitárias ser algo nitidamente benéfico ao sistema
capitalista e às empresas que a implementam, não significa, de modo
algum, que este caminho será seguido, pois o caráter anárquico da
organização capitalista como um todo (especialmente por conta da atuação
dos burgueses, proprietários de capital) está sempre em conflito com o
caráter minuciosamente organizado das empresas administradas pela classe
dos gestores. Esse conflito de estratégias e formas de gestão do
capitalismo se dá em um terreno permeado de contradições que podem,
porventura, impedir a vitória da tendência apontada neste livro, dando
lugar às formas mais nefastas e menos refinadas de exploração do
trabalho. Trata"-se, nalguma medida, de um embate entre formas arcaicas e
formas modernas de gestão das relações sociais e formas de organização
da exploração do valor, sendo que nada garante que os mecanismos de
mais"-valia absoluta, embora pouco eficazes, sigam desempenhando papel
central no sistema como um todo, o que implica que a violência seguirá
desempenhando um papel mais central do que as refinadas formas de
incremento da lucratividade por meio da assimilação da agenda da
diversidade.

Assim, pode"-se justamente levantar, contra as teses deste livro,
\emph{outras tendências} igualmente operantes no capitalismo dos dias
atuais, como por exemplo, o encarceramento em massa e extermínio de
negros e moradores das periferias, a financeirização e o crescimento do
capital fictício etc. O mais plausível é que se trate de tendências
complementares, modos distintos de lidar com facetas da sociabilidade
capitalista. A barbárie do capitalismo brucutu e a barbárie do
capitalismo colorido podem conviver num mesmo mundo capitalista, onde
uma e outra tendência assume a primazia a depender da região e da
correlação de forças nos embates de classes e nas lutas sociais. Dito de
modo direto: na teia de contradições do capitalismo atual, alguns negros
serão selecionados para serem os novos gestores em empresas modernas
enquanto outros serão selecionados para serem pura e simplesmente
exterminados e, do mesmo modo, algumas mulheres serão selecionadas para
liderarem empresas, enquanto outras continuarão a desempenhar funções
rebaixadas na hierarquia social. O processo de seleção, em si, não tem
nada de novo. O que há de novo é a participação ativa da esquerda
identitária no sentido de oferecer, ao capitalismo em crise, soluções
que lhe permitem retomar o fôlego para novos ciclos de acumulação.

Num mundo de predominância da ideologia do empreendedorismo de si mesmo
(Dardot e Laval) temos tanto os empreendedores coloridos, para os quais
estão reservados assentos na mesa das elites, quanto os não
empreendedores, carne barata a ser triturada nos moinhos repressivos do
capital. A diferença, entretanto, é que enquanto algumas das tendências
atuantes no cenário capitalista de hoje em dia são reciclagens,
aprimoramentos e aprofundamentos de tendências conhecidas, este livro se
debruçou sobre uma \emph{tendência nova}, recentíssima, que envolve as
práticas políticas do que se convencionou seguir chamando de
``esquerda''.

Se a agenda da diversidade está assegurada em um capitalismo
desenvolvido, o que temos hoje é um novo conjunto de questões
relacionadas à tentativa de constituir grupos à parte, que surjam e se
consolidem enquanto grupos hierárquicos, elites destinadas a chefiar
essas pretensas identidades. Isso desloca o problema da questão da
desigualdade para a questão da formação de novas burocracias, o que é
destrutivo não apenas em termos de pretensões anticapitalistas, mas
também em termos de luta contra o racismo, a lgbtfobia e o machismo.

Se de fato a questão central, a ser enfrentada pela política
anticapitalista, residir na formação de novas burocracias e de novas
elites, os próprios identitários de destaque possuem o interesse de
acentuar as cisões e os preconceitos de que são e de que dizem ser
vítimas (o que vem a resultar no mesmo, numa cyber"-sociedade calcada no
espetáculo, na \emph{performance} e no simulacro) e que lhes garante
individualmente lugares de status e ascensão social.

Ao elevar midiática e ideologicamente o machismo, homofobia e racismo a
patamares maiores que os reais estas elites identitárias consolidam"-se
como chefias de rebanhos cada vez maiores de trabalhadores e
trabalhadoras, direcionando essas pessoas para modalidades de luta que
não enfrentam de fato o racismo, o machismo e a lgbtfobia e sim garantem
o desenvolvimento econômico e a ascensão de novas elites negras,
femininas e \versal{LGBT}s.

Por fim, se as teses defendidas neste livro constituem uma dura crítica
às respostas identitárias às questões relativas às opressões, não muito
melhor é a situação das teorias e organizações classistas que, malgrado
os esforços, ainda não conseguiram oferecer, à classe trabalhadora,
alternativas teóricas, práticas e organizativas que pudessem articular
satisfatoriamente a luta contra a exploração e contra as formas de
opressão, sentidas na pele cotidianamente.

Buscamos trazer alguns elementos para mostrar que, em termos de
anticapitalismo, as propostas teórico"-práticas identitárias são muito
problemáticas. Resta"-nos, enquanto trabalhadoras e trabalhadores
interessados na construção de um mundo melhor, inventar e construir os
alicerces políticos capazes de pôr fim tanto à exploração quanto ao
racismo, machismo e outras formas de discriminação. Para tal, cabe
observar com cuidado a relação de complementariedade ou antagonismo
entre capitalismo e estas formas de opressão e dominação, bem como as
formas como têm se dado o desenvolvimento econômico e a atuação de
empresas e governos na conversão das lutas em torno de pautas
identitárias em algo lucrativo que reforça o sistema.

Precisamos estar atentos às apropriações capitalistas das formas de luta
e bandeiras que, mais fortemente nas últimas décadas, trabalhadores e
trabalhadoras têm levantado enquanto questões essenciais para uma vida
melhor. O desafio é grande e precisamos nos armar para estar à altura do
que a história nos exige. Um primeiro passo é compreender e reconhecer
os erros estratégicos das lutas sociais que empreendemos até aqui.

\chapter{Bibliografia}

\begin{bibliohedra}
\tit{ABBOTT}, Lawrence J. et~al. ``Female Board Presence and the Likelihood
of Financial Restatement,'' Accounting Horizons, vol. 26, no. 4, 2012.

\tit{ABDULLAH}, Shamsul et~al. ``Women on Boards of Malaysian Firms: Impact
on Market and Accounting Performance,'' Social Science Research Network,
Working Paper Series, September 2012.

\tit{ABÍLIO}, L. Sem maquiagem. O trabalho de um milhão de revendedoras de
cosméticos. \versal{SP}: Boitempo, 2014.

\tit{ADAMS}, R. B., e D. \versal{FERREIRA}. Women in the Boardroom and Their Impact on
Governance and Performance. Journal of Financial Economics 94, no. 2:
291--309. 2009.

\tit{ADLER}, Roy. ``Profit, Thy Na me Is \ldots{} Woman?'' Miller"-McCune,
February 27, 2009.

\tit{ADRIÃO}, K. G. et~al. O movimento feminista brasileiro na virada do
século \versal{XX}: reflexões sobre sujeitos políticos na interface com as noções
de democracia e autonomia. Rev. Estud. Fem. vol.19 n.3 Florianópolis,
2011

\tit{AGUIAR}, V. Empreendedores negros ainda têm dificuldade em conseguir
crédito. 2014.
Disponível em: \textless\emph{https://cutt.ly/pysiWk5}\textgreater{}.

\tit{AHERN}, Kenneth R., e Amy K. \versal{DITTMAR}. The Changing of the Boards: the
Impact of Firm Valuation of Mandated Female Board Representation.
Quarterly Journal of Economics 127, no. 1: 137--97. 2012.

\tit{ALVAREZ}, S. Um outro mundo (também feminista\ldots{}) é possível: construindo
espaços transnacionais e alternativas globais a partir dos movimentos.
Rev. Estud. Fem. vol.11 n.2 Florianópolis, 2003

\tit{ARANTES}, Durval. A História africana pode resgatar a autoestima dos
afrodescendentes. 2015.
Disponível em: \textless{}\emph{https://cutt.ly/OysiE3z}\textgreater{}.

\tit{ASHCRAFT}, C. \& \versal{BREITZMAN}, A. ``Who Invents \versal{IT}?: An Analysis of Women's
Participation in Information Technology Patenting'' National Center for
Women \& Information Technology, 2007.

\tit{AVERY}, Derek R. et~al. ``Is There Method to the Madness? Examining How
Racioethnic Matching Influences Retail Store Productivity,'' Personnel
Psychology, vol. 65, Spring 2012.

\tit{BAIN}\mbox{} \& \versal{COMPANY}. Sem atalhos: O caminho para as mulheres alcançarem o
topo. 2013.

\tit{BANCO MUNDIAL}. Igualdade de Gênero e Desenvolvimento. 2012.
Disponível em: \textless{}\emph{http://siteresources.worldbank.org/INTWDR2012}\textgreater{}.

\tit{BARSTED}, L. L. As relações da Revista Estudos Feministas com os
movimentos de mulheres. Rev. Estud. Fem. vol.16 n.1 Florianópolis, 2008.

\tit{BASTHI}, A. (org.) Guia para Jornalistas sobre Gênero, Raça e Etnia.
Brasília: \versal{ONU} Mulheres; Federação Nacional dos Jornalistas (\versal{FENAJ});
(Fundo de Alcance dos Objetivos do Milênio, \versal{F"-ODM}). 2014.

\tit{BEAR}, Stephen et~al. ``The Impact of Board Diversity and Gender
Composition on Corporate Social Responsibility and Firm Reputation,''
Journal of Business Ethics, vol. 97, n. 2, 2010.

\tit{BERNARDO}, J. Democracia totalitária. \versal{RJ}: Cortez, 2005.

\titidem. Economia dos conflitos sociais. \versal{SP}: Expressão
Popular, 2009.

\titidem. A geopolítica das companhias transnacionais. Passa
Palavra, 2011. Disponível em: \textless{}\emph{https://cutt.ly/gysiNoM}\textgreater{}.

\titidem. Labirintos do fascismo: na encruzilhada da ordem e
da revolta. 3ª versão, revista e aumentada, 2018. Disponível em:
Disponível em: \textless{}\emph{http://tiny.cc/ju8vnz}\textgreater{}.

\tit{BLACKWELL}, M. \& \versal{NABER}, N. Interseccionalidade em uma era de
globalização. As implicações da Conferência Mundial contra o Racismo
para práticas feministas transnacionais. Rev. Estud. Fem. vol.10 n.1
Florianópolis, 2002.

\tit{BLAZOVICH}, Janell L. et~al. ``Do Gay"-friendly Corporate Policies
Enhance Firm Performance?,'' Social Science Research Network, Working
Paper Series, 2013.

\tit{BORGES}, P. As faces do empreendedorismo negro. 2016.
Disponível em: \textless{}\emph{http://almapreta.com/realidade/as-faces-do-empreendedorismo-negro/}\textgreater{}.

\tit{BOSETTI}, V. et~al. ``Migration, Cultural Diversity and Innovation: A
European Perspective,'' Innocenzo Gasparini Institute for Economic
Research, Working Paper n. 469, 2012.

\tit{BRAMMER}, Stephen et~al. ``Corporate Reputation and Women on the
Board,'' British Journal of Management, vol. 20, n.1, 2009.

\tit{BRITO}, F. \& \versal{OLIVEIRA}, P. R. Até o último homem. Visões cariocas da
administração armada da vida social. \versal{SP}: Boitempo, 2013.

\tit{CAMPBELL}, K., \& \versal{MINGUEZ"-VERA}, A. Gender Diversity in the Boardroom and
Firm Financial Performance. Journal of Business Ethics 83, n.3, 2008.

\titidem. ``Female Board Appointments and Firm
Valuation: Short and Long"-Term Effects,'' Journal of Management and
Governance, vol. 14, no. 1, 2009.

\tit{CASTRO}, C. Apple não quer mais mulheres nem negros nas chefias. 2016.
Disponível em: \textless{}\emph{http://economico.sapo.pt/noticias/apple-nao-quer-mais-mulheres-nem-negros-nas-chefias\_239802.html}\textgreater{}.

\tit{CATALYST}. Engaging men in gender initiatives: What change agents need to
know. New York, 2009.

\titidem. The Bottom Line: Corporate Performance and Women's
Representation on Boards (2004--2008). Nancy M. Carter and Harvey M.
Wagner. 2011.

\titidem. Advancing Women Leaders: The Connection Between Women Board
Directors and Women Corporate Officers. Lois Joy. 2008.

\titidem. \versal{EU} Legal Instruments for Gender Quotas in Management Boards.
New York. 2013.

\titidem. Why Diversity Matters. July 2013.
Disponível em: \textless{}\emph{http://tiny.cc/v38vnz}\textgreater{}.

\titidem. Advancing Women Leaders: The Connection Between Women Board
Directors and Women Corporate Officers. Lois \versal{JOY}, 2008.

\tit{CEERT}. Cabo Verde entre 10 países africanos com políticas mais
favoráveis às mulheres.
Disponível em: \textless{}\emph{http://www.ceert.org.br/en/noticias/genero-mulher/10625/cabo-verde-entre-10-paises-africanos-com-politicas-mais-favoraveis-as-mulheres}\textgreater{}.

\titidem. Pesquisa pretende conhecer os negros empreendedores
brasileiros. 2015.
Disponível em: \textless{}\emph{https://cutt.ly/pysi7tc}\textgreater{}.
2015.

\titidem. Pela primeira vez, negros são maioria dos
empreendedores no Brasil, porém desigualdades persistem. 2015.
Disponível em: \textless{}\emph{https://cutt.ly/Xysi7J4}\textgreater{}.

\titidem. O fosso entre brancos e negros no mercado de trabalho.
2016.
Disponível em: \textless{}\emph{https://cutt.ly/Iysi5gQ}\textgreater{}.

\titidem. Mercado de trabalho: Desigualdades de raça e gênero
no executivo federal. 2015.
Disponível em: \textless{}\emph{https://cutt.ly/6ysi52G}\textgreater{}.

\tit{COSTA}, C. L. As publicações feministas e a política transnacional da
tradução: reflexões do campo. Rev. Estud. Fem. vol.11 n.1 Florianópolis,
2003.

\titidem. Feminismos e pós"-colonialismos. Rev. Estud. Fem. vol.21
n.2 Florianópolis, 2013

\tit{CREDIT SUISSE RESEARCH INSTITUTE}. Gender Diversity and Corporate
Performance. August. Zurich. 2012.

\tit{DALE"-OLSEN}, H., P. Schøne, e M. Verner. Diversity among Directors: The
Impact on Performance of a Quota for Women on Company Boards. Feminist
Economics 19, no. 4: 110--35. 2014.

\tit{DEVILLARD}, S.; \versal{HUNT}, V. \& \versal{YEE}, L. Still looking for room at the top:
Ten years of research on women in the workplace. 2018.
Disponível em: \textless{}\emph{https://cutt.ly/gysoqEq}\textgreater{}.

\tit{DEZSO}, C. \& \versal{ROSS}, D. ``When Women Rank High, Firms Profit,'' Columbia
Business School Ideas at Work, June 2008.

\titidem. ``Does Female Representation in Top Management
Improve Firm Performance? A Panel Data Investigation''. Strategic
Management Journal, vol. 33, n.9, 2012.

\tit{DIAS}, T. A relação entre mulheres no comando e o lucro das empresas.
2016.
Disponível em: \textless{}\emph{https://cutt.ly/yysowMR}\textgreater{}.

\tit{DINIZ}, D. \& \versal{FOLTRAN}, P. Gênero e feminismo no Brasil: uma análise da
Revista Estudos Feministas. Rev. Estud. Fem. vol.12 Número Especial,
Florianópolis, 2004.

\tit{ERHARDT}, Niclas L., James D. \versal{WERBEL}, and Charles B. \versal{SHRADER}. Board of
Director Diversity and Firm Financial Performance. Corporate Governance:
An International Review 11, April: 102--11. 2003.

\tit{EU}\mbox{} (European Union). The Quota"-Instrument: Different Approaches across
Europe. European Commission Network to Promote Women in Decision"-Making
in Politics and the Economy. Brussels. 2011.

\titidem. More women in senior positions. Key to economic
stability and growth. 2010.
Disponível em: \textless{}\emph{http://ec.europa.eu/danmark/documents/alle\_emner/beskaeftigelse/more\_women\_in\_senior\_positions.pdf}\textgreater{}.

\tit{EXAME}. Negros já são maioria entre empreendedores. 2015.
Disponível em: \textless{}\emph{https://cutt.ly/2ysoe54}\textgreater{}.

\tit{FERREIRA}, E. S. \& \versal{BORGES}, D. T. Caderno Espaço Feminino: ampliando
espaços e enfrentando desafios. Rev. Estud. Fem. vol.12 no.spe
Florianópolis, 2004

\tit{FLABBI}, Luca et all. ``Do Female Executives Make a Difference? The
Impact of Female Leadership on Firm Performance and Gender Gaps in Wages
and Promotions'', August 7, 2012.

\tit{FOLKMAN}, Zenger. A Study in Leadership: Women do it Better than Men
(2012); Jack Zenger and Joseph Folkman, ``Are Women Better Leaders than
Men?'' \versal{HBR} Blog Network, March 15, 2012

\tit{FRANCOEUR}, Claude et~al. ``Gender Diversity in Corporate Governance and
Top Management,'' Journal of Business Ethics, vol. 81, no. 1, 2008.

\tit{FRASER}, N. Mapeando a imaginação feminista: da redistribuição ao
reconhecimento e à representação. Revista Estudos Feministas,
Florianópolis, v. 15, n. 2, maio de 2007.

\titidem. Reconhecimento sem ética? Lua Nova, São Paulo, v.70,
2007b.

\tit{GROSSI}, M. P. A Revista Estudos Feministas faz 10 anos: uma breve
história do feminismo no Brasil. Rev. Estud. Fem. vol.12, Número
especial, Florianópolis, 2004.

\tit{HEFORSHE}. Movimento ElesPorElas (HeForShe) de Solidariedade da \versal{ONU}
Mulheres pela Igualdade de Gênero -- Visão Geral, Empresas,
Universidades, Governos, Kit de Ação.
Disponível em: \textless{}\emph{http://www.onumulheres.org.br/wp-content/uploads/2015/03/}\textgreater{}.

\tit{HERRING}, Cedric. ``Does Diversity Pay? Race, Gender, and the Business
Case for Diversity'' American Sociological Review, vol. 74, n.2, 2009.

\tit{HOMAN}, Astrid C. \& \versal{GREER}, Lindred L. ``Considering Diversity: The
Positive Effects of Considerate Leadership in Diverse Teams,'' Group
Processes Intergroup Relations, vol. 16, n.1. January 2013.

\tit{HUNT}, V.; \versal{LAYTON}, D. \& \versal{PRINCE}, S. Why diversity matters. jan. 2015.
Disponível em: \textless{}\emph{https://cutt.ly/RysotCq}\textgreater{}.

\tit{INSTITUTO ETHOS \& IBOPE INTELIGÊNCIA}. Perfil Social, Racial e de Gênero
das 500 Maiores Empresas do Brasil e Suas Ações Afirmativas. 2010.

\tit{INSTITUTO ETHOS, BID, PREFEITURA DE SÃO PAULO}. Perfil Social, Racial e
de Gênero dos 200 Principais Fornecedores da Prefeitura de São Paulo.
2016.

\tit{INSTITUTO ETHOS}. Gestão para a Diversidade. Moda ou veio para ficar?.
2015.
Disponível em: \textless{}\emph{https://cutt.ly/JysouiT}\textgreater{}.

\titidem. Por mais diversidade no Oscar e nas corporações. 2016.
Disponível em: \textless{}\emph{https://cutt.ly/UysoisD}\textgreater{}.

\titidem. Novos caminhos para a gestão da diversidade. 2016.
Disponível em: \textless{}\emph{https://cutt.ly/Nysoi9M}\textgreater{}.

\titidem. Mais mulheres nos conselhos de administração mudam a
maneira de fazer negócio?. 2012.
Disponível em: \textless{}\emph{https://cutt.ly/iysoaqD}\textgreater{}.

\titidem. O Fórum São Paulo Diverso e a importância das empresas
na luta pela igualdade. 2015.
Disponível em: \textless{}\emph{https://cutt.ly/cysosk5}\textgreater{}.

\titidem. Novos indicadores Ethos"-\versal{MM}360 para promoção da equidade
de gênero estão disponíveis para sugestões em consulta pública. 2015.
Disponível em: \textless{}\emph{https://cutt.ly/Lysos5M}\textgreater{}.

\titidem. A presença feminina nas empresas. 2015.
Disponível em: \textless{}\emph{https://cutt.ly/7ysod9p}\textgreater{}.

\titidem. A promoção da igualdade racial pelas empresas. 2013.
Disponível em: \textless{}\emph{https://cutt.ly/eysof7a}\textgreater{}.

\tit{INSTITUTO ADOLPHO BAUER}. Desafios para o empreendedorismo negro. 2016.
Disponível em: \textless{}\emph{https://cutt.ly/oysogLb}\textgreater{}.

\tit{IPEA}. Nota Técnica \versal{IPEA} 2011 (maio, n. 8). Planejamento e Financiamento
das Políticas para as Mulheres: possibilidades para o Plano Plurianual
2012-2015.
Disponível em: \textless{}\emph{https://cutt.ly/KysohcW}\textgreater{}.

\tit{JACOME}, M. L. \& \versal{VILLELA}, S. (org.). Orçamentos sensíveis a gênero:
Conceitos. \versal{ONU} Mulheres. Brasília. 2012 e 2012a.

\titidem \mbox{} (org.). Orçamentos sensíveis a gênero:
Experiências. \versal{ONU} Mulheres. Brasília. 2012b.

\tit{JAIME}, P. Executivos negros e movimento antirracista no Brasil. 2013.
Disponível em: \textless{}\emph{https://cutt.ly/iysojoM}\textgreater{}.

\tit{JOECKS}, Jasmin et~al. ``Gender Diversity in the Boardroom and Firm
Performance: What Exactly Constitutes a `Critical Mass'?'' Social
Sciences Research Network, Working Paper Series, February 2012.

\tit{JURKUS}, A. P. \& \versal{WOODWARD}. L. Women in Top Management and Agency Costs.
Journal of Business Research 64, no. 2, 2011.

\tit{LAGO}, M.C.S. Revista estudos feministas, Brasil, 16 anos: uma narrativa.
Ex aequo, n.19 Vila Franca de Xira, 2009.

\tit{LARKIN}, Meredith B. et~al. ``Board Gender Diversity, Corporate
Reputation and Market Performance,'' International Journal of Banking
and Finance, vol. 9, n. 1, 2012.

\tit{LOPES}, M.M. \& \versal{PISCITELLI}, A. Revistas científicas e a constituição do
campo de estudos de gênero: um olhar desde as ``margens''. Rev. Estud.
Fem. vol.12, Número especial, Florianópolis, 2004.

\tit{MANDEL}, E. (1982). O capitalismo tardio. São Paulo: Editora Abril.

\tit{MARIMUTHU}, M. Ethnic and Gender Diversity in Boards of Directors and
Their Relevance to Financial Performance of Malaysian Companies, 2009.

\tit{MATOS}, M. Teorias de gênero ou teorias e gênero? Se e como os estudos de
gênero e feministas se transformaram em um campo novo para as ciências.
Rev. Estud. Fem. vol.16 n.2 Florianópolis, 2008.

\tit{MATSA}, David A., and Amalia R. \versal{MILLER}. A Female Style in Corporate
Leadership? Evidence from Quotas. American Economic Journal: Applied
Economics 5, no. 3: 136--69. 2013.

\tit{MAYORGA}, C. et~al. As críticas ao gênero e a pluralização do feminismo:
colonialismo, racismo e política heterossexual. Rev. Estud. Fem. vol.21
n.2 Florianópolis, 2013

\tit{MCKINSEY}\mbox{} \& \versal{COMPANY}. Women Matter: A Corporate Performance Driver, 2007.

\titidem. Unlocking the Full Potential of Women at Work. New
York. 2012a.

\titidem. Women Matter: Making the Breakthrough. 2012b.

\titidem. Women Matter: A Latin American Perspective.
Unlocking Women's Potential to Enhance Corporate Performance. New York.
2013.

\titidem. The Power of Parity: How Advancing Women's Equality
Can Add \$12 Trillion to Global Growth. New York. 2015.

\titidem. Skill shift: Automation and the future of the
workforce. Discussion Paper. May 2018.
Disponível em: \textless{}\emph{https://cutt.ly/1ysoj7U}\textgreater{}.

\titidem; LeanIn.Org. Women in the Workplace, 2018.
Disponível em: \textless{}\emph{https://cutt.ly/7ysokEo}\textgreater{}.

\titidem. Delivering through diversity. January 2018.
Disponível em: \textless{}\emph{https://cutt.ly/MysozeY}\textgreater{}.

\titidem. Closing the tech gender gap through philanthropy
and corporate social responsibility. September 2018.
Disponível em: \textless{}\emph{https://cutt.ly/FysozV1}\textgreater{}.

\tit{MEYER}. D. E. et~al. Vulnerabilidade, gênero e políticas sociais: a
feminização da inclusão social. Rev. Estud. Fem. vol.22 no.3
Florianópolis, 2014.

\tit{MINELLA}, L. S. et~al. Feminismos e publicações: pulsações de teorias e
movimentos. Rev. Estud. Fem. vol.12, Número Especial, Florianópolis
Set./Dez de 2004. .

\titidem. A contribuição da Revista Estudos Feministas para o
debate sobre gênero e feminismo. Rev. Estud. Fem. vol.12 no.spe
Florianópolis, 2004.

\tit{MOND}, N. Construindo espaços transnacionais a partir dos feminismos.
Rev. Estud. Fem. vol.11 n.2 Florianópolis, 2003.

\tit{MONTESINOS}, V. Feministas e tecnocratas na democratização da América
Latina. Rev. Estud. Fem. vol.11 n.2 Florianópolis, 2003.

\tit{MYERS}, A. O valor da diversidade racial nas empresas. Estud. afro"-asiát.
v.25 n.3 Rio de Janeiro, 2003.

\titidem. Executivos negros, um olhar comparativo Brasil X \versal{EUA}.
Disponível em: \textless{}\emph{https://cutt.ly/pysoxBl}\textgreater{}.
2016.

\tit{NIELSEN}, B. \& \versal{NIELSEN}, S. ``Top Management Team Nationality Diversity
and Firm Performance: A Multilevel Study,'' Strategic Management
Journal, vol. 34, 2013.

\tit{OIT}. Women in Business and Management: Gaining momentum. Organização
Internacional do Trabalho 2015.
Disp. em: \textless{}\emph{https://cutt.ly/5ysovwV}\textgreater{}.

\tit{ONU}. As Nações Unidas e as Políticas de Redução da Desigualdade Racial.
Brasília, 3 de março de 2010.
Disponível em: \textless{}\emph{https://cutt.ly/Iysov8n}\textgreater{}.

\tit{ONU MUJERES}. Informe de los Objetivos de Desarrollo del Milenio: gráfica
de género 2012.
Disponível em: \textless{}\emph{https://cutt.ly/yysob8z}\textgreater{}.

\tit{UN WOMEN}. Un women executive director: We're here together tonight to
mobilize a vast, far"-reaching solidarity movement of men and boys. 2014.

\titidem. Training for gender equality and women's empowerment.
Disponível em: \textless{}\emph{https://cutt.ly/8ysonDP}\textgreater{}.

\titidem. Progress of the world's women 2015-2016. Transforming
economies, Realizing rights. 2015.

\tit{ONU MULHERES}. Informe Anual 2015-2016, 2014-2015, 2013-2014, 2012-2013 e
2011-2012. Disponível em: \textless{}\emph{https://cutt.ly/5ysomQ6}\textgreater{}.

\titidem. Princípios do empoderamento das mulheres. \versal{ONU} Mulheres,
Pacto Global Rede Brasileira. 2016.
Disponível em: \textless{}\emph{http://portuguese.weprinciples.org/}\textgreater{}.

\titidem. O futuro que as mulheres querem. \versal{ONU} Mulheres e Rio+20.
2012.
Disponível em: \textless{}\emph{https://cutt.ly/hysoQ0T}\textgreater{}.

\titidem. Modelo de protocolo latino"-americano para investigação de
mortes violentas de mulheres (femicídios/feminicídios). \versal{ONU} Mulheres.
2014.

\titidem. Movimento ElesPorElas (HeForShe) de Solidariedade da \versal{ONU}
Mulheres pela Igualdade de Gênero.
Disponível em: \textless{}\emph{https://cutt.ly/7ysoEtQ}\textgreater{}.

\tit{OTONI}, I. Sem emprego para trans. 2014.
Disponível em: \textless{}\emph{http://www.revistaforum.com.br/digital/132/sem-emprego-para-trans/}\textgreater{}.

\tit{PALMISANO}, S. J. The Globally Integrated Enterprise. Foreign Affairs,
Maio"-Junho de 2006.

\tit{PAOLI}, M.C. O mundo do indistinto: sobre gestão, violência e política.
In: A era da indeterminação. São Paulo: Boitempo, 2007.

\tit{PARROTTA}, et~al. ``The Nexus Between Labor Diversity and Firm's
Innovation,'' Discussion Paper Series, Forschungsinstitut zur Zukunft
der Arbeit, n. 6972, October 2012.

\tit{PASSAPALAVRA}. Dossiê: Feminismo.
Disponível em: \textless{}\emph{https://cutt.ly/5ysoRFI}\textgreater{}.

\tit{PAUL}, L. \& \versal{DONAGGIO}, A. Participação de mulheres em cargos de alta
direção: Relações sociais de gênero, Direito Societário e Governança
Corporativa. Fundação Getúlio Vargas. 2013.

\tit{PERRIN}, F. Boicote às empresas sem negros. 2015.
Disponível em: \textless{}\emph{http://www.ceert.org.br/}\textgreater{}.

\tit{PORTO}, R.M. Consórcio de publicações feministas: a visibilidade do
feminismo e sua divulgação. Rev. Estud. Fem. vol.12, Número especial,
Florianópolis, 2004.

\tit{PURI}, Lakshmi. ``Women in leadership, women empowered, and women at the
helm make a difference''. 2016.
Disponível em: \textless{}\emph{https://cutt.ly/nysoYYm}\textgreater{}.

\tit{REN}, Ting \& \versal{WANG}, Zheng. ``Female Participation in \versal{TMT} and Firm
Performance: Evidence from Chinese Private Enterprises,'' Nankai
Business Review International, vol. 2, n. 2, 2011.

\tit{RIBEIRO}, Djamila. O perfil do empreendedor negro no Brasil. Carta
Capital. 09/12/2015.

\tit{SCHMIDT}, S. P. Como e por que somos feministas. Rev. Estud. Fem. vol.12,
Número especial, Florianópolis, 2004.

\tit{SCHULD}, Kimberly. How the Ford Foundation Created Women's Studies. By:
FrontPageMagazine.com \textbar{} Friday, 2004.

\tit{SEBRAE SP}. Negros são donos de metade das micro e pequenas empresas.
10/09/2013.

\tit{SILVA}, Carmen. Desafios das publicações feministas. Rev. Estud. Fem.
{[}online{]}, vol.21, n.2. 2013

\tit{SMITH}, N. \& \versal{VERNER}, M. Do Women in Top Management Affect Firm
Performance? A Panel Study of 2500 Danish Firms. International Journal
of Productivity and Performance Management 55, n. 7, 2006.

\tit{SPITZ}, C. Cresce parcela de empregadores negros no Brasil em dez anos.
Jornal O Globo, 27/06/2013.

\tit{TAVARES}, R. R. Igualdade de gênero e o empoderamento das mulheres. In:
\versal{BARSTED}, L. L. \& \versal{PITANGUY}, J. O Progresso das Mulheres no Brasil
2003--2010. 2011

\tit{TERJESEN}, Siri \& \versal{SINGH}, Val. ``Female Presence on Corporate Boards: A
Multi"-Country Study of Environmental Context,'' Journal of Business
Ethics, vol.83, n.1, 2008.

\tit{TOMMASI}, L. Culturas de periferia: entre o mercado, os dispositivos de
gestão e o agir político. Política \& Sociedade (Impresso), v. 12, p.\,11-34, 2013b.

\titidem. `Guerra ao tráfico', violência policial e os
limites da democracia brasileira. Revista \versal{IEB}, v. 59, p.\,397-404, 2014b.

\titidem. Juventude, projetos sociais, empreendedorismo e
criatividade: dispositivos, artefatos e agentes para o governo da
população jovem. Passagens: Revista Internacional de História Política e
Cultura Jurídica, v.6, p.\,287-311, 2014.

\titidem; \versal{VELAZCO}, D. J. A produção de um novo regime discursivo
sobre as favelas cariocas e as muitas faces do empreendedorismo de base
comunitária. Revista do Instituto de Estudos Brasileiros, v.0, p.\,15-42,
2013.


\tit{TORCHIA}, Mariateresa et~al. ``Women Directors on Corporate Boards: From
Tokenism to Critical Mass,'' Journal of Business Ethics, v.102, 2011.

\tit{VEIGA}, A. M. Uma viagem transnacional do feminismo: outra lente para a
história. Rev. Estud. Fem. vol.17 n.3 Florianópolis, 2009

\tit{VOROS}, I. Negro se apresenta mais cedo ao mercado de trabalho e sai mais
tarde. 2015.
Disponível em: \textless{}\emph{http://www.redebrasilatual.com.br/cidadania/2015/11/negro-se-apresenta-mais-cedo-ao-mercado-de-trabalho-e-sai-mais-tarde-9284.html}\textgreater{}.

\tit{WILL}. A teoria dos privilégios, uma política da derrota. 2014.
Disponível em: \textless{}\emph{https://cutt.ly/fysoU4A}\textgreater{}.

\tit{WOOSTER}, Martin Morse. The Ford Foundation's International Agenda:
Supports Palestinian, Feminist and Population Control Groups. 2004.

\tit{WORLD ECONOMIC FORUM}, Global Gender Gap Report 2012 e 2014. Cologny,
Switzerland
Disponível em: \textless{}\emph{http://reports.weforum.org/global-gender-gap-report-2014/pressreleases/}\textgreater{}.
\end{bibliohedra}