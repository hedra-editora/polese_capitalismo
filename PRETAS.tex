\textbf{Pablo Polese} é professor de Sociologia no \textsc{ifms}.\,Possui graduação em
Ciências Sociais pela \textsc{unesp} e Doutorado em Serviço Social pela \textsc{uerj}.

\textbf{Racismo, machismo, capitalismo identitário} busca demonstrar os modos como as empresas capitalistas assimilam as
pressões sociais decorrentes das lutas identitárias, em especial as
lutas feminista e negra. Por meio dos mecanismos de mais"-valia relativa,
de desenvolvimento da produtividade e concessões estratégicas para os
trabalhadores, as empresas arquitetam toda uma infraestrutura social e
assim se tornam aptas a integrar as demandas das populações
``periféricas'' e das lutas contra as opressões de raça, gênero e
sexualidade. Ao assim proceder, reforçam suas próprias raízes políticas,
ideológicas e culturais nos locais onde atuam, estreitando os laços
econômicos entre patrões e trabalhadores. Com a dinamização das elites
empresariais, decorrente das pressões das lutas identitárias, ganha novo
fôlego o desenvolvimento capitalista. Este livro trata, portanto, do
modo como os capitalistas lidam com a agenda da diversidade, se
antecipando e convertendo a luta contra o machismo, a homofobia e o
racismo em algo lucrativo.

